%\documentclass[a4paper]{refart}
%\usepackage{listings}
%\usepackage{color}
%\usepackage{amsmath}
%\title{GoSam BLHA Interface How-To}
%\author{T. Reiter, G.~Luisoni}
%\definecolor{lstbg}{rgb}{0.9,0.9,0.9}
%\lstset{basicstyle=\tt,backgroundcolor=\color{lstbg}}
%\begin{document}
%\maketitle
%\tableofcontents

\chapter{The Binoth Les Houches Accord Interface}
\label{sec:blha}


The interface of \gosam{} with a Monte Carlo event generator program 
is based on the Binoth-Les Houches Accord (BLHA)
standard interface.
\gosam{}-2.0 supports both BLHA1~\cite{Binoth:2010xt}
and BLHA2~\cite{Alioli:2013nda}.
Certainly, a dedicated interface without using the BLHA is also possible.

%We assume that \gosam{} has been downloaded and installed using the script
%\texttt{gosam-installer.py} which comes with the distribution. 
%You should ensure
%that the file \texttt{gosam.py} is in your \texttt{\$PATH} variable.


\section{Preparation of the order file}
This step should be done by the Monte Carlo (MC) program. 
We give a generic example of an order file for the process 
$pp\to (Z\to e^+e^-)+$\,jet in both BLHA1 and BLHA2 standards 
in Figs.~\ref{fig:BLHA1} and \ref{fig:BLHA2}.
%in the files \texttt{order1.lh} \texttt{order2.lh}  
\begin{figure}[htb!]
\begin{subfigure}[]{0.49\textwidth}
%\framebox(158,210){%
\fbox{
    \parbox[t][][c]{145\unitlength}{\tt\scriptsize
\# OLP\_order.lh   \\
\# created by MC Sherpowig-1.0\\
\# Process: p p $->$ e+ e- jet\\
Model                    SMdiag\\
CorrectionType           QCD\\
IRregularisation         DRED\\
AlphasPower              2\\
AlphaPower               1\\
MatrixElementSquareType  CHsummed\\
OperationMode            CouplingsStrippedOff\\
SubdivideSubprocess      no\\
\# Subprocesses \\
1 -1 $->$ 11 -11 21\\ 
1 21 $->$ 11 -11 1\\
2 -2 $->$ 11 -11 21\\
...\\
21 -2 $->$ 11 -11 -2\\
\\
\# Process specific GoSam settings\\
\#@ symmetries family,generation}
}
\end{subfigure}
\begin{subfigure}[]{0.49\textwidth}
%\framebox(178,210){%
\fbox{
    \parbox[t]{160\unitlength}{\tt\scriptsize
\# vim: syntax=olp\\
\#@OLP GOSAM 2.0\\
\#@IgnoreUnknown False\\
\#@IgnoreCase False\\
\#@SyntaxExtensions \\
CorrectionType QCD $|$ OK\\
IRregularisation DRED $|$ OK\\
AlphasPower 2 $|$ OK\\
AlphaPower  1   $|$ OK\\           1\\
MatrixElementSquareType CHsummed $|$ OK\\
OperationMode  CouplingsStrippedOff $|$ OK\\
1 -1 $->$ 11 -11 21 $|$ 1 1\\
1 21 $->$ 11 -11 1  $|$ 1 2\\
2 -2 $->$ 11 -11 21 $|$ 1 3\\
...\\
21 -2 $->$ 11 -11 -2 $|$ 1 13\\}
}
\end{subfigure}
\caption{Examples of order and contract files for Z+jet, with BLHA1 standards.}
\label{fig:BLHA1}
\end{figure}  



\begin{figure}[h]
\begin{subfigure}[]{0.49\textwidth}
%\framebox(150,310){%
\fbox{
    \parbox[t][][b]{148\unitlength}{\tt\scriptsize
\#  OLP\_order.lh \\
\# created by MC Sherpowig-2.0\\
InterfaceVersion         BLHA2\\
CorrectionType           QCD\\
IRregularisation         DRED\\
WidthScheme              ComplexMass\\
EWScheme                 alphaGF\\
AccuracyTarget           0.0001\\
DebugUnstable            True\\

AlphasPower              1\\
AmplitudeType ccTree\\
1 -1 $->$ 11 -11 21 \\
...\\
21 -2 $->$ 11 -11 -2 \\
AmplitudeType scTree\\
1 -1 $->$ 11 -11 21 \\
...\\
21 -2 $->$ 11 -11 -2 \\
AmplitudeType Loop\\
1 -1 $->$ 11 -11 21 \\
...\\
21 -2 $->$ 11 -11 -2 \\
\\
AlphasPower              2\\
AmplitudeType Tree\\
1 1 $->$ 11 -11 1 1 \\ 
...\\
21 21 $->$ 11 -11 2 -2\\}}
\end{subfigure}
%\parbox{5\unitlength}{}
\begin{subfigure}[]{0.49\textwidth}
%\framebox(163,310){%
\fbox{
    \parbox[t][][b]{160\unitlength}{\tt\scriptsize
\# vim: syntax=olp\\
\#@OLP GOSAM 2.0\\
\#@IgnoreUnknown False\\
\#@IgnoreCase False\\
\#@SyntaxExtensions \\
InterfaceVersion BLHA2 $|$ OK\\
CorrectionType QCD $|$ OK\\
IRregularisation DRED $|$ OK\\
WidthScheme              ComplexMass $|$ OK\\
EWScheme                 alphaGF $|$ OK\\
AccuracyTarget           0.0001 $|$ OK\\
DebugUnstable            True $|$ OK\\

AlphasPower 1 $|$ OK\\
AmplitudeType ccTree $|$ OK\\
1 -1 $->$ 11 -11 21 $|$ 1 131\\
...\\
21 2 $->$ 11 -11 2 $|$ 1 70\\
AmplitudeType scTree | OK\\
1 -1 $->$ 11 -11 21 $|$ 1 145\\
...\\
21 2 $->$ 11 -11 2 $|$ 1 71\\
AmplitudeType Loop $|$ OK\\
1 -1 $->$ 11 -11 21 $|$ 1 137\\
...\\
21 2 $->$ 11 -11 2 $|$ 1 63\\
\\
AlphasPower 2 $|$ OK\\
AmplitudeType Tree $|$ OK\\
1 1 $->$ 11 -11 1 1 $|$ 1 42\\
...\\
21 21 $->$ 11 -11 -2 2 $|$ 1 106\\}
}
\end{subfigure}
\caption{Order and contract files for Z+jet with BLHA2 standards.}
\label{fig:BLHA2}
\end{figure}  

\paragraph{Remarks}
\begin{itemize}
\item The order file can have any name and any extension.
      We use  the extension \texttt{.lh}
      for order files and \texttt{.olc} for contract files.
\item The options \texttt{WidthScheme, EWScheme} in the BLHA2  example are optional.
\item The options \texttt{SubdivdeSubprocess}  has the following effect
      on the code generation with \gosam{}: if set to 
      \texttt{no}, \gosam{} generates one label per subprocess, if set to
      \texttt{yes} it generates one label per helicity subamplitude
      and therefore \emph{many} labels per subprocess.
      
\item \gosam{} specific settings can be put into commentary lines starting
      with the letter combination `\texttt{\#@}'. This is not part of the
      BLHA standard. The line `\texttt{\#@ symmetries}\dots' restricts the
      helicity subamplitudes being generated to the ones relevant for this
      particular process. 
%      The command \texttt{\#@ filter.nlo NOT(SCALELESS)}
%      excludes scaleless diagrams from the amplitude generation,as they will be zero anyway.
\end{itemize}

\clearpage

\section{Running GoSam}
To run \gosam{} within the MC/OLP setup one can use the following command:\\
%\begin{lstlisting}[language=bash]
{\tt gosam.py --olp --mc=MCname }\\
{\tt    --config=<your-path-to>/gosam.conf order.lh}
%\end{lstlisting}

\paragraph{Remarks}
\begin{itemize}
\item The option \texttt{--olp} is mandatory whenever 
   a BLHA order file is processed.
\item The option \texttt{--mc=MCname} is optional. By specifying the
   name of the Monte Carlo which is the intended partner program,
   \gosam{} can choose some settings simplifying the communication and linking.
   One can either specify \texttt{--mc=MCname} or \texttt{--mc=name/version}.
   Alternatively (and also optional),
   one can put this information into the order file:
%\begin{lstlisting}
{\tt \#@olp.mc.name mypreferredmc}\\
{\tt \#@olp.mc.version 1.0.0}
%\end{lstlisting}
   The short option for \texttt{--mc} is \texttt{-M}.
%   The supported MC names so far are {\tt powhegbox, sherpa}.
%   The interface with Herwig++/Matchbox does not need any extra settings when using BLHA2.
\item The option \texttt{--config} points to the configuration file.
   The latter contains \gosam{} specific settings, such as paths to the 
   libraries, diagram filters, etc.
   
   \attention {\it Is this still necessary with the new installation script?}
   
   If  this option is left out \gosam{} searches for a configuration
   in one of the following locations:
   \begin{itemize}
   \item \gosam{} installation directory,
   \item user's home directory,
   \item current working directory.
   \end{itemize}
   Possible names for default configuration files are \texttt{gosam.in},
   \texttt{gosam.conf} and \texttt{.gosam}. Therefore the easiest way is to simply copy
   \texttt{<your-path-to>/\hspace{0pt}gosam.conf}
   into the current directory and leave out this option.
   It is possible to specify more than one \texttt{--config} option, where the latter
   overwrites already present information from previous use of this
   option. The short form is \texttt{-c}.
\item  The option \texttt{--destination=<dir>} allows 
   to place the generated files into the directory \texttt{<dir>}.
   The short form is \texttt{-D<dir>}.
\item One can specify the name of the contract file which should be written
   using the option \texttt{--output-file=<contractfile>} or simply
   \texttt{-o<contractfile>}.
\item The option \texttt{--force} will overwrite an already existing
   contract file without any warning. 
%It can  however be quite useful for debugging to.
\end{itemize}

\paragraph{The contract file}
From the contract file one can see whether the run of \gosam{} was successful.
If everything went smoothly it should look like the one in Fig.~\ref{fig:BLHA1}
resp. Fig.~\ref{fig:BLHA2}.
All settings are either acknowledged by the word \texttt{OK} or, in case
of a failure, by the word \texttt{error} followed by an error message.


The subprocesses receive an assignment to one or more
labels per subprocess. In the line\\
{\tt 2 -2 $\to$ 11 -11 21 $|$ 1 3}\\
the suffix \texttt{| 1 3}
states that this subprocess has been assigned to \texttt{1}
single label which has the value \texttt{3}. 
Had we set \texttt{SubdivideSubprocess} (keyword in BLHA1)
to \texttt{yes} this line might have looked like\\
{\tt 2 -2 $\to$ 11 11 21 $|$ 4 0 1 2 3}\\
meaning that the subamplitudes
%\footnote{\gosam{} at the moment only splits with respect to helicity subamplitudes.} 
have been assigned to
\texttt{4} labels (which is the first number after the bar) with
the values \texttt{0} to \texttt{3}, each denoting 
an individual helicity subamplitude. These labels will enter the
first argument of the routine \texttt{olp\_evalsubprocess}.
In order to retrieve the full amplitude the calling (MC) program should sum
over the contributions from all labels. Alternatively, it is possible to
sample the different channels by Monte Carlo techniques.

\section{Producing the libraries containing the virtual amplitudes}
The following sequence of commands will generate and compile the matrix
element files:\\
%\begin{lstlisting}[language=bash]
{\tt sh ./autogen.sh --prefix=`pwd`}\\
{\tt make install}
%\end{lstlisting}

Now one should find a subdirectory \texttt{lib/} containing the files
\begin{itemize}
\item \texttt{libgolem\_olp.a} for static linking,
\item \texttt{libgolem\_olp.so} for dynamic linking,
\item \texttt{libgolem\_olp.la} for linking with libtool.
\end{itemize}

\attention {\it is the following still necessary/valid?}

One can set  a variable \texttt{LIBGOLEM} by the following command:\\
%\begin{lstlisting}[language=bash]
{\tt export LIBGOLEM=\ }\\
{\tt "-L`pwd`/lib/ -lgolem\_olp `sh ./config.sh -olibs`"}\\
%\end{lstlisting}
A similar construction would also work inside a makefile:\\
{\tt GOSAM\_ME\_DIR=<path-to-config.sh>}\\
{\tt LIBGOSAM="-L\${GOSAM\_ME\_DIR}/lib/ -lgolem\_olp }\\
{\tt    `sh \${GOSAM\_ME\_DIR}/config.sh -olibs`"}


\section{Calling the interface routines}

For the default settings the call of the interface routines 
will be automatic, so the user does not have to care about the details described below.

We should note however that there are slight differences in naming (underscoring) and calling
conventions (call by reference versus call by value) depending on the
extensions in use. For \texttt{--mc=powhegbox} the extension \texttt{f77}
is automatically included and therefore the underscoring works such that
\texttt{gfortran} used as a Fortran\,77 compiler would not complain.
For all other Monte Carlo programs we follow the C/C++ conventions
(see the file \texttt{olp.h}).

In the following, we will describe BLHA1 and BLHA2 conventions separately, 
even though large parts are identical for the two BLHA versions.

\subsection{BLHA1}

\subsubsection{Initialization}
\lstset{language=[77]{Fortran}}
The generated \gosam{} library is initialized with the call\\
{\tt        call olp\_start("path/to/contract.olc",ierr)}\\
The variable \texttt{ierr} should be declared as an integer. If the contract
file is not found, \texttt{ierr} is set to a negative value. A non-negative
value indicates success.

Please note that calling \texttt{olp\_start} is mandatory even if the contract
file is not present or not read.

\subsubsection{Importing external model files}
If the contract file contains the option
\texttt{ModelFile}, which should point to a SLHA file,
the matrix element code tries to load the parameters from that file.

\subsubsection{Setting options (optional)}
Parameters can be passed by calling \texttt{olp\_option}.\\
{\tt        call olp\_option("name=value",ierr)}

Note that the initialization of derived parameters only works correctly
if the corresponding input parameters are set with \texttt{olp\_option}
\emph{before} \texttt{olp\_start} is called.

Example:
\begin{lstlisting}
       call olp\_option("mZ=91.234",ierr)
       call olp\_option("mW=80.123",ierr)
c  at this point sin(theta\_w) is not up to date.
        call olp\_start("path/to/contract.olc",ierr)
c  now sin(theta\_w) is set consistently
\end{lstlisting}

Some options can be changed at any time; it is instructive to 
look at the file
\texttt{common/model.f90} which contains  the available
parameter names and  their settings.

\subsubsection{Computing the matrix element}

In BLHA1, the routine which returns a value for the matrix element is
\texttt{olp\_evalsubprocess}:
\begin{lstlisting}
       integer ilabel
       double precision moms(5*nlegs)
       double precision mu,params(1)
       double precision res(4)
c      ...
       call olp_evalsubprocess(
      &        ilabel,moms,mu,params,res)
\end{lstlisting}

The first argument, \texttt{ilabel} is one of the labels from the
contract file. The momenta are passed in the argument \texttt{moms},
which has the format
\begin{displaymath}
\mathtt{(/}
E_1, p^x_1, p^y_1, p^z_1, m_1,
E_2, p^x_2, p^y_2, p^z_2, m_2, \ldots
E_N, p^x_N, p^y_N, p^z_N, m_N
\mathtt{/)}
\end{displaymath}
The momenta are expected to be given in physical (in-out) 
kinematics: $p_1+p_2=p_3+\ldots+p_N$.
The components are in units of GeV.

The argument \texttt{mu} is the renormalisation scale $\mu$ (not $\mu^2$!)
in GeV. The argument {\tt params} is an array of which the first argument is
$\alpha_s(\mu)$. Any further array entries are ignored within BLHA1\footnote{
Passing more than one parameter is implemented by the \texttt{Parameters}
option in the order file, which is  not part of the BLHA1 standard.}.

The last argument is an array of length four which is filled by the subroutine, 
containing the result of the evaluation. The entries have as a unit some
power of GeV ($\mathrm{GeV}^{(4-N)}$).
\begin{align}
\label{eq:res}
\mathcal{M}_B^\dagger\mathcal{M}_B&=\mathtt{res(4)}\nonumber\\
2\mathrm{Re}\left(\mathcal{M}_B^\dagger\mathcal{M}_V\right)&=
\frac{(4\pi)^\varepsilon}{\Gamma(1-\varepsilon)}\left(
\frac{\mathtt{res(1)}}{\varepsilon^2}
+\frac{\mathtt{res(2)}}{\varepsilon}
+\mathtt{res(3)}
\right)
\end{align}
This means that the coefficients \texttt{amp(1:3)} contain
an explicit factor of $\alpha_s(\mu)/(4\pi)$.

\subsubsection{Finalize (optional)}
There is also a routine \texttt{olp\_finalize} which is only needed
if the client code needs to call \texttt{olp\_start} more than once, e.g.
\begin{lstlisting}
       do i=1,max_i
          write(line,'(A3,F6.3)') "mZ=", mZ(i)
          call olp_option(line,ierr)
c Need olp_start to update dependent parameters
          call olp_start(name,ierr)
c         ...
          call olp_finalize()
       enddo
\end{lstlisting}

\subsection{BLHA2}

{\it still to be completed/improved ...}
%\end{document}
\subsubsection{Initialization}

\subsubsection{Importing external model files}

\subsubsection{Setting options (optional)}
Parameters are now passed by the subroutine \\
{\tt OLP\_SetParameter(char* para, double* re, double* im, int* ierr)}


\subsubsection{Computing the matrix element}

In BLHA2, the routine which returns a value for the matrix element is\\
{\small {\tt OLP\_EvalSubProcess2(int* i,double* pp,double* mu,double* rval,double* acc)}}


The arguments are:
\begin{itemize}
\item i: pointer to a (one element) array with the label of the subprocess as given in the contract file
\item pp: pointer to an array of momenta, conventions $(E_j,k_j^x,k_j^y,k_j^z,M_j)$
\item mu: pointer to the renormalisation scale 
\item rval: pointer to an array of return values
\item acc: pointer to a one element array with the outcome of the 
OLP internal accuracy check 
\end{itemize}


The last argument is an array of length four which is filled by the subroutine, 
containing the result of the evaluation, as specified in eq.~(\ref{eq:res}).
The default settings for the prefactor can be changed using the option {\tt nlo\_prefactor}, 
see section \ref{chp:setup-of-a-process}.

For more details concerning the BLHA2 conventions we refer to \cite{Alioli:2013nda}.
