\documentclass[11pt,a4paper]{refrep}
\usepackage{xspace}
\usepackage[final]{remark}
\usepackage{listings}
\usepackage{amsmath,amssymb}
\usepackage[pdftex]{hyperref}

\input lstform
\input lstolp
\newcommand{\golemversion}{{1{.}0}}
\newcommand{\golem}{{\tt GoSam}\xspace}
%\newcommand{\golemv}[1][\golemversion]{{{\tt Golem}~\golemversion}\xspace}
\newcommand{\golemv}[1][\golemversion]{{\tt GoSam}\xspace}
\newcommand{\golemVC}{{\tt golem95}\xspace}
\newcommand{\packagename}{{gosam-\golemversion}}
\newcommand{\hepforge}{{\sc HepForge}\xspace}

\newcommand{\qgraf}{{\tt QGraf}\xspace}
\newcommand{\form}{{\tt Form}\xspace}
\newcommand{\python}{{\tt Python}\xspace}
\newcommand{\fortranXC}{{\tt Fortran\,95}\xspace}
\newcommand{\fortranMMIII}{{\tt Fortran\,2003}\xspace}
\newcommand{\pjfry}{{\tt PJFry}\xspace}
\newcommand{\haggies}{{\tt haggies}\xspace}
\newcommand{\samurai}{{\sc Samurai}\xspace}

\newcommand{\contl}{{\ensuremath{\hookrightarrow}}}
\newcommand{\fmslash}[1]{{#1}\!\!\!/}
\newcommand{\pslash}[1][{}]{\ensuremath{\fmslash{p}_{#1}}}
\newcommand{\kslash}[1][{}]{\ensuremath{\fmslash{k}_{#1}}}

\title{\golemv Manual}
\author{The GoSam Collaboration}
\date{Version \today}

\begin{document}
\hypersetup{%
	pdfborder={0 0 0},%
	pdftitle={GoSam \golemversion{} Manual},%
	pdfauthor={Thomas Reiter, The GoSam Collaboration},%
	pdfsubject={High Energy Physics/Higher Order Corrections},%
	pdfkeywords={NLO, automatization},%
	pdfdisplaydoctitle
}

\begin{fullpage}
\maketitle
\tableofcontents
\end{fullpage}

\chapter{Introduction}
\section{Synopsis}
\golemv is a general one-loop evaluator for matrix elements.
The program produces \fortranXC code from a given process
description by evaluating Feynman diagrams and translating
the associated one-loop diagrams into a numerical representation
of the numerator such that it can be evaluated and reduced
numerically with either
the \golemVC library~\cite{Golem95:2008,Cullen:2011kv}
or \samurai~\cite{Mastrolia:2010nb}
or \pjfry~\cite{Yundin:2011,Fleischer:2010sq}.

\section{Conventions}
In this manual, shell commands are indicated
by lines starting with a dollar sign (\texttt{\$})
and are given for the \texttt{bash} shell only.
Lines that are broken for type setting reasons and should
continue the previous line(s) start with a \contl.

\python program fragments are denoted by the `\texttt{>>>}' and
`\texttt{...}' (for continuation lines) prompts.

\chapter{Setup}
\section{Prerequisites}
The generation of matrix element code using \golemv can be understood
as a three step process, although the three steps are not necessarily
obvious to the user. In principle, each step could be run on a different
machine and the programs listed below only need to be available during
the respective step.
\begin{enumerate}
\item During the {\bf setup of the process directory} \python
and \qgraf need to be installed. This phase is initiated by
running \texttt{gosam.py}
or any user written \python script of similar functionality.
\item During the {\bf code generation} only \form and \haggies are run.
This phase is initiated by running \texttt{make source}.
\item During the {\bf compilation and running} of the matrix element
a \fortranXC compiler and the chosen reduction libraries need to be installed.
At the level of the matrix element, this phase is initiated
by running \texttt{make compile}. Please note that running
\texttt{make compile} will invoke \texttt{make source} if the latter
has not been run successfully before that.
\end{enumerate}

Before running the \golemv package, Please ensure that
the following programs are available on your system.
The numbers indicate during which phase of the code generation
the tools will be required.

\marginlabel{\qgraf (1)} \qgraf~\cite{Nogueira:1991ex}
is required in version 3{.}1 or
higher and can be downloaded from
\url{http://cfif.ist.utl.pt/~paulo/qgraf.html}.

\marginlabel{\python (1)} This program has been tested with \python{}
versions 2{.}6, 2{.}7 and 3{.}1.
%It is currently not guaranteed to work with any other version. 
%Since \python{} 3{.}x introduces many new
%language features there might be conflicts which have to be resolved before
%this program is ready to be run with versions of \python{} 3{.}x.
Please see also \url{http://python.org}.

\marginlabel{\form (2)} You will need \form~\cite{Vermaseren:2000nd}
version 3{.}3 (build
11--aug--2010 or later).
The most recent version is available from
\url{http://www.nikhef.nl/~form/}.

\marginlabel{\haggies (2)} The code generator \haggies{}~\cite{Reiter:2009ts}
is included in the \golemv{} distribution already.
Alternatively, it can also be obtained separately from the URL
\url{http://sourceforge.net/projects/haggies/}.
\haggies requires \texttt{Java} in version~1{.}5 or higher.
The current version of \golemv requires \haggies{} in version 1{.}1 or
higher.

\marginlabel{\golemVC/\samurai/\pjfry (3)}
For one-loop calculations, at least one of these three libraries is
required. If the program is used for the extraction of the $R_2$ term
only, the libraries are not required.
\begin{itemize}
\item \golemVC can be downloaded from
\url{http://projects.hepforge.org/golem/}.
\item \samurai can be downloaded from
\url{http://projects.hepforge.org/samurai/}.
\item \pjfry can be obtained via \texttt{git} from
\url{https://github.com/Vayu/PJFry/}.
\end{itemize}

\marginlabel{\texttt{refrep.cls} (3)} The documentation
is based on the \LaTeX-class \texttt{refrep}, which appears
not to be in the default installation of all \LaTeX{}
distributions. It can be downloaded from \url{http://www.ctan.org/}
as part of the \texttt{refman} package.
This file is only needed if one intends to run \texttt{make doc},
which generates some documentation for the matrix element.

\attention Please note that these programs might have
license policies which are different from the license
applying to \golemv. The authors of \golem do \emph{not}
take any responsibility for any problems related to the
above mentioned software packages.

\section{Download}
The \golemv{} source code can be downloaded either via \texttt{subversion}
or via HTTP download.

\subsection{Subversion}
You can check-out a working copy of the repository with the command
\begin{example}
\$ svn co http://svn.hepforge.org/gosam/trunk/ gosam-\golemversion{}
\end{example}
This will create a folder \texttt{gosam-\golemversion} in your current directory.
Authenticated users can use the URL
\begin{example}
\$ svn co svn+ssh://svn.hepforge.org/hepforge/svn/\\
   \contl{}gosam/trunk/ gosam-\golemversion{}
\end{example}
to gain read and write access to the project files.

\subsection{HTTP Download}
Under \url{http://www.hepforge.org/downloads/gosam/}
you can download the sources of \golemv{} using a web browser or a
HTTP client like \texttt{wget} or \texttt{curl}.
If you received \golemv{} as a tar-ball you can unpack it using the command
\begin{example}
\$ tar xzvf \packagename{}.tar.gz
\end{example}

\section{Installation}
\golemv{} is distributed as a \python package.
The installation of the source package is done by running
the setup script. One of the following scenarios will be
encountered most probably:
\begin{itemize}
\item If the \python installation resides in \texttt{/usr} or
   \texttt{/usr/local}
   and the user has super-user privileges:
\begin{example}
\$ sudo python setup.py install
\end{example}

\item If the user wants (or has to) maintain an alternative installation
   path for \python modules. $\mathtt{<XXXX>}$ here denotes the name of the
   alternative installation tree:
\begin{example}
\$ python setup.py install --prefix=<XXXX>
\end{example}
   The \texttt{prefix} option can also be permanently set in the
   user's \texttt{pydistutils} config file\footnote{%
   On Unix systems and MacOS this file is called
   \texttt{\$HOME/.pydistutils.cfg}, on Windows it is
   \texttt{\%HOME\%\textbackslash{}pydistutils.cfg}.} by adding the
   following lines.
\begin{example}
[install]\\
prefix=<XXXX>
\end{example}
   After successful installation the user should also update the
   environment variable:
\begin{example}
PATH=\$PATH:<XXXX>/bin\\
\end{example}

   For Bourne shell compatible shells (bash, zsh, \dots), the \verb!PATH! environment
   variable can be permanently changed by adding the following lines to \verb!~/.profile!:
\begin{example}
   PATH="\$PATH:<XXXX>/bin"\\
   export PATH
\end{example}

   For csh compatible shells (tcsh, \dots), the following line need to be added to \verb!~/.cshrc!:
\begin{example}
   setenv PATH "\$PATH:<XXXX>/bin"\\
\end{example}


To enable this environment variables change, you need to run \verb!source ~/.profile! or
\verb!source ~/.cshrc! and \verb!rehash!, or re-login.

\end{itemize}
%  A more complete tutorial on the installation of \python packages
%  can be found at \url{http://docs.python.org/install/}.

\section{Directory Structure}
The \golemv{} source directory has the structure as described below:

\marginlabel{\texttt{doc/}} This directory contains the documentation
and example setup files. You can run \texttt{make} in this directory
to generate the document \texttt{refman.pdf}; this is the document you
are currently reading.

\marginlabel{\texttt{models/}} For each implemented model this directory
contains the \qgraf model file (no extension), a \form interface
(\texttt{*.hh}) and a \python module (\texttt{*.py}). Currently,
only the Standard Model (\texttt{sm}) is distributed with \golemv.
A second version of the Standard Model (\texttt{smdiag}) implements
diagonal flavour structure ($V_{\text{CKM}}=\mathsf{diag}\{1,1,1\}$)
The structure of the model files is discussed in more detail in
Chapter~\ref{sec:modelfiles}. Model files for the MSSM based on 
LanHEP~\cite{Semenov:2010qt} and FeynRules/UFO~\cite{Degrande:2011ua}  
can be found in the directory 
\texttt{examples/model/}.

\marginlabel{\texttt{templates/}} Contains templates for the creation
of the files in the process directory. The contents are transformed
by the class \texttt{golem.util.parser.Template} and its subclasses
in \texttt{golem.templates.*}. The translation of the templates is
controled by the file \texttt{templates.xml} of the same directory.

\marginlabel{\texttt{src/python/}} All model independent \python modules
can be found in this directory tree.

\marginlabel{\texttt{src/form/}} Here one finds all \form files
which are not part of the template.

\marginlabel{\texttt{build}} This directory is created during
building and installation of this package by running \texttt{setup.py}.
The files in this directory are of temporary nature and can be safely
removed.

\marginlabel{\texttt{dist}} This directory is created by running
\texttt{setup.py} with the \texttt{sdist} or \texttt{bdist} command
and contains the distributable package files.
To create a tar-ball from the working copy, Please run
\begin{example}
\$ python setup.py sdist --formats=gztar
\end{example}
For more information
please run
\begin{example}
\$ python setup.py --help-commands
\end{example}

\marginlabel{\texttt{examples}} This directory contains some simple example
processes for which \golemv{} has been compared to the literature.

\marginlabel{\texttt{olp}} Files in this directory are used by
\texttt{gosam.py --olp}, which is \golemv's implementation of the
Les Houches interface for one-loop programs~\cite{Binoth:2010xt}.

%%%%%%%%%%%%%%%%%%%%%%%%%%%%%%%%%%%%%%%%%%%%%%%%%%%%%%%%%%%%%
\chapter{Setup of a Process}
\section{Introduction}
\label{chp:setup-of-a-process}
This chapter provides a step by step guide how to
set up a new process. 

In order to generate the matrix element for a given process one has to
create a process specific setup file, which we call {\em process card}. 

The syntax of this file is
closely related to that of Java \texttt{.properties} files.
The detailed syntax and a 
full list of options are given in 
Appendix~\ref{chp:appendix-template.in}.
%\seealso{Appendix \ref{chp:appendix-template.in}}
Here we first give a commented example, which should be sufficient to explain the 
most important features of a {\em process card}. 



\section{Example: \texorpdfstring{$e^+e^-\rightarrow t\bar{t}$}{e+e- to tt-bar}
at NLO in QCD}

It is recommended to generate and modify a template file for the 
process card instead of
starting from scratch. This can be done by invoking the shell command
\begin{example}
\$ gosam.py --template eett.in
\end{example}
This would generate the file \texttt{eett.in} with some documentation
for all accepted options. The options are filled with some default values,
which can be set in a global configuration file.
The script will search\footnote{in this order} in the \golemv{} directory,
in the user's home directory and in the current working directory for a file
named `\texttt{.golem}' or `\texttt{golem.in}'. Such a file can be generated
with the following command:
\begin{example}
\$ golem-config.py > golem.in
\end{example}

In the following brief tutorial it is assumed that the process
$e^+e^-\rightarrow t\bar{t}$ should be calculated to order
$\mathcal{O}(\alpha)\mathcal{O}(\alpha_s)$ (virtual corrections);
the tree-level process is of order $\mathcal{O}(\alpha)$. We neglect
the exchange of a $Z$ or a Higgs boson and treat the electron massless.
The output directory is assumed to be in the relative path
\texttt{eett}.

\begin{lstlisting}[gobble=3,%
     numbers=left,caption={{\tt eett.in}},%
     basicstyle=\ttfamily]
 1 process_path=eett
 2 process_name=eett
 3 in=    e+, e-
 4 out=   t, t~
 5 model= sm
 6 order= gs, 0, 2
 7 
 8 qgraf.options=nosnails,notadpoles,onshell
 9 qgraf.verbatim=\
10    true=iprop[Z, 0, 0];\n\
11    true=iprop[H, 0, 0];
12 zero=me
13 one=gs,e
14
15 extensions=samurai
16 samurai.fcflags=`pkg-config --cflags samurai`
17 samurai.ldflags=`pkg-config --libs samurai`
\end{lstlisting}

The above lines are discussed one by one. The line numbers on the left
are only included for better readability and \emph{must not} be included
in the setup file.
\begin{enumerate}
\item[1] The option \texttt{process\_path} specifies the directory
to which all generated files and directories are written.
The directory which is specified here must already exist.
\\Specification of a process path is mandatory.
\item[2] Setting a process name is optional but recommended. All module
names will be prefixed with the  process name (e.g. \texttt{precision}
$\to$ \texttt{eett\_precision}). This will avoid name conflicts if at
a later stage more than one matrix elements are linked into one
executable.
\item[3--4] The options \texttt{in} and \texttt{out} specify the particles
of the initial and final state. The particle names must be defined in the
selected model file. As the model files usually define mnemonics for the
particle names there might be several ways of specifying the same process.
Instead of `\lstinline[basicstyle=\ttfamily]{e+}'
one could have written `\lstinline[basicstyle=\ttfamily]{ep}'
or `\lstinline[basicstyle=\ttfamily]{positron}'.
\seealso{Appendix~\ref{chp:model-files}}
For a complete list of
alternative particle names please refer to the documentation of the
according model file.
\\Specifying \texttt{in} and \texttt{out} particles is mandatory.
\item[5] The option \texttt{model} specifies which model files should
be used in order to generate and evaluate the diagrams. 
%It is only possible to refer to files that are located in
%\texttt{\$GOLEMPATH/models}.
\\This option is mandatory.
\item[6] The option \texttt{order} is a comma separated list with
three entries. The first entry specifies a symbol that denotes a coupling
constant. In the Standard Model file \texttt{sm} the only two possibilities
are `\lstinline[basicstyle=\ttfamily]{gs}' for the strong coupling constant
$g_s$ and `\lstinline[basicstyle=\ttfamily]{e}' for the electro-weak coupling.
The second number is the power of the chosen coupling constant for the
tree-level diagrams and the third parameter specifies the power of that
coupling constant for the one-loop diagrams.
\attention Note that the numbers
refer to the powers in the diagrams of the amplitude
rather than the squared amplitude. In the above example the
string `\lstinline[basicstyle=\ttfamily]{gs, 0, 2}' specifies that
the tree-level diagrams should be of order $g_s^0$ and the one-loop
diagrams should be of order $g_s^2$ and an unspecified
power of $e$ in both cases. 
If there is no tree level, i.e. the process is loop induced, 
the keyword \texttt{NONE} should be put as second item in the list,
instead of the tree level power of the coupling.

The value of this option is
translated into a \texttt{vsum} constraint in the file \texttt{qgraf.dat}.
\\This option is mandatory.
\item[8--11] The option
\texttt{qgraf.options} creates the line
`\lstinline[basicstyle=\ttfamily]{options=}\dots;' in the file
\texttt{qgraf.dat}. The value of the option \texttt{qgraf.verbatim} is
passed verbatim to the file \texttt{qgraf.dat}.
In our example we specify that loops of size one and self-energy insertions
at external lines should be omitted in the graph generation. Lines
9--11 suppress the generation of diagrams containing Higgs and $Z$ bosons.
\attention As these commands are passed verbatim to \qgraf no mnemonic names
are allowed here, e.g. the Higgs particle has to be denoted by
`\lstinline[basicstyle=\ttfamily]{H}' and cannot be replaced by
`\lstinline[basicstyle=\ttfamily]{h}'.
For a complete list of available options, Please consult the
\qgraf manual. For a complete list of particle names see 
Appendix \ref{sec:model-files:sm} resp. the
documentation of the model file.
\\These options can be omitted.
\item[12--13] The keywords \texttt{zero} and \texttt{one} specify
a set of symbols that should be treated as zero (resp. one). These
simplifications are applied at the symbolical level. Only symbols
that appear in the \form interface of the model file should be
specified here (masses, couplings, CKM-matrix elements, etc).
In the example we specify the electron mass
`\lstinline[basicstyle=\ttfamily]{me}' to be zero and we do not keep
the coupling constants in the calculation explicitly ($g_s=e=1$).
\\These options can be omitted.
\item[15] The option \texttt{extensions} contains a list of extensions
to the core of the program.
\item[16--17] For each extension one can add options of the form
\textit{extension}{.}\textit{name}. Currently the program is scanning
for options of the form \textit{extension}{.}\texttt{ldflags} and
\textit{extension}{.}\texttt{fcflags}. These options are copied
to the contens of the according variables (\texttt{FCFLAGS} and
\texttt{LDFLAGS}) in the makefiles.
\end{enumerate}

In order to populate the specified process directory with files
one invokes
\begin{example}
\$ gosam.py eett.in
\end{example}

\section{Process Directory Structure}
After running \texttt{golem} with an appropriate setup file the
process directory contains a number of files which are described below.

\marginlabel{\tt codegen/} This directory contains files which are only
relevant for code generation. These files will therefore not be included
in a tar-ball created with \texttt{make dist}.

\marginlabel{\tt common/} Fortran files which are common to all helicity
amplitudes and to the constructed matrix element code. This directory
is always compiled first.

\marginlabel{\tt doc/} Contains all files (apart from
\texttt{pyxotree.tex} and \texttt{pyxovirt.tex}) which are
necessary for creating
\texttt{doc/process.ps}, which lists all Feynman diagrams of this process, 
together with colour and helicity info.

\marginlabel{\tt helicity*} This directory contains all files for a specific
helicity amplitude. The labeling of the helicities can be found in
\texttt{doc/process.ps}. Before invoking \texttt{make source}, 
this directory only contains the makefiles.

\marginlabel{\tt matrix} This folder contains the code to combine
the helicity amplitudes into a matrix element. Here one also finds
the test program \texttt{test.f90}. This folder is always compiled last.

\marginlabel{\tt model,model.hh,model.py} The files from the \texttt{model/}
directory of \golemv. The original files are renamed, e.g.
\texttt{sm} $\to$ \texttt{model}, \texttt{sm.py} $\to$ \texttt{model.py}
and \texttt{sm.hh} $\to$ \texttt{model.hh}.

\marginlabel{\tt diagrams-[01].hh} The diagram files generated
by \qgraf.

\marginlabel{\tt config.sh} This script facilitates linking with external
programs. For details, run
\begin{example}
\$ sh ./config.sh -help
\end{example}

\marginlabel{\tt process.hh} contains the process dependent definitions
for \form. This file is used by \texttt{golem.frm} to generate the expressions
for each diagram in every helicity configuration.

\marginlabel{\tt process.dat} contains the on-shell conditions, the number of
incoming particles and an expression for momentum conservation. This
file is needed by the program \texttt{golem-analyzer.py}.

\marginlabel{\tt func.txt} Defines dependencies between parameters of
the model files.

\marginlabel{\tt Makefile.conf} This files contains the settings
which might need to be modified by the user. Please check the contents
of this file if you have trouble running the makefiles.

\marginlabel{\tt Makefile\\Makefile.source} These two files are part of
each directory. \texttt{Makefile.source} is used when calling
\texttt{make source}. Running \texttt{make} from the process
directory will pass through all subdirectories. The following
targets of \texttt{make} are recommended for direct use:
\begin{description}
\item[\texttt{help}] lists all major targets.
\item[\texttt{source}] generate source files, mainly \fortranXC files.
\item[\texttt{compile}] compile the \fortranXC sources.
\item[\texttt{dist}] create a tar-ball of the source files.
\item[\texttt{clean}] remove object files and intermediate files.
\item[\texttt{very-clean}] remove files including targets of
                                \texttt{make source}.
\item[\texttt{doc}] create various documents related to the process.
      To obtain a description of the \emph{topologies}, you need to run
      \texttt{source} before \texttt{make doc}.
\end{description}

\section{Code Generation and Compilation}
The \fortranXC code is generated by the command
\begin{example}
\$ make source
\end{example}
and can be compiled using
\begin{example}
\$ make compile
\end{example}
Please note that the \texttt{compile} target invokes
the \texttt{source} target if necessary.

A simple test program,
which gives the value of the amplitude at a randomly generated phase space point,
can be found in the directory \texttt{matrix/},
in order to compile and run it, type
\begin{example}
\$ cd matrix
\$ make test.exe
\$ ./test.exe
\end{example}
The program will generate a file \texttt{\_debug.xml},
which, depending on the settings contains the values of helicity amplitudes
and diagrams for a set of phase space points.

\subsection{Customization}
\paragraph{Runtime Parameters.}
Many settings can be changed without recompiling the code, by
creating and modifying the file \texttt{matrix/param.dat}.
This file has a very simple format:
\begin{itemize}
\item Lines starting with a comment character (`!', `\#', `;')
      in the first column and blank lines are ignored.
\item All other lines have the format
\begin{example}
\textit{name} = \textit{float}\\
\# \textrm{or}\\
\textit{name} = \textit{float}, \textit{float}
\end{example}
      where the first line defines a real number and the second
      line defines a complex number, and \textit{name} is a 
      parameter de.
\item Whitespace is ignored but must not appear inside names or
      literals. Physical lines can not be continued nor can
      multiple entries appear on one line.
\end{itemize}
The list of recognized names can be found in the file
\texttt{common/model.f90}. In addition there are some model
independent parameters:
\smallskip

\begin{tabular}{l@{\quad}p{0.7\textwidth}}
   \texttt{samurai\_scalar} & selects a library of scalar integrals
   (see \samurai documentation).\\
   \texttt{samurai\_test} & sets a method to detect unstable points
   (see \samurai documentation). \\
   \texttt{samurai\_verbosity} & sets the verbosity level of
   \samurai; it should be set to zero in a production environment
   (see \samurai documentation).\\
   \texttt{renormalisation} & An integer number indicating if no
   renormalisation (0) or $\beta$-function renormalisation (1,
   QCD only) should be applied. Other values are reserved for future
   extensions. \\
   \texttt{gauge\textit{i}o} & for the external vector particle with
   index~$i$ (e.g. \texttt{gauge1o}, \texttt{gauge2o}\ldots),
   if not defined as a constant. \\
   \texttt{gauge\textit{i}z} & as \texttt{gauge\textit{i}o}.
   The polarisation vector is transformed into
   \begin{displaymath}
   \varepsilon^\mu(k_i)\to\mathtt{gauge\mathit{i}o}\cdot\varepsilon^\mu(k_i)
      + \mathtt{gauge\mathit{i}z}\cdot k_i^\mu
   \end{displaymath}
   This allows for a quick check of gauge invariance.
\end{tabular}
Furthermore, all model constants that have not been specified as zero or one
can be set in this way. 
%In the builtin Standard Model file (\texttt{sm}) one
One can can re-set, for example, the value for the Higgs mass using the entry
\begin{example}
mH = 124.5
\end{example}
Please note that upper and lower case letters have to be distinguished
and that the names need to be spelled exactly as defined in \texttt{model.py}.

\paragraph{Compile Time Parameters.}
Other configuration options can be found in the file \texttt{common/config.f90}
but require the recompilation of the source code
(\texttt{make clean; make compile}).
\smallskip
Examples of options contained in \texttt{config.f90} are
\begin{maxipage}
\begin{tabular}{lp{0.6\textwidth}}
\texttt{ki} & the floating point kind used throughout the calculation.\\
\texttt{debug\_lo\_diagrams} & controls if information about the
    tree level diagrams is written to the output file.\\
\texttt{debug\_nlo\_diagrams} & controls if information about the
    loop-diagrams is written to the output file.\\
\texttt{include\_eps\_terms} & controls if
    terms of order $\varepsilon$ multiplying
    poles are taken into account.\\
\texttt{include\_eps2\_terms} & controls if
    terms of order $\varepsilon^2$ multiplying
    double poles are taken into account.\\
\texttt{include\_color\_avg\_factor} & controls if the color averaging
    factor for inital state partons is multiplied to the final result.\\
\texttt{include\_helicity\_avg\_factor} & controls if the helicity averaging
    factor for inital state particles is multiplied to the final result.\\
\texttt{include\_symmetry\_factor} & controls if the symmetry
    factor for identical final state particles
    is multiplied to the final result. \\
\texttt{use\_sorted\_sum} & controls if the diagrams are summed using
    the algorithm Malcolm~\cite{Malcolm:1970}, which reduces the error
    accumulated in presence of large cancellations.
\end{tabular}
\end{maxipage}

\section{Drawing the Feynman Diagrams}
In order to print out the diagrams the makefile contains the target
\texttt{doc} which produces the file \texttt{process.ps}.
We use \LaTeX{} plus the package \textsf{axodraw}~\cite{Vermaseren:1994je}
to create the graphical representation.

The layout of the diagrams is determined by the algorithm used in
\textsf{feynMF}~\cite{Ohl:1995kr}, modelling the propagators by springs.
The implemented algorithm works in two steps: first, the topology is
disentangled by ordering the external legs such that the diagram can
be drawn as a planar graph. The coordinates $e_k$
of the external legs are
fixed along a contour around the drawing area.\footnote{Currently,
this contour is chosen as an ellipse but in principle any convex
shape could be used.}
In a second step the remaining degrees of freedom, the coordinates
of the vertices $v_i=(x_i, y_i)$, are fixed by minimizing the Lagrangian
\begin{multline}
L(v_1, \ldots, v_n; e_1, \ldots, e_N) =\\
 \frac14\sum_{i,j=1}^n t_{ij}\left(v_i-v_j\right)^2
+\frac12\sum_{i=1}^n\sum_{k=1}^N\lambda_{ik}\left(v_i-e_k\right)^2
\end{multline}
Here, $n$ is the number of vertices and $N$ is the number of external
legs.
Minimization of the Lagrangian leads to a system of linear equations, which
can easily be solved.
\begin{align*}
&\frac{\partial L}{\partial v_r}=0\\
\Leftrightarrow&
 \frac12\sum_{i,j=1}^n t_{ij}\left(v_i-v_j\right)
     \cdot\left(\delta_{ir}-\delta_{jr}\right)
+\sum_{i=1}^n\sum_{k=1}^N\lambda_{ik}\left(v_i-e_k\right)
     \cdot\delta_{ir}=0\\
\Leftrightarrow&
M_{rj}v_j\equiv
 \sum_{j=1}^n t_{rj}\left(v_r-v_j\right)
+\left(\sum_{k=1}^N\lambda_{rk}\right)v_r
=\sum_{k=1}^N\lambda_{rk}e_k
\end{align*}
In the last step we used the symmetry of $t_{ij}$.
The matrix $M$ can be written as
\begin{equation}
M_{rc}=\left\{\begin{array}{ll}
\left(\sum_{i\neq r}t_{ri}\right)+
\left(\sum_{k=1}^N\lambda_{rk}\right),&
r=c\\
-t_{rc},&\text{otherwise}
\end{array}\right.
\end{equation}

The symbol $t_{ij}$ is the sum of the spring constants of all
propagators connecting vertices $i$ and $j$; similarly, $\lambda_{ik}$
is the spring constant of the leg $k$ if it is connected to vertex $i$
and zero otherwise.

\section{Import of Model Files}

Examples about how to import model files can be found in the subdirectory 
 \texttt{examples}.

\subsection{Import from FeynRules}
A model description in the UFO~\cite{Degrande:2011} format consists of a \texttt{Python} package
stored in a directory. In order to import the model into \golem{} one needs
to set the \texttt{model} variable specifying the keyword \texttt{FeynRules}
in front of the directory name, where we assume that
the model description is in the directory \texttt{\$HOME/models/MSSM\_UFO}.
\begin{example}
model= FeynRules,\$HOME/models/MSSM\_UFO
\end{example}

\subsection{Import from LanHEP}
In order to use model files generated by LanHEP the following steps
have to be taken:
\begin{enumerate}
\item When generating the tables using LanHEP, one should include the
   following option to ensure that the generated tables have the correct
   headings\footnote{\golemv{} relies on the column names rather than
   some specific order.}. The number of spaces in the column headers are
   irrelevant as long as the columns are wide enough to contain the
   respective values.
\begin{example}
   prtcformat\\
      fullname: '  fullname  ',\\
      name:     '  name   ',\\
      aname:    '  aname  ',\\
      spin2:    '  spin2  ',\\
      mass:     '  mass  ',\\
      width:    '  width  ',\\
      color:    '  color  ',\\
      aux:      '  aux  ',\\
      texname:  '      texname      ',\\
      atexname: '     atexname      ',\\
      pdg:      '  pdg   '.
\end{example}
\item If the model file is not already equipped with pdg codes
   the user might want to use the \verb!prtcprop! command in
   LanHEP to add the relevant codes.
\item In the setup file, one needs to specify the model as a pair
   of path and integer number. If the table files are under the directory
   \texttt{lanhep/ued/} in the tables \texttt{func7.mdl}, \texttt{lgrng7.mdl},
   \texttt{prtcls7.mdl} and \texttt{vars7.mdl}, the correct statement in
   the setup file would be
\begin{example}
   model=lanhep/ued, 7
\end{example}
\item The use of user defined functions (\texttt{external\_func} in LanHEP)
   requires an adaption of the file \texttt{codegen/haggies-l0.in}. If one
   wants to use the function \texttt{double foo(double,double)} the
   following line sould be added.
\begin{example}
@define mdlfoo : real, real -> real =\\ "foo(\%2\$s, \%3\$s)";
\end{example}
   The function also needs to be declared in \texttt{codegen/functions.out}
   in the subroutine \texttt{init\_functions}
\end{enumerate}

\section{Handling Big Processes}
Although the default settings should work for most cases, very big processes
in terms of the number of diagrams and the size of the expressions can cause
the compiler to become very slow or even to crash. In this section we discuss
solutions which can help to reduce the load for the compiler and to speed-up
the code generation. It should be mentioned that some of these measures can
have a negative impact on the runtime efficiency of the generated code.

\subsection{Grouping of Tree Level Diagrams}
By default the expressions of all tree-level diagrams are grouped into one
file. This has the advantage that subexpressions which appear in several
tree-level diagrams can be reused across the amplitude. In some cases
it can happen that the sum of all terms of the tree-level diagrams is too big
to be compiled in one subroutine. In this case it is recommended to set
the option \texttt{group} to \texttt{false}.

\subsection{Computation of Abbreviations}
The constant, i.e. $q$- and $\mu^2$ independent parts of the numerators
of the one-loop diagrams are factored out from the numerators and computed
as abbreviations. In some cases the list of abbreviations is too big to
be compiled into one subroutine. One can restrict the number of instructions
that go into a single subroutine by setting \texttt{abbrev.limit} to a positive
number in the setup file. 
The variable \texttt{abbrev.level}, which by default is set to 
\texttt{helicities}, can be set to \texttt{groups} or 
\texttt{diagrams} if the list of abbreviations common to a helicity configuration 
is too large.

If the list of abbreviations causes \haggies{}
to crash, one needs to increase the amount of memory reserved for Java.
This can be done by adding the \texttt{-Xmx} option to the call of Java.
A typical setting of the variable \texttt{haggies.bin} would be
\begin{example}
haggies.bin=java -Xmx3g -jar \\
\contl\ \$\{GOLEMPATH\}/haggies/haggies.jar
\end{example}
which assigns 3\,GB of memory to Java.

\subsection{Splitting the Process}
If a process becomes too big in order to be linked\footnote{
Currently, most systems support programs to a size up to 4\,GB.
Although 64\,bit systems can handle a much bigger address space,
the current limitation comes from some legacy code in the GNU linker.}
there are some possibilities to split the process into independent
programs:
\begin{itemize}
\item the generation of a subset of the helicity configurations, 
e.g. one helicity configuration
      per process directory.
\item the generation of a subset of diagrams. If the diagrams are not
      split according to gauge invariant subsets the user should ensure
      that all subsets are called with the same set of phase space points.
      An easy way of splitting the diagrams into subsets is by using
      the option \texttt{select.nlo=\textit{$\langle$first$\rangle$}:\textit{%
       $\langle$last$\rangle$}}, where \texttt{first} and \texttt{last} refer to the 
       diagram numbers in {\em process.ps}.
\end{itemize}

\section{Advanced Usage}
The call to the executable \texttt{gosam.py} can be simulated
inside more complex \python programs.
It is an easy exercise to
run the file generation in user defined \python scripts as long as
one includes the module files in the environment variable
\texttt{PYTHON\_PATH}. The following script emulates the
program \texttt{gosam.py}:
\begin{lstlisting}[language=python]
>>> from golem.util.config import Properties
>>> from golem.util.main_misc import *
>>> props = Properties()
>>> props.setProperty("in", ["e+", "e-"])
>>> props.setProperty("out", ["t", "t~"])
>>> # ... populate props with further values ...
>>> workflow(props)
>>> generate_process_files(props)
\end{lstlisting}

\section{Advanced Diagram Selection}
\golemv implements several ways of selecting subsets of diagrams:
\begin{itemize}
\item by restricting QGraf,
\item by selecting specific diagrams by their number,
\item by defining filters using \python.
\end{itemize}

\subsection{Restricting the Generation with QGraf}
The options for restricting the set of diagrams at the level
of the diagram generation is the most efficient way since this
happens already at the earliest possible stage.
However, QGraf's built-in filters are sometimes
too limited in order to express more advanced criteria.

\golemv{} allows one to pass information to QGraf through the option
\texttt{qgraf.options} and through \texttt{qgraf.verbatim},
\texttt{qgraf.verbatim.lo} and \texttt{qgraf.verbatim.nlo}.
For the exact syntax the user is refered to the QGraf documentation.

\subsection{Selecting Diagrams by their Number}
An a posteriori selection 'by eye' can be achieved after all (also unwanted)
diagrams of a process have been generated and inspected in
\texttt{doc/process.ps}. The user can then modify the options
\texttt{select.lo} and \texttt{select.nlo} and rerun \texttt{gosam.py}.

%This option is mainly intended for debugging purposes in large processes.

\subsection{Filtering Diagrams in \python{}}
The user can write short \python{} functions in order to decide whether
a specific diagram is to be taken or not. This function should return
\texttt{True} for all diagrams which are kept, and \texttt{False} for
all diagrams which should be discarded. These functions are passed by
the options \texttt{filter.lo} and \texttt{filter.nlo}.

Longer functions should be defined in an external file, which can be
passed using \texttt{filter.module}.

When writing a filter the one can use the predefined particle lists
\texttt{QUARKS}, \texttt{LEPTONS}, \texttt{FERMIONS} and \texttt{BOSONS}.
The underscore (\texttt{\_}) matches any field.

A diagram object \texttt{d} has the following methods which are inteded
to be used in filters.
Alternative predefined functions and functors are also given.
\begin{description}
\item[\texttt{d.rank()}] returns the tensor rank of a diagram.\\
   $\mathtt{RANK}\equiv\lambda\mathtt{d}.(\mathtt{d.rank()})$
\item[\texttt{d.loopsize()}] returns the number of propagators
   in the loop of a diagram.\\
   $\mathtt{LOOPSIZE}\equiv\lambda\mathtt{d}.(\mathtt{d.loopsize()})$
\item[\texttt{d.sign()}] computes the sign coming from closed
   fermion loops.\\
   $\mathtt{SIGN}\equiv\lambda\mathtt{d}.(\mathtt{d.sign()})$
\item[\texttt{d.isNf()}] reports if a diagram contains a closed
   quark loop of size two where all loop propagators are massless.\\
   $\mathtt{NF}\equiv\lambda\mathtt{d}.(\mathtt{d.isNf()})$
\item[\texttt{d.isMassiveQuarkSE()}] returns True if the diagram
   contains a QCD self energy insertion at a massive quark line.\\
   $\mathtt{MQSE}\equiv\lambda\mathtt{d}.(\mathtt{d.isMassiveQuarkSE()})$
\item[\texttt{d.isScaleless()}] returns True if the loop integral associate
   with this diagram carries no scale.\\
   $\mathtt{SCALELESS}\equiv\lambda\mathtt{d}.(\mathtt{d.isScaleless()})$
\item[\texttt{d.vertices(f1,f2,\ldots)}] returns the number of vertices
   in the diagram with the specified fields. The arguments \texttt{f1},
   \texttt{f2}, \dots are lists of field names.\\
   $\mathtt{VERTICES(f1,f2,\ldots)}\equiv%
    \lambda\mathtt{d}.(\mathtt{d.vertices(\mathtt{f1},\mathtt{f2},\ldots)})$
\item[\texttt{d.loopvertices(f1,f2,\ldots)}]
   same as \texttt{vertices}, but only counts vertices which have
   loop propagators attached.\\
   $\mathtt{LOOPVERTICES(f1,f2,\ldots)}\equiv%
    \lambda\mathtt{d}.(\mathtt{d.loopvertices(%
    \mathtt{f1},\mathtt{f2},\ldots)})$
\item[\texttt{d.iprop(f,**opts)}] returns the number of propagators
   of the given fields. Optional arguments are
   \texttt{momentum} to specify the momentum of the propagator,
   \texttt{twospin} to filter by the $2\times$ the spin,
   \texttt{massive} to specify whether massive or massless propagators
   should be considered and \texttt{color} to filter for certain color
   representations.\\
   $\mathtt{IPROP(\ldots)}\equiv%
    \lambda\mathtt{d}.(\mathtt{d.iprop(\ldots)})$
\item[\texttt{d.chord(f,**opts)}] same as \texttt{iprop}
   but only counts loop propagators.\\
   $\mathtt{CHORD(\ldots)}\equiv%
    \lambda\mathtt{d}.(\mathtt{d.chord(\ldots)})$
\item[\texttt{d.bridge(f,**opts)}] same as \texttt{iprop}
   but only counts propagators which are not in a loop.\\
   $\mathtt{BRIDGE(\ldots)}\equiv%
    \lambda\mathtt{d}.(\mathtt{d.bridge(\ldots)})$
\item[\texttt{d.QuarkBubbleMasses()}] returns a list of
   all different masses in a closed quark loop of size two
   or an empty list if the diagram is not a quark bubble.\\
   $\mathtt{QBMASSES}\equiv%
    \lambda\mathtt{d}.(\mathtt{d.QuarkBubbleMasses()})$
\end{description}

Furthermore, the following predefined filters exist:
\begin{description}
\item[\texttt{NFGEN(f1,f2,\ldots)}] for closed quark loops of size two
   this filter returns true only if all loop propagators belong to one
   of the fields in the argument list. For all diagrams which are not
   quark bubbles it returns True.
\item[\texttt{AND(filter1,filter2,\ldots)}] returns True if all filters
   in the argument list return True.
\item[\texttt{OR(filter1,filter2,\ldots)}] returns True if at least one filter
   in the argument list returns True.
\item[\texttt{NOT(filter)}] returns True if the argument evaluates to False.
\item[\texttt{TRUE}] always returns True.
\item[\texttt{FALSE}] always returns False.
\end{description}

\chapter{The Binoth Les Houches Accord Interface}

\section{Initialisation Phase}
The script \texttt{gosam.py --olp} which comes with \golemv{} can be used to generate
matrix elements compatible with the specifications of the Binoth Les Houches
Accord~\cite{Binoth:2010xt}. This script expects at least the name of an order file.
This order file is usually but not necessarily created by a Monte Carlo program. An
example file for the partonic $2\to3$ processes of $pp\rightarrow t\bar{t}+\text{jets}$
is given below:
\lstset{literate={->}{{$\rightarrow$}}1}
\begin{lstlisting}[language=olp,numbers=left]
MatrixElementSquareType CHsummed
IRregularisation        tHV
OperationMode           CouplingsStrippedOff
SubdivideSubprocess     yes
AlphasPower             3
CorrectionType          QCD

# Here comes the list of subprocesses
# specified through PDG codes
# g     g -> t t-bar g
 21    21 -> 6 -6   21
# u u-bar -> t t-bar g
  2    -2 -> 6 -6   21
# u     g -> t t-bar u
  2    21 -> 6 -6    2
\end{lstlisting}
The line numbers are not part of the file.
The arrow `$\rightarrow$' is generated by the two characters `\verb|->|'.
The following options are part of the Standard and accepted by
\golemv{}:
\lstset{language=olp}
\begin{description}
\item[\texttt{MatrixElementSquareType}] accepts the values
	\lstinline!Hsummed!, \lstinline!Csummed!,
	\lstinline!Haveraged!, \lstinline!Caveraged!,
	\lstinline!CHsummed!, \lstinline!CHaveraged!.

	The value \lstinline!NOTsummed! is not supported. 
	Sensible combinations are also allowed, as in
\begin{lstlisting}[language=olp]
MatrixElementSquareType Hsummed Caveraged
\end{lstlisting}

        In \golemv{} this statement is optional.
	Any quantity which is not explicitely averaged is assumed to be
	summed
\item[\texttt{CorrectionType}] accepts the values
        \lstinline!QCD!, \lstinline!QED! and \lstinline!EW!, whereas
	\golemv{} does not distinguish between the latter two (this behaviour
	might change in the future when appropriate model files are available).

	This statement is mandatory and must not be omitted.
\item[\texttt{IRregularisation}] accepts the values \lstinline!tHV!
        ('t~Hooft-Veltman scheme) and \lstinline!DRED! (dimensional reduction).
	The value \lstinline!CDR! (conventional dimensional regularisation)
	is not supported and therefore rejected.

%	\attention The support for \lstinline!DRED! is limited.
%        This featureis still considered experimental. 
%       It should only be used with great care.

	This statement is mandatory and must not be omitted.
\item[\texttt{MassiveParticleScheme}] accepts the value \lstinline!OnShell!
	only.
	At the moment this option has no effect on the generation of the matrix
	element. This statement is optional; if it appears in the order file
	a warning is issued, reminding the user that no UV-counterterms for
	massive particles are implemented yet.
\item[\texttt{IRsubtractionMethod}] accepts the value \lstinline!None! only.
	\golemv{} does not provide any subtracted output.

	This statement is optional.
\item[\texttt{ModelFile}] accepts the name of parameter file in the
	Les Houches Accord format. The script reads the parameter file
	setting all masses to zero which are not specified explicitly
	to be non-zero.

	This statement is mandatory.

	It is recommended to use absolute paths here as the file will later
	be read in the function \texttt{OLP\_Start} in the matrix element
	code, which might be located elsewhere.
\item[\texttt{OperationMode}] accepts the value
	\lstinline!CouplingsStrippedOff! only.

	This statement is optional. If it is given, the coupling constants
	are stripped off from the amplitude.
\item[\texttt{SubDivideSubProcess}] accepts logical values
	(\lstinline!yes! or \lstinline!no!).

	If the value is \lstinline!yes! a separate channel for each
	helicity is assigned. Otherwise there will be one channel per
	subprocess.

	This statement is optional. Its default value is \lstinline!no!.
\item[\texttt{AlphasPower}] the power of $\alpha_s$ of the Born cross-section.
	At least one of the options \lstinline!AlphaPower! and
	\lstinline!AlphasPower! has to be specified.
\item[\texttt{AlphaPower}] the power of $\alpha$ of the Born cross-section.
	At least one of the options \lstinline!AlphaPower! and
	\lstinline!AlphasPower! has to be specified.
\end{description}

The options which have been proposed for electro-weak corrections
are currently not supported.

\subsection{Command Line Arguments of \texttt{gosam.py --olp}}
The syntax for the invocation of \texttt{gosam.py} is as follows:
\begin{example}
\$ gosam.py --olp \{$\langle$\textit{option}$\rangle$\}\\
    \contl $\langle$\textit{order file}$\rangle$
    \{$\langle$\textit{order file}$\rangle$\} \\
    \contl \{$\langle$\textit{key}$\rangle$=$\langle$\textit{value}$\rangle$\}
\end{example}
The allowed options are given below. The list of
$\langle$\textit{key}$\rangle$\texttt{=}$\langle$\textit{value}$\rangle$-pairs
supplements the options given in the configuration files.
\begin{description}
\item[\texttt{-h}, \texttt{--help}]
      Prints a help screen with all available command line options and exits.
\item[\texttt{-d}, \texttt{--debug}] 
      With this options the script will print lots of extra information to
      the screen, which is usually not useful for non-experts.
\item[\texttt{-v}, \texttt{--verbose}] 
      The script will print information e.g. about creating directories
      and reading files.
\item[\texttt{-w}, \texttt{--warn}]
      Warnings and errors are printed. This is the default setting.
\item[\texttt{-q}, \texttt{--quiet}]
      Only errors are printed, no warnings are issued.
\item[\texttt{-l}\textit{file}, \texttt{--log-file=}\textit{file}]
      All messages are written to a log file. When one or more log files
      are specified the information is still written to the screen with
      the latest specified level of detail. The following example will
      read the order file \texttt{test.olo}; messages at the debug
      level will be written to \texttt{detailed.log}, warnings and
      errors are written to \texttt{short.log} and only errors are printed
      to the screen.
      \begin{example}
\$ gosam.py --olp -d -ldetailed.log -w \\
	\contl{} -lshort.log -q test.olo
      \end{example}
\item[\texttt{-c}\textit{file}, \texttt{--config=}\textit{file}]
      Overlay default config files by the specified file.
      Usually, the script first searches in the default locations for
      configuration files. Afterwards, all files specified by \texttt{-c}
      options are read in the order in which they are encountered. Values
      which are already set by earlier files will be overwritten.
      See also option `\texttt{-C}'.
\item[\texttt{-C}, \texttt{--no-defaults}]
      The script will not search for configuration
      files (\texttt{.golem} and \texttt{golem.in}) in the standard locations
      (\golemv{} installation directory, user's home directory and
      current working directory).
\item[\texttt{-f}, \texttt{--force}] Overwrite contract files without asking.
      The default behaviour is that contract files are not overwritten. If
      a contract file already exists the program gives an error message.
\item[\texttt{-e}, \texttt{--use-single-quotes}]
      Activates syntax extensions that allow the use of single quotes in
      order and contract files (See Section~\ref{sec:olp:extensions}).
\item[\texttt{-E}, \texttt{--use-double-quotes}]
      Activates syntax extensions that allow the use of double quotes in
      order and contract files (See Section~\ref{sec:olp:extensions}).
\item[\texttt{-b}, \texttt{--use-backslash}]
      Activates syntax extensions that allow the use of backslash escape
      sequences in
      order and contract files (See Section~\ref{sec:olp:extensions}).
\item[\texttt{-i}, \texttt{--ignore-case}] 
      Activates syntax estensions which make the parsing of order
      and contract files case-insensitive
      (See Section~\ref{sec:olp:extensions}).
\item[\texttt{-x}, \texttt{--ignore-unknown}]
      Unknown statements or values in order and contract files will be ignored.
      The default behaviour is that unknown statements and/or values will lead
      to an error message.
\item[\texttt{-o}\textit{file}, \texttt{--output-file=}\textit{file}]
      Specifies the name of the contract file(s). The following set of
      wildcard sequences can be used to derive the name of the contract file
      from the name of the order file. A value of
      `\texttt{-}' writes to the standard output.
      \begin{description}
      \item[\texttt{\%f}] The full file name (e.g. `\texttt{dir/process.olo}')
      \item[\texttt{\%F}] The file name without any leading path
         (`\texttt{process.olo}')
      \item[\texttt{\%p}] Path name only (`\texttt{dir/}')
      \item[\texttt{\%s}] The stem of the file name (`\texttt{process}')
      \item[\texttt{\%e}] The extension of the file name (`\texttt{.olo}')
      \end{description}
      By default this option is set to `\texttt{\%p\%s.olc}'.
\item[\texttt{-D}\textit{dir}, \texttt{--destination=}\textit{dir}]
      Chooses the output directory, to which each process is written.
      The same wildcards as above can be used. By default, all output is
      written to the current working directory. It is therefore not recommended
      to set this option using wildcards when more than one order file is
      specified.
\item[\texttt{-t}\textit{path}, \texttt{--templates=}\textit{path}]
      Sets an alternative templates directory or template
      XML-file.
\item[\texttt{-z}, \texttt{--scratch}]
      Overwrites all process files, including those which otherwise
      would be preserved (\texttt{Makefile.conf}, \texttt{config.f90} etc).
\end{description}

\subsection{\golem{} Extensions to the Original Standard}
\label{sec:olp:extensions}
Modern file systems allow for path names which cannot be expressed in
the original formulation of the Les Houches accord. Therefore \golemv{}
implements syntax extensions for order and contract files for including
special characters in statements, especially in file names (as in
\lstinline!ModelFile!).

\begin{description}
\item[double qoutes] This syntax extension proposes that inside a pair
   of double quotes (ASCII character~\#34) special characters lose their
   special meaning. The backslash acts as escape character, with the following
   set of escape sequences being allowed:
   \begin{itemize}
   \item[\texttt{\textbackslash t}] expands to a horizontal tabulator
        character (ASCII character~\#9),
   \item[\texttt{\textbackslash n}] expands to a new line character
   	(ASCII character~\#10),
   \item[\texttt{\textbackslash f}] expands to a form feed character
   	(ASCII character~\#12),
   \item[\texttt{\textbackslash r}] expands to a carriage return character
   	(ASCII character~\#13),
   \item[\texttt{\textbackslash x}\textit{hh}], where \textit{hh} are two
        hexadecimal digits expands to the character of which the
        ASCII code is the hexadecimal number represented by the
	digits~\textit{hh}.
   \item any other character following a backslash expands to itself,
        in particular \texttt{\textbackslash"}
	and~\texttt{\textbackslash\textbackslash}.
   \end{itemize}
\item[single quotes] This syntax extension proposes that inside a pair
   of single quotes (ASCII character~\#39) all characters lose their
   special meaning. There is no escape character. A literal single quote
   is generated by a sequence of two single quotes (Pascal like).
\item[backslash escapes] This syntax extension proposes that any character
   following a backslash loses its special meaning.
\end{description}

Different extensions might prove useful on different operating systems.
On a Windows system, the file name \verb!F:\Golem Files\mssm.slha! can
only be expressed with the proposed syntax extensions and would have the
following three equivalent representations:
\begin{itemize}
\item \verb!F:\\Golem\ Files\\mssm.slha!
\item \verb!'F:\Golem Files\mssm.slha'!
\item \verb!"F:\\Golem Files\\mssm.slha"!
\end{itemize}

The three extensions can be switched on by the command line options
of \texttt{gosam.py --olp},
`\texttt{-E}', `\texttt{-e}' and `\texttt{-b}' respectively.

\subsection{Advanced Usage}
The core functionality of the script \texttt{gosam.py --olp} is implemented
by the function \texttt{golem.util.olp.process\_order\_file}, which has the
the following signature:
\begin{example}
process\_order\_file(order\_file\_name, out\_file, out\_dir,\\
   \ \ conf,
   templates=None,
   ignore\_case=False,\\
   \ \ ignore\_unknown=False,
   single\_quotes=False,\\
   \ \ double\_quotes=False,
   backslash\_escape=False)
\end{example}
\begin{description}
\item[\texttt{order\_file\_name}] (character string) name of the order
   file.
\item[\texttt{out\_file}] (file object, open for writing)
   contract file.
\item[\texttt{out\_dir}] (character string) name of an existing directory
   to which all matrix-element files will be written.
\item[\texttt{conf}] (\texttt{golem.util.config.Properties})
   configuration shared by all subprocesses.
\item[\texttt{templates}] (character string) template directory or
   name of an XML-file.
\item[\dots] all other arguments activate the corresponding
   syntax extensions.
\end{description}
The return value is zero in case of a success and one if an error occurred.

A list of options read from default config files can be obtained by
the function
\texttt{golem.util.main\_misc.find\_config\_files()}. The following
example suggests the usage of the interface from a \python{}-based
Monte Carlo program
\begin{lstlisting}[language=python]
import os
import golem

# Monte Carlo program prepars the process
# and writes order file proc.olo ...
# (not shown in example)

conf = golem.util.main_misc.find_config_files()
f = open("proc.olc", 'w')
os.mkdir("proc/")

# Add own options
conf[golem.properties.model] = \
   "FeynRules, ${HOME}/models/mssm_ufo"
conf[golem.properties.fc_bin] = "gfortran"
err_flag = golem.util.olp.process_order_file(\
   "proc.olo", f, "proc/", conf)

if err_flag > 0:
	print "Problems generating OLP"
	print "Please consult the file proc.olc"
\end{lstlisting}


\section{Runtime Phase}

After the script \texttt{gosam.py --olp} or any equivalent program has been
run successfully, the files in the newly created process directories are
compiled by invoking \texttt{make} in the respective top-level directory.
This generates the object file \texttt{olp\_module.o} which contains all
API functions. The library for a given process can be linked using the
script \texttt{config.sh} in the same directory. The make-file of a client
program would typically contain code similar to the following:
\begin{lstlisting}[language=bash]
PROCESS_PATH=path/to/your/process-files
LDFLAGS+=$(shell sh $(PROCESS_PATH)/config.sh -libs)
\end{lstlisting}

\attention The module \texttt{olp\_module.f90} uses \fortranMMIII{}
extensions (\texttt{ISO\_C\_BINDING}) for establishing a well defined
interface for the linker. Older \fortranXC{} compilers might therefore
not be able to compile this module. Please refer to the compiler documentation
for details.

\subsection{API Functions}
\lstset{language=C}
The file \texttt{olp.h} contains the following prototypes.
\begin{lstlisting}[language=C]
void OLP_Start(char*,int*);
void OLP_EvalSubProcess(int,double*,
   double,double*,double*);
void OLP_Finalize();
void OLP_Option(char*,int*);
\end{lstlisting}
The first two functions are defined exactly as proposed
in~\cite{Binoth:2010xt}. The other two functions extend the original
standard. It should, however, be noted that the generated matrix element code
can be run without any calls to either \lstinline!OLP_Finalize!
or \lstinline!OLP_Option!.

\subsubsection*{\texttt{OLP\_Start}}
\begin{lstlisting}[language=C]
void OLP_Start(char* contract_file, int* success);
\end{lstlisting}
This function should be called before the first evaluation of the matrix
element. It ensures that all global variables in the matrix element code
are initialized properly. The argument \lstinline!contract_file!
should receive the (full) name of the contract file which was generated
together with the matrix element. The integer \lstinline!success! is
initialized by \lstinline!OLP_Start! to either the value one, indicating
success, or zero, indicating that an error occurred during initialization.

Matrix elements generated with \golemv{} will try to read the SLHA model
file specified by the option \lstset{language=olp}\lstinline!ModelFile!
in the contract file. It is not required that the contract file used in
the runtime phase points to the same model file as used during the
initialisation phase. However, values which were set to zero during
initialisation will remain zero during the runtime phase.

\subsubsection*{\texttt{OLP\_EvalSubProcess}}
\lstset{language=C}
\begin{lstlisting}[language=C]
void OLP_EvalSubProcess(int label, double* momenta,
   double scale, double* parameter, double* amp);
\end{lstlisting}
This function retrieves the values for a channel of the OLP for a given
phase space point. A channel might be a subprocess or a
gauge invariant partial amplitude, depending on the settings in the
contract file. The channel is labeled by the argument \lstinline!label!.
The second argument is a one-dimensional array holding the $5\times N$
components of the momenta for an $N$-particle process. They are in the order
\begin{displaymath}
(E^{(1)}, p_x^{(1)}, p_y^{(1)}, p_z^{(1)}, m^{(1)},
 E^{(2)}, p_x^{(2)}, p_y^{(2)}, p_z^{(2)}, m^{(2)},
 \ldots, m^{(N)})
\end{displaymath}
The third argument is the renormalization scale (not its square).
A list of scale dependent parameters is passed in the fourth argument.
Its first entry is expected to be $\alpha_s(\mu)$. Any further entries
are user-defined; the user is expected to adapt the subroutine
\texttt{init\_event\_parameters} in \texttt{olp\_module.f90}
if he wishes to make use of any additional parameters.

The last argument is an array of length four. Its entries are, in this order,
\begin{enumerate}
\item the coefficient of the $1/\varepsilon^2$ pole in the Laurent series
   of the interference term between virtual and Born amplitude,
\item the coefficient of the $1/\varepsilon$ pole in the Laurent series
   of the interference term between virtual and Born amplitude,
\item the $\mathcal{O}(1)$ term in the Laurent series
   of the interference term between virtual and Born amplitude,
\item the square of the Born amplitude.
\end{enumerate}

Matrix elements generated by \golemv{} use the convention that in case
of an error during the evaluation of the matrix element, the fourth
entry is set to $(-1)$. It is therefore recommended that client programs
check for the positiveness of the Born matrix element.

\subsubsection{\texttt{OLP\_Finalize}}
\begin{lstlisting}[language=C]
void OLP_Finalize();
\end{lstlisting}

This function should be called after the last evaluation of the matrix
element. It allows the OLP to close any open file handles, to release
allocated memory and to exit gracefully.
Although on most modern operating systems this is done automatically,
it is good practice and therefore recommended to always call this function
before exiting the program.

\subsubsection{\texttt{OLP\_Option}}
\begin{lstlisting}[language=C]
void OLP_Option(char* assignment, int* success);
\end{lstlisting}

This function can be used to update internal parameters of the OLP which
are not part of the standard. The first argument is a character string
containing a textual representation of the requested assignment. The
second argument will be set by the function according to the success
of the request.

Matrix elements generated with \golemv{} accept any string which would
also be valid as a (non-comment) line in a parameter
file (see~\texttt{model.f90}). Typical calls would be
\begin{lstlisting}[language=C]
OLP_Option("samurai_test=3", &flag);
/* The previous call requires 
 * reinitialization of the OLP */
OLP_Start(contract_file, &flag);
OLP_Option("Nf=5", &flag);
/* Setting the Higgs mass: */
OLP_Option("mH=124.5", &flag);
\end{lstlisting}

\subsection{The OLP Socket Protocol}
The necessity to link the client program each time another OLP is used
might become cumbersome, especially when one likes to work with more than
one OLP at the same time. We have therefore developed a socket protocol
which enables any client program to access the same functionality as
defined in the Les Houches accord~\cite{Binoth:2010xt} through a TCP/IP
connection with a server hosting the OLP. In this way it is possible
to access multiple OLPs simultaneously and to load OLPs at runtime.

\subsubsection{The OLP Socket Server}
OLPs generated with \golemv{} contain additional files in their
top-level directory implementing a server for the OLP Socket protocol.
These files are
\begin{description}
\item[\texttt{olp\_daemon.c}] ANSI-C file with service routines especially
    network related routines,
\item[\texttt{olp\_daemon.h}] ANSI-C file, header for \texttt{olp\_daemon.c},
\item[\texttt{olp\_protocol.l}] Lex/Flex file, part of the grammar definition
    of the protocol and
\item[\texttt{olp\_protocol.y}] Yacc/Bison file, part of the grammar
    definition of the protocol, contains the main program.
\end{description}
These files are compiled with the command
\begin{example}
\$ make olp\_daemon EXTRA\_LDFLAGS=\dots
\end{example}
It is often necessary to specify the variable \verb!EXTRA_LDFLAGS!
to provide the necessary run-time libraries of the \fortranXC{} compiler.

The compiled program can be run with the following options
\begin{example}
\$ olp\_daemon [-p \textit{port}] [-s|-S] [-f] \textit{file-name}
\end{example}
\begin{description}
\item[\texttt{-f}\textit{file-name}] name of a contract file (required).
\item[\texttt{-p}\textit{port}] port at which the program accepts connections,
   default: 7711.
\item[\texttt{-s}/\texttt{-S}] forbid resp. allow the
   \texttt{SHUTDOWN} command, default: allow.
\item[\texttt{-r}/\texttt{-R}] forbid resp. allow the
   \texttt{RESTART} command, default: allow.
\item[\texttt{-d}] detach from terminal (run as daemon).
\end{description}

\subsubsection{OLP Socket Clients}
Sample client implementations for \texttt{C++}, \texttt{Java} and
\python{} are provided in the directory \texttt{olp/contrib/}.
Below, a brief example for the \texttt{C++} case is given:
\begin{lstlisting}[language=C]
olp::OLPClient OLP_EvalSubProcess("localhost", 7711);
OLP_EvalSubProcess(0, num_legs, mom, scale,
                   num_param, param, amp);
OLP_EvalSubprocess.close();
\end{lstlisting}
The class \texttt{OLPClient} overwrites the operater \texttt{()}
emulating the original protocol as closely as possible. For technical
reasons, two additional arguments (\texttt{num\_legs} and \texttt{num\_param})
are required, specifying the number of external legs and the length of the
array \texttt{param} respectively.

\subsubsection{Definition of the Protocol}
The protocol consists of statements sent by the client to the server.
Each statement is terminated by a newline character. The server responds
with one line starting with a three digit number followed by a space
and an optional message. The three digit number contains the response code.
A response code of 200 signals success, all other values denote an error.

% \TODO{complete description}

%%%%%%%%%%%%%%%%%%%%%%%%%%%%%%%%%%%%%%%%%%%%%%%%%%%%%%%%%%%%%%%%%%
% APPENDIX APPENDIX APPENDIX APPENDIX APPENDIX APPENDIX APPENDIX %
%%%%%%%%%%%%%%%%%%%%%%%%%%%%%%%%%%%%%%%%%%%%%%%%%%%%%%%%%%%%%%%%%%
\appendix

\chapter{Conventions of the Amplitude}

\section{Convention of \golemVC}
The integral library \golemVC{} computes integrals of the form
\begin{multline}
\int\frac{\mu^{2\varepsilon}\mathrm{d^nk}}{i\pi^{n/2}}\frac{k^{\mu_1}\cdots k^{\mu_r}}{%
((k+r_1)^2-m_1^2)\cdots(k+r_N)^2-m_N^2)}=\\
r_\Gamma\cdot\left[\frac{c_{-2}}{\varepsilon^2}+\frac{c_{-1}}{\varepsilon}+c_0
+{\mathcal{O}}(\varepsilon)\right]
\end{multline}
where $n=(4-2\varepsilon)$ and
\begin{equation}
r_\Gamma=\frac{\Gamma(1+\varepsilon)\Gamma^2(1-\varepsilon)}{%
   \Gamma(1-2\varepsilon)}.
\end{equation}
The integration measure for the internal momentum $k$ is
\begin{equation}
\frac{\mu^{2\varepsilon}\mathrm{d}^nk}{(2\pi)^n}
=\mu^{2\varepsilon}\frac{i}{2^n\pi^{n/2}}\cdot\frac{{\mathrm d}^nk}{i\pi^{n/2}}
=\frac{(4\pi)^\varepsilon \cdot i}{(4\pi)^2}\cdot%
 \frac{\mu^{2\varepsilon}{\mathrm d}^nk}{i\pi^{n/2}}.
\end{equation}

\section{Convention of \golemv}
The factor from above which does not go into the integral definition of
\golemVC{} can be written as
\begin{equation}
\frac{(4\pi)^\varepsilon \cdot i}{(4\pi)^2}=
\frac{(4\pi)^\varepsilon}{(2\pi)(4\pi)}\frac{i}{2}
\end{equation}
The factor of $i/2$ is included in the amplitude definition of \golemv{}.
The factors $(2\pi)$ and $(4\pi)$ are later used to build up a factor of
$\alpha_x/2\pi$, where $\alpha_x$ is either $\alpha$ or $\alpha_s$.

In the following we assume that the coupling constants\footnote{
$e$ and $g_s$ in the standard model} have been set to one in the
setup of \golemv{}. This ensures that the one-loop matrix
element in QCD is calculated in the $\overline{\mathrm{MS}}$ scheme as
\begin{equation}
\left\vert\mathcal{M}\right\vert^2_{\text{1-loop}}=
\frac{\alpha_s}{2\pi}\frac{(4\pi)^\varepsilon}{\Gamma(1-\varepsilon)}
\cdot\left[\frac{c_{-2}}{\varepsilon^2}+\frac{c_{-1}}{\varepsilon}+c_0
+{\mathcal{O}}(\varepsilon)\right](g_1^{n_1}\cdots g_q^{n_q})
\end{equation}
The factor $(g_1^{n_1}\cdots g_q^{n_q})$ are the coupling constants
appearing in the squared tree-level matrix element. \golemv{} will
return the coefficients $c_{-2}$, $c_{-1}$ and $c_0$.

The conversion between different conventions for the $\Gamma$-functions
is straightforward:
\begin{equation}
\frac{1}{\Gamma(1-\varepsilon)}=r_\Gamma+{\mathcal O}(\varepsilon^3)=
\left(1-\frac{\pi^2}{6}\varepsilon^2\right)\Gamma(1+\varepsilon)
   +{\mathcal O}(\varepsilon^3)
\end{equation}

The relevant terms in the expansion of $r_\Gamma$ are
\begin{equation}
r_\Gamma=e^{-\gamma_E\varepsilon}
\left(1-\frac{\pi^2}{12}\varepsilon^2\right)+\mathcal{O}(\varepsilon^3)
\end{equation}

If one prefers to pull out a factor of
$e^{-\gamma_E\varepsilon}(4\pi)^{\varepsilon}$ the appropriate
definition of the matrix element up to terms of $\mathcal{O}(\epsilon)$ is
\begin{equation}
\frac{\left\vert\mathcal{M}\right\vert^2_{\text{1-loop}}}%
{e^{-\gamma_E\varepsilon}(4\pi)^\epsilon}=
\frac{\alpha_s}{2\pi}
\cdot\left[\frac{c_{-2}}{\varepsilon^2}+\frac{c_{-1}}{\varepsilon}
+\left(c_0-\frac{\pi^2}{12}\,c_{-2}\right)
\right](g_1^{n_1}\cdots g_q^{n_q})
%+{\mathcal{O}}(\varepsilon)
\end{equation}

\chapter{Explicit Reduction of the $R_2$ Terms}
The $R_2$ term \cite{Ossola:2008xq} consists of all terms of the numerator
containing an explicit $\varepsilon$ or $\mu^2$ coming from the Lorentz
algebra. For an explicit reduction of these terms, a list of all integrals
of the form
\begin{align}
\int\frac{\mu^{2\varepsilon}\mathrm{d}^nk}{i\pi^{n/2}}
\frac{N(\hat{q})\cdot\mu^{2\alpha}\cdot\varepsilon^\beta}{D_0\cdots D_N}
\end{align}
where either $\alpha$ or $\beta$ is a positive integer number
and the denominators are $D_i=(q+r_i)^2-m_i^2+i\delta$.
Note that integrals where both $\alpha$ and $\beta$ are
non-zero, will not contribute to the final result.
We expand the above tensor integral and only consider the term
of rank $r$, similarly to Eq.~(208) in Ref.~\cite{Reiter:2009kb}:
\begin{multline}
I_N^{n,\alpha,\beta;\mu_1\ldots\mu_r}=
(-1)^{r}\frac{\Gamma(\alpha-\varepsilon)}{\Gamma(-\varepsilon)}
\varepsilon^\beta
\sum_{l=0}^{\lfloor r/2\rfloor}\left(-\frac12\right)^l
\sum_{j_1,\ldots,j_{r-2l}=1}^N
\times\\
\left[\hat{g}^{\bullet\bullet}\ldots
\hat{g}^{\bullet\bullet}r_{j_1}^\bullet
\cdots r_{j_{r-2l}}^\bullet\right]^{\mu_1\ldots\mu_r}
I_N^{n+2\alpha+2l}(j_1,\ldots,j_{r-2l}).
\end{multline}
Here, the integral $I_N^d(j_1,j_2,\ldots)$ denotes a Feynman parameter
integral with the parameters $z_{j_1}, z_{j_2}, \ldots$ in the numerator,
\begin{multline}
I_N^d(j_1,\ldots, j_p)=\\
(-1)^N\Gamma\left(N-\frac{d}2\right)%
\int\!\!\mathrm{d}^N_\Box\!z\,\delta_z
\frac{\prod_{\nu=1}^p z_{j_\nu}}{%
\left[-\frac12 z^{\mathsf{T}}Sz-i\delta\right]^{N-d/2}},
\end{multline}
where $\mathrm{d}^N_\Box\!z=
\prod_{j=1}^N\mathrm{d}z_j\Theta(z_j)\Theta(1-z_j)$
and $\delta_z=\delta(1-\sum_i z_i)$.
The square brackets $[\ldots]^{\mu_1\ldots\mu_p}$ expand to the sum of
all possible assignments of indices to the $\hat{g}^{\bullet\bullet}$-tensors
where a (one) arbitrary assignment of indices to the momenta $r_j^\bullet$
is chosen.

We only need to consider integrals containing an UV pole, which leads to
a rational term when multiplied with $\varepsilon$ stemming either from
$\varepsilon^\beta$ or from
\begin{equation}
\frac{\Gamma(\alpha-\varepsilon)}{\Gamma(-\varepsilon)}=
(\alpha-1)!\left[-\varepsilon +{\mathcal O}(\varepsilon^2)\right],
\quad\text{for}\,\alpha>0.
\end{equation}
The UV divergence comes from the Gamma function
\begin{equation}
\Gamma\left(N-\frac{n+2\alpha+2l}2\right)=
\Gamma(\varepsilon-(2+\alpha+l-N))\equiv\Gamma(\varepsilon-\eta)
\end{equation}
in the Feynman parameter integral~$I_N^{n+2\alpha+2l}$.
Hence, we examine further the expression
\begin{multline}
\varepsilon\cdot I_N^{n+2l+2\alpha}(l_1,\ldots, l_{r-2l})=\\
\left\{\begin{array}{lr}
{\mathcal O}(\varepsilon),&\eta<0\\
(-1)^N\frac1{2^\eta\eta!}\int\mathrm{d}^N_\Box\!z\delta_z
\left[z^{\mathsf{T}}Sz\right]^\eta
\prod_{i=1}^{r-2l}z_{l_i},&\eta\geq0
\end{array}\right.
\end{multline}

The remaining integration can be understood as a special case of the
Feynman parameter identity
\begin{equation}
\frac{1}{\prod_{j=1}^N A_j^{\alpha_j}}=\frac{\Gamma(\alpha)}{
\prod_{j=1}^N \Gamma(\alpha_j)}\int\!\mathrm{d}^N_\Box\!z\,\delta_z
\frac{\prod_{j=1}^N z_j^{\alpha_j-1}}{\left(
\sum_{j=1}^N z_j A_j\right)^\alpha}
\end{equation}
for $A_j=1$, in which case one finds
\begin{equation}
\int\!\mathrm{d}^N_\Box\!z\,\delta_z
\prod_{j=1}^N z_j^{\alpha_j-1}=\frac{\prod_{j=1}^N \Gamma(\alpha_j)}%
{\Gamma(\alpha)}
\end{equation}

All phenomenologically relevant, non-zero cases for renormalizable
gauge theories (working in Feynman gauge) are listed below:
\begin{align}
I_1^{n,0,1}&=-\frac12 S_{11}\\
I_1^{n,0,1;\mu_1}&=\frac12 S_{11} \cdot r_1^{\mu_1}\\
I_2^{n,1,0}&=-\frac16\left(S_{11}+S_{12}+S_{22}\right)\\
I_2^{n,0,1}&=1\\
I_2^{n,0,1;\mu_1}&=-\frac12\left(r_1^{\mu_1}+r_2^{\mu_1}\right)\\
I_2^{n,0,1;\mu_1\mu_2}&=
\frac16\left(%
2r_1^{\mu_1}r_1^{\mu_2}
+r_1^{\mu_1}r_2^{\mu_2}
+r_2^{\mu_1}r_1^{\mu_2}
+2r_2^{\mu_1}r_2^{\mu_2}\right)\nonumber\\
&-\frac1{12}\hat{g}^{\mu_1\mu_2}\left(S_{11}+S_{12}+S_{22}\right)\\
I_3^{n,1,0}&=\frac12\\
I_3^{n,1,0;\mu_1}&=-\frac16\left(r_1^{\mu_1}+r_2^{\mu_1}+r_3^{\mu_1}\right)\\
I_3^{n,0,1;\mu_1\mu_2}&=\frac14\hat{g}^{\mu_1\mu_2}\\
I_3^{n,0,1;\mu_1\mu_2\mu_3}&=-\frac{1}{12}
\sum_{l=1}^3\left[\hat{g}^{\bullet\bullet}r^{\bullet}\right]^{\mu_1\mu_2\mu_3}%
\\
I_4^{n,1,0;\mu_1\mu_2}&=\frac{1}{12}\hat{g}^{\mu_1\mu_2}\\
I_4^{n,2,0}&=-\frac16\\
I_4^{n,0,1;\mu_1\mu_2\mu_3\mu_4}&=\frac1{4!}\left[\hat{g}^{\bullet\bullet}%
\hat{g}^{\bullet\bullet}\right]^{\mu_1\mu_2\mu_3\mu_4}
\end{align}
All other integrals of that type are identically zero.



\chapter{The included Model Files}
\label{chp:model-files}

\section{Format of the Model Files}\label{sec:modelfiles}
\golemv{} expects three files for a proper model definition:
\begin{description}
\item[$\langle model\rangle$\texttt{.hh}] \form{} file containing the Feynman rules
\item[$\langle model\rangle$\texttt{.py}] \python{} file
\item[$\langle model\rangle$] (no extension) \qgraf{} model file
\end{description}

\subsection{The \python{} File}
Thy \python{} file contains the following definitions
\begin{description}
\item[\texttt{model\_name}] a variable of string type containing a human-readable
     name for this model, such as ``Standard Model (Feyn. Gauge) w/o Higgs'' etc.
\item[\texttt{particles}] a \python{} \texttt{dict} that contains all particles
     \emph{and} anti-particles of the model. The keys are the \qgraf{} names of the
     fields; the values are objects of the class \texttt{Particle}.
     The constructor has the arguments
     \begin{verbatim}
Particle(name, two_spin, mass, color_rep, partner, width='0')
     \end{verbatim}
\item[\texttt{mnemonics}] a \python{} \texttt{dict} of
     human-readable particle names. The values are objects of the class
     \texttt{Particle}. It is save to refer to the dictionary \texttt{particles}.
\item[\texttt{parameters}] a \python{} \texttt{dict} of
     model parameters with their default values. Both key and value are strings.
\item[\texttt{functions}] a \python{} \texttt{dict} of
     variable names and initialization expressions. Both key and value are strings.
\item[\texttt{types}] the types of all parameters and functions indicated by
     \texttt{'R'} for real numbers and \texttt{'C'} for complex numbers.
\item[\texttt{latex\_names}] a \python{} \texttt{dict} assigning \LaTeX{}
     code to the field names. Math mode is assumed.
\item[\texttt{line\_styles}] a \python{} \texttt{dict} assigning line styles
     to field names. The line style used when drawing Feynman diagrams.
     Allowed values are \texttt{photon}, \texttt{ghost}, \texttt{scalar},
     \texttt{gluon}, \texttt{fermion}.
\end{description}

\subsection{The \qgraf{} File}
The propagators in the \qgraf{} file must contain the following functions:
\begin{description}
\item[\texttt{TWOSPIN}] twice the spin of the particle.
\item[\texttt{COLOR}]   the color representation of the particle $\in\{1,3,8\}$.
\item[\texttt{MASS}]    the mass of the particle.
\item[\texttt{WIDTH}]   the width of the particle (currently not used).
\item[\texttt{AUX}]     must be zero for most fields. Tensor Ghosts, as introduced
                        by CalcHep have the value $1$ here.
\item[\texttt{CONJ}]    for self-conjugate particles the value is \texttt{('+')},
                        otherwise it is \texttt{('+','-')}.
\end{description}

The vertices must provide all fields that should be accessible in \texttt{VSUM} statements
and therefore also the ones that \golemv{} uses in the \texttt{order} option.

\subsection{The \form{} File}
There are two possible ways of specifying the Feynman rules in the \form{} file.
If a model contains only Standard Model like interactions one can make use of
the file \texttt{src/form/vertices.hh} in the \golemv{} directory and just define
the coefficients \texttt{CL} and \texttt{CR} in front of the vertices. This
strategy is implemented by the modelfiles \texttt{models/sm}. The file
\form{} contains a procedure \texttt{VertexConstants} which
replaces the the vertex constants by their symbols. A QED example would be
\begin{lstlisting}[language=form]
#Procedure VertexConstants
   Id CL([field.em], [field.ep], [field.ph]) = e;
   Id CR([field.em], [field.ep], [field.ph]) = e;
#EndProcedure
\end{lstlisting}
In the header of the \form{} file all model specific
symbols and functions need to be defined. For this simple
model we have the fields and the coupling constant as only
new symbols.
\begin{lstlisting}[language=form]
Symbols [field.em], [field.ep], [field.ph], e;
\end{lstlisting}

Instead of using the file \texttt{vertices.hh} one can also use
his own vertex definitions. In this case the \form{} file must contain
the definition
\begin{lstlisting}[language=form]
#Define USEVERTEXPROC "1"
\end{lstlisting}
and it must define the procedure \texttt{ReplaceVertices}. An example
for QED is given below.
\begin{maxipage}
\begin{lstlisting}[language=form]
#Procedure ReplaceVertices
Identify Once vertex(iv?,
      [field.ep], idx1?, -1, k1?, idx1L1?, -1, idx1C1?,
      [field.em], idx2?,  1, k2?, idx2L1?,  1, idx2C1?,
      [field.ph], idx3?,  2, k3?, idx3L2?,  1, idx3C1?) =
   PREFACTOR(i_ * e) *
   NCContainer(Sm(idx3L2), idx1L1, idx2L1) *
   node(idx1, idx2, idx3);
#EndProcedure
\end{lstlisting}
\end{maxipage}
It should be noted that \golemv{} expects the procedure \texttt{VertexConstants}
to exist in both cases. If all the constants are already substituted inside
\texttt{ReplaceVertices} the file must still provide a possibly empty empty
implementation of \texttt{VertexConstants}. \golemv{} ensures that
\texttt{VertexConstants} is always called after \texttt{ReplaceVertices}.

It is recommended to wrap any factors that are global prefactors to the diagram
into the argument of the function \texttt{PREFACTOR} as \golemv{} scans for these
functions and brackets them out. Each vertex definition must contain a factor
\texttt{node} which contains the indices\footnote{In \qgraf's terminology
these indices are a combination of vertex and ray index of the field.}
of the fields at this vertex.

The \qgraf{} style file generates vertex functions as follows:
\begin{multline*}
\mathtt{vertex}(\mathrm{vertex\,index},\\
   \mathrm{field}_1, \mathrm{index}_1, \pm2\mathrm{spin}_1, \mathrm{momentum}_1, \mu_1, \pm\mathrm{color\,rep}_1, %
   \mathrm{color\,index}_1,\\
   \mathrm{field}_2, \mathrm{index}_2, \pm2\mathrm{spin}_2, \mathrm{momentum}_2, \mu_2, \pm\mathrm{color\,rep}_2, %
   \mathrm{color\,index}_2,\\
   \vdots\\
   \mathrm{field}_n, \mathrm{index}_n, \pm2\mathrm{spin}_n, \mathrm{momentum}_n, \mu_n, \pm\mathrm{color\,rep}_n, %
   \mathrm{color\,index}_n)
\end{multline*}

The entries are:
\begin{description}
\item[vertex index] The unique index of this vertex. (\texttt{iv1}, \texttt{iv2}, \dots)
\item[$\mathrm{field}_i$] The field name of the $i$-th particle. These names are constructed
from the \qgraf{} field name as \texttt{[field.$\langle name\rangle$]}.
\item[$\mathrm{index}_i$] A unique name for this ``ray'' (at index $1$ they are \texttt{idx1r1},
   \texttt{idx1r2}, \ldots)
\item[$\pm2\mathrm{spin}_i$] twice the spin of the $i$-th particle.
   The sign distinguishes particles~($+$) from antiparticles~($-$).
\item[$\mathrm{momentum}_i$] the incoming momentum of the $i$-th particle.
\item[$\mu_i$] the Lorentz index of the $i$-th particle. Depending on the spin of the particle
   this is a spinor index (spin $1/2$), a Lorentz index (spin $1$) or a dummy index (spin $0$).
   For higher spins this index must be split into its components using the function
   \texttt{SplitLorentzIndex}. For its proper definition the reader is referred to
   the document \texttt{src/form/lorentz.pdf}.
\item[$\pm\mathrm{color\,rep}_i$] the color representation of the $i$-th particle. Allowed
   values currently are $\pm1,\pm3,\pm8$, although the sign only really makes sense for the
   fundamental representation $3$ and its conjugate $\bar{3}\equiv-3$.
\item[$\mathrm{color\,index}_i$] The color index of the $i$-th particle. Depending on the color
   representation this is an index in the fundamental, the adjoint or the trivial representation.
\end{description}

All symbols defined in \texttt{src/form/symbols.hh} are also accessible in this \form{} file.
\attention Note: until recently the definition of \texttt{Sqrt2} and \texttt{sqrt2} were part
of the model file. Now these symbols are part of \texttt{src/form/symbols.hh} and must not be
redefined.

\attention All Dirac matrices and metric tensors must use the notation introduced by \texttt{spinney}.
The metric tensor is $g^{\mu\nu}=\mathtt{d}(\mu, \nu)$ and $\gamma^\mu=\mathtt{Sm}(\mu)$,
$\gamma_5=\mathtt{Gamma5}$, $\Pi_+=\mathtt{ProjPlus}$, $\Pi_-=\mathtt{ProjMinus}$. All non-commuting
objects must reside inside the function \texttt{NCContainter} (see~example).

The color structure must use the objects $t_{ij}^A=\mathtt{T}(A, i, j)$ (where the color flow is such
that$j$ is the index of an anti-quark), $f^{ABC}=\mathtt{f}(A, B, C)$ and
$f^{ABE}f^{CDE}=\mathtt{f4}(A,B,C,D)$. At vertices coupling colored with colorless particles
it might be necessary to use the \texttt{d\_} tensor to file the color flow through the vertex.

\attention Note that all propagators and wave functions are defined in a model independent
way in the files \texttt{src/form/propagators.hh} and \texttt{src/form/legs.hh}. Please,
refrain from modifying these files directly but make all changes to \texttt{src/form/lorentz.nw}.

In theories with Maiorana fermions the model file should include the following
line:
\begin{lstlisting}[language=form]
#Define DISPOSEQGRAFSIGN "1"
\end{lstlisting}

\section{Standard Model (\texttt{sm})}
\label{sec:model-files:sm}
\subsection{Synopsis}
The model `\texttt{sm}' contains the Feynman rules for the
Standard Model in Feynman gauge as described
in~\cite[Appendix~A]{Boehm:2001}.

\subsection{Particle Content}
\marginlabel{Leptons}
\begin{tabular}{|l|l|l|p{2cm}|}
\hline
Name&Alternative Names&Mass&Comment\\
\hline
\tt ep & \tt positron e+ & \tt me& $e^+$\\
\tt em & \tt electron e- & \tt me& $e^-$\\
\tt ne & & $0$ & $\nu_e$\\
\tt nebar & \tt ne\~& $0$ & $\bar{\nu}_e$\\
\hline
\tt mup & \tt mu+ & \tt mmu& $\mu^+$\\
\tt mum & \tt mu- & \tt mmu& $\mu^-$\\
\tt nmu & & $0$ & $\nu_\mu$\\
\tt nmubar & \tt nmu\~ & $0$ & $\bar{\nu}_\mu$\\
\hline
\tt taup & \tt tau+ & \tt mtau& $e^+$\\
\tt taum & \tt tau- & \tt mtau& $e^-$\\
\tt ntau & & $0$ & $\nu_\tau$\\
\tt ntaubar & \tt ntau\~ & $0$ & $\bar{\nu}_\tau$\\
\hline
\end{tabular}

\marginlabel{Quarks}
\begin{tabular}{|l|l|l|p{2cm}|}
\hline
Name&Alternative Names&Mass&Comment\\
\hline
\tt U & \tt u & \tt mU& $u$\\
\tt Ubar & \tt u\~ & \tt mU& $\bar{u}$\\
\tt D & d & \tt mD & $d$\\
\tt Dbar & \tt d\~& mD & $\bar{d}$\\
\hline
\tt S & \tt s & \tt mS& $u$\\
\tt Sbar & \tt s\~ & \tt mS& $\bar{u}$\\
\tt C & c & \tt mC & $d$\\
\tt Cbar & \tt c\~& mC & $\bar{d}$\\
\hline
\tt T & \tt t & \tt mT& $t$\\
\tt Tbar & \tt t\~ & \tt mT& $\bar{t}$\\
\tt B & b & \tt mB & $b$\\
\tt Bbar & \tt b\~& mB & $\bar{b}$\\
\hline
\end{tabular}

\marginlabel{Gauge Bosons}
\begin{tabular}{|l|l|l|p{2cm}|}
\hline
Name&Alternative Names&Mass&Comment\\
\hline
\tt g & \tt gluon & $0$ & $g$ \\
\tt A & \tt photon gamma & $0$ & $\gamma$ \\
\tt Z & & \tt mZ & $Z$ \\
\tt Wp & \tt W+& \tt mW & $W^+$ \\
\tt Wm & \tt W-& \tt mW & $W^-$ \\
\hline
\end{tabular}

\marginlabel{Scalar Bosons}
\begin{tabular}{|l|l|l|p{2cm}|}
\hline
Name&Alternative Names&Mass&Comment\\
\hline
\tt H & \tt h higgs & \tt mH & $H$ \\
\tt phim & \tt phi- & \tt mW & $\phi^-$ \\
\tt phip & \tt phi+ & \tt mW & $\phi^+$ \\
\tt chi &  & \tt mZ & $\chi$ \\
\hline
\end{tabular}

\marginlabel{Ghost Fields}
\begin{tabular}{|l|l|l|p{2cm}|}
\hline
Name&Alternative Names&Mass&Comment\\
\hline
\tt gh &  & $0$ & $u^g$\\
\tt ghbar &  & $0$ & $\bar{u}^g$ \\
\tt ghA &  & $0$ & $u^A$ \\
\tt ghAbar &  & $0$ & $\bar{u}^A$ \\
\tt ghZ &  & \tt mZ & $u^Z$ \\
\tt ghZbar &  & \tt mZ & $\bar{u}^Z$ \\
\tt ghWp &  & \tt mW & $u^+$ \\
\tt ghWpbar &  & \tt mW & $\bar{u}^+$ \\
\tt ghWm &  & \tt mW & $u^-$ \\
\tt ghWmbar &  & \tt mW & $\bar{u}^-$ \\
\hline
\end{tabular}

\subsection{Parameters}
This section lists all model parameters which are not already
listed as particle masses.

\medskip
\begin{tabular}{|l|l|l|}
\hline
Name & Symbol & Description\\
\hline
\tt NC & $N_C$ & Number of colors in QCD\\
\tt e & $e$ & electro-weak coupling constant: $\alpha=e^2/(4\pi)$\\
\tt gs & $g_s$ & strong coupling constant: $\alpha_s=g_s^2/(4\pi)$\\
\tt sw & $s_w=\sin\theta_w$ & sine of weak mixing angle\\
\tt cw & $c_w=\cos\theta_w$ & cosine of weak mixing angle\\
\tt VUD & $V_{ud}$ & CKM mixing matrix element\\
\tt CVDU & $V_{du}^{\dagger}$ & --- '' ---\\
\tt VUS & $V_{us}$ & --- '' ---\\
\tt CVSU & $V_{su}^{\dagger}$ & --- '' ---\\
\tt VUB & $V_{ub}$ & --- '' ---\\
\tt CVBU & $V_{bu}^{\dagger}$ & --- '' ---\\
\tt VCD & $V_{cd}$ & --- '' ---\\
\tt CVDC & $V_{dc}^{\dagger}$ & --- '' ---\\
\tt VCS & $V_{cs}$ & --- '' ---\\
\tt CVSC & $V_{sc}^{\dagger}$ & --- '' ---\\
\tt VCB & $V_{cb}$ & --- '' ---\\
\tt CVBC & $V_{bc}^{\dagger}$ & --- '' ---\\
\tt VTD & $V_{td}$ & --- '' ---\\
\tt CVTD & $V_{dt}^{\dagger}$ & --- '' ---\\
\tt VTS & $V_{ts}$ & --- '' ---\\
\tt CVST & $V_{st}^{\dagger}$ & --- '' ---\\
\tt VTB & $V_{tb}$ & --- '' ---\\
\tt CVTB & $V_{bt}^{\dagger}$ & --- '' ---\\
\hline
\end{tabular}

\chapter{Template for a Process Setup File}
\label{chp:appendix-template.in}
\seealso{Chapter \ref{chp:setup-of-a-process}}
In order to create a new process setup file one can invoke
\begin{example}
\$ gosam.py --template your\_new\_file.in
\end{example}
This is the recommended way of obtaining the most recent
documentation of the available options.

The syntax of a general process setup file should obey the following rules:
\begin{itemize}
\item A setup file ({\em process card}) consists of a sequence of lines representing
key-value pairs. A key-value pair can span across several lines
if each of the lines except the last line is terminated by a backslash.
\item A setup file is allowed to contain any number of blank lines or
comment lines, 
indicated by a `\lstinline[basicstyle=\ttfamily]{!}' or a
`\lstinline[basicstyle=\ttfamily]{#}' as its first non-blank character.
\item The key and the value are separated by a blank, a colon
`\lstinline[basicstyle=\ttfamily]{:}' or
an equals sign `\lstinline[basicstyle=\ttfamily]{=}'. Notice that the line
`\lstinline[showspaces=true,basicstyle=\ttfamily]{key = value}'
will be interpreted as the key `\lstinline[basicstyle=\ttfamily]{key}'
followed by the value
`\lstinline[showspaces=true,basicstyle=\ttfamily]{= value}'
as the terminator of the key is the blank and not the equals sign.
In order to produce one of the terminators literally as a part of the
key one has to escape it with a backslash, e.g.
`\lstinline[showspaces=true,basicstyle=\ttfamily]{very\ long\ name:value}'
would translate to the key
`\lstinline[showspaces=true,basicstyle=\ttfamily]{very long name}' and the
value `\lstinline[basicstyle=\ttfamily]{value}'.
\item The escape characters
`\lstinline[basicstyle=\ttfamily]{\\}',
`\lstinline[basicstyle=\ttfamily]{\n}',
`\lstinline[basicstyle=\ttfamily]{\r}',
`\lstinline[basicstyle=\ttfamily]{\f}' and
`\lstinline[basicstyle=\ttfamily]{\t}' work as usual. Backslashes in front
of any other character are just dropped.
\item Leading and trailing blanks are removed from the key and the value
by default and must be escaped to preserve them. Whitespace is also
removed in front and after commas if the value is interpreted as a
comma separated list.
\item If an option expects a logical value, the literals
`\lstinline[basicstyle=\ttfamily]{1}',
`\lstinline[basicstyle=\ttfamily]{true}',
`\lstinline[basicstyle=\ttfamily]{.true.}',
`\lstinline[basicstyle=\ttfamily]{t}',
`\lstinline[basicstyle=\ttfamily]{.t.}',
`\lstinline[basicstyle=\ttfamily]{yes}' and
`\lstinline[basicstyle=\ttfamily]{.y.}'
are recognized as the value \emph{true}. These values are interpreted
case-insensitively. If a value is not recognized as \emph{true} it
corresponds to \emph{false}. 
\end{itemize}
\attention Note, that deviating from the Java standard, unicode
escapes, such as `\lstinline[basicstyle=\ttfamily]{\u10EF}', have not
been implemented; neither are octal and hexadecimal escape sequences
recognized.

\begin{fullpage}
\input template.tex
%\lstinputlisting[basicstyle=\ttfamily\small,frame=tblr,%
%	caption={\texttt{template.in}}]{template.in}
\end{fullpage}

\chapter*{Conditions of Use}
    GoSam -- An automated One-Loop matrix element generator.\\
    Copyright (C) 2011, 2012  The GoSam Collaboration\\
    \begin{itemize}               
                \item Gavin Cullen
                \item Nicolas Greiner
                \item Gudrun Heinrich
                \item Gionata Luisoni
                \item Pierpaolo Mastrolia
                \item Giovanni Ossola
                \item Thomas Reiter
                \item Francesco Tramontano
    \end{itemize}

    This program is free software: you can redistribute it and/or modify
    it under the terms of the GNU General Public License as published by
    the Free Software Foundation, either version 3 of the License, or
    (at your option) any later version.

    This program is distributed in the hope that it will be useful,
    but WITHOUT ANY WARRANTY; without even the implied warranty of
    MERCHANTABILITY or FITNESS FOR A PARTICULAR PURPOSE.  See the
    GNU General Public License for more details.

    You should have received a copy of the GNU General Public License
    along with this program.  If not, see \url{http://www.gnu.org/licenses/}.

    Scientific publications prepared using the present version of
    GoSam or any modified version of it or any code linking to
    GoSam or parts of it should make a clear reference to the publication:

    \begin{quote}
        G. Cullen et al.,\\
        ``Automated One-Loop Calculations with GoSam,''\\
        arXiv:1111.2034 [hep-ph]\\
    \end{quote}

\section*{The GNU General Public License Version 3}

\begin{fullpage}
\begin{center}
{\parindent 0in

Copyright \copyright\  2007 Free Software Foundation, Inc. \texttt{http://fsf.org/}

\bigskip
Everyone is permitted to copy and distribute verbatim copies of this

license document, but changing it is not allowed.}

\end{center}

\begin{center}
{\Large \sc Preamble}
\end{center}
The GNU General Public License is a free, copyleft license for
software and other kinds of works.

The licenses for most software and other practical works are designed
to take away your freedom to share and change the works.  By contrast,
the GNU General Public License is intended to guarantee your freedom to
share and change all versions of a program--to make sure it remains free
software for all its users.  We, the Free Software Foundation, use the
GNU General Public License for most of our software; it applies also to
any other work released this way by its authors.  You can apply it to
your programs, too.

When we speak of free software, we are referring to freedom, not
price.  Our General Public Licenses are designed to make sure that you
have the freedom to distribute copies of free software (and charge for
them if you wish), that you receive source code or can get it if you
want it, that you can change the software or use pieces of it in new
free programs, and that you know you can do these things.

To protect your rights, we need to prevent others from denying you
these rights or asking you to surrender the rights.  Therefore, you have
certain responsibilities if you distribute copies of the software, or if
you modify it: responsibilities to respect the freedom of others.

For example, if you distribute copies of such a program, whether
gratis or for a fee, you must pass on to the recipients the same
freedoms that you received.  You must make sure that they, too, receive
or can get the source code.  And you must show them these terms so they
know their rights.

Developers that use the GNU GPL protect your rights with two steps:
(1) assert copyright on the software, and (2) offer you this License
giving you legal permission to copy, distribute and/or modify it.

For the developers' and authors' protection, the GPL clearly explains
that there is no warranty for this free software.  For both users' and
authors' sake, the GPL requires that modified versions be marked as
changed, so that their problems will not be attributed erroneously to
authors of previous versions.

Some devices are designed to deny users access to install or run
modified versions of the software inside them, although the manufacturer
can do so.  This is fundamentally incompatible with the aim of
protecting users' freedom to change the software.  The systematic
pattern of such abuse occurs in the area of products for individuals to
use, which is precisely where it is most unacceptable.  Therefore, we
have designed this version of the GPL to prohibit the practice for those
products.  If such problems arise substantially in other domains, we
stand ready to extend this provision to those domains in future versions
of the GPL, as needed to protect the freedom of users.

Finally, every program is threatened constantly by software patents.
States should not allow patents to restrict development and use of
software on general-purpose computers, but in those that do, we wish to
avoid the special danger that patents applied to a free program could
make it effectively proprietary.  To prevent this, the GPL assures that
patents cannot be used to render the program non-free.

The precise terms and conditions for copying, distribution and
modification follow.

\begin{center}
{\Large \sc Terms and Conditions}
\end{center}


\begin{enumerate}

\addtocounter{enumi}{-1}

\item Definitions.

``This License'' refers to version 3 of the GNU General Public License.

``Copyright'' also means copyright-like laws that apply to other kinds of
works, such as semiconductor masks.

``The Program'' refers to any copyrightable work licensed under this
License.  Each licensee is addressed as ``you''.  ``Licensees'' and
``recipients'' may be individuals or organizations.

To ``modify'' a work means to copy from or adapt all or part of the work
in a fashion requiring copyright permission, other than the making of an
exact copy.  The resulting work is called a ``modified version'' of the
earlier work or a work ``based on'' the earlier work.

A ``covered work'' means either the unmodified Program or a work based
on the Program.

To ``propagate'' a work means to do anything with it that, without
permission, would make you directly or secondarily liable for
infringement under applicable copyright law, except executing it on a
computer or modifying a private copy.  Propagation includes copying,
distribution (with or without modification), making available to the
public, and in some countries other activities as well.

To ``convey'' a work means any kind of propagation that enables other
parties to make or receive copies.  Mere interaction with a user through
a computer network, with no transfer of a copy, is not conveying.

An interactive user interface displays ``Appropriate Legal Notices''
to the extent that it includes a convenient and prominently visible
feature that (1) displays an appropriate copyright notice, and (2)
tells the user that there is no warranty for the work (except to the
extent that warranties are provided), that licensees may convey the
work under this License, and how to view a copy of this License.  If
the interface presents a list of user commands or options, such as a
menu, a prominent item in the list meets this criterion.

\item Source Code.

The ``source code'' for a work means the preferred form of the work
for making modifications to it.  ``Object code'' means any non-source
form of a work.

A ``Standard Interface'' means an interface that either is an official
standard defined by a recognized standards body, or, in the case of
interfaces specified for a particular programming language, one that
is widely used among developers working in that language.

The ``System Libraries'' of an executable work include anything, other
than the work as a whole, that (a) is included in the normal form of
packaging a Major Component, but which is not part of that Major
Component, and (b) serves only to enable use of the work with that
Major Component, or to implement a Standard Interface for which an
implementation is available to the public in source code form.  A
``Major Component'', in this context, means a major essential component
(kernel, window system, and so on) of the specific operating system
(if any) on which the executable work runs, or a compiler used to
produce the work, or an object code interpreter used to run it.

The ``Corresponding Source'' for a work in object code form means all
the source code needed to generate, install, and (for an executable
work) run the object code and to modify the work, including scripts to
control those activities.  However, it does not include the work's
System Libraries, or general-purpose tools or generally available free
programs which are used unmodified in performing those activities but
which are not part of the work.  For example, Corresponding Source
includes interface definition files associated with source files for
the work, and the source code for shared libraries and dynamically
linked subprograms that the work is specifically designed to require,
such as by intimate data communication or control flow between those
subprograms and other parts of the work.

The Corresponding Source need not include anything that users
can regenerate automatically from other parts of the Corresponding
Source.

The Corresponding Source for a work in source code form is that
same work.

\item Basic Permissions.

All rights granted under this License are granted for the term of
copyright on the Program, and are irrevocable provided the stated
conditions are met.  This License explicitly affirms your unlimited
permission to run the unmodified Program.  The output from running a
covered work is covered by this License only if the output, given its
content, constitutes a covered work.  This License acknowledges your
rights of fair use or other equivalent, as provided by copyright law.

You may make, run and propagate covered works that you do not
convey, without conditions so long as your license otherwise remains
in force.  You may convey covered works to others for the sole purpose
of having them make modifications exclusively for you, or provide you
with facilities for running those works, provided that you comply with
the terms of this License in conveying all material for which you do
not control copyright.  Those thus making or running the covered works
for you must do so exclusively on your behalf, under your direction
and control, on terms that prohibit them from making any copies of
your copyrighted material outside their relationship with you.

Conveying under any other circumstances is permitted solely under
the conditions stated below.  Sublicensing is not allowed; section 10
makes it unnecessary.

\item Protecting Users' Legal Rights From Anti-Circumvention Law.

No covered work shall be deemed part of an effective technological
measure under any applicable law fulfilling obligations under article
11 of the WIPO copyright treaty adopted on 20 December 1996, or
similar laws prohibiting or restricting circumvention of such
measures.

When you convey a covered work, you waive any legal power to forbid
circumvention of technological measures to the extent such circumvention
is effected by exercising rights under this License with respect to
the covered work, and you disclaim any intention to limit operation or
modification of the work as a means of enforcing, against the work's
users, your or third parties' legal rights to forbid circumvention of
technological measures.

\item Conveying Verbatim Copies.

You may convey verbatim copies of the Program's source code as you
receive it, in any medium, provided that you conspicuously and
appropriately publish on each copy an appropriate copyright notice;
keep intact all notices stating that this License and any
non-permissive terms added in accord with section 7 apply to the code;
keep intact all notices of the absence of any warranty; and give all
recipients a copy of this License along with the Program.

You may charge any price or no price for each copy that you convey,
and you may offer support or warranty protection for a fee.

\item Conveying Modified Source Versions.

You may convey a work based on the Program, or the modifications to
produce it from the Program, in the form of source code under the
terms of section 4, provided that you also meet all of these conditions:
  \begin{enumerate}
  \item The work must carry prominent notices stating that you modified
  it, and giving a relevant date.

  \item The work must carry prominent notices stating that it is
  released under this License and any conditions added under section
  7.  This requirement modifies the requirement in section 4 to
  ``keep intact all notices''.

  \item You must license the entire work, as a whole, under this
  License to anyone who comes into possession of a copy.  This
  License will therefore apply, along with any applicable section 7
  additional terms, to the whole of the work, and all its parts,
  regardless of how they are packaged.  This License gives no
  permission to license the work in any other way, but it does not
  invalidate such permission if you have separately received it.

  \item If the work has interactive user interfaces, each must display
  Appropriate Legal Notices; however, if the Program has interactive
  interfaces that do not display Appropriate Legal Notices, your
  work need not make them do so.
\end{enumerate}
A compilation of a covered work with other separate and independent
works, which are not by their nature extensions of the covered work,
and which are not combined with it such as to form a larger program,
in or on a volume of a storage or distribution medium, is called an
``aggregate'' if the compilation and its resulting copyright are not
used to limit the access or legal rights of the compilation's users
beyond what the individual works permit.  Inclusion of a covered work
in an aggregate does not cause this License to apply to the other
parts of the aggregate.

\item Conveying Non-Source Forms.

You may convey a covered work in object code form under the terms
of sections 4 and 5, provided that you also convey the
machine-readable Corresponding Source under the terms of this License,
in one of these ways:
  \begin{enumerate}
  \item Convey the object code in, or embodied in, a physical product
  (including a physical distribution medium), accompanied by the
  Corresponding Source fixed on a durable physical medium
  customarily used for software interchange.

  \item Convey the object code in, or embodied in, a physical product
  (including a physical distribution medium), accompanied by a
  written offer, valid for at least three years and valid for as
  long as you offer spare parts or customer support for that product
  model, to give anyone who possesses the object code either (1) a
  copy of the Corresponding Source for all the software in the
  product that is covered by this License, on a durable physical
  medium customarily used for software interchange, for a price no
  more than your reasonable cost of physically performing this
  conveying of source, or (2) access to copy the
  Corresponding Source from a network server at no charge.

  \item Convey individual copies of the object code with a copy of the
  written offer to provide the Corresponding Source.  This
  alternative is allowed only occasionally and noncommercially, and
  only if you received the object code with such an offer, in accord
  with subsection 6b.

  \item Convey the object code by offering access from a designated
  place (gratis or for a charge), and offer equivalent access to the
  Corresponding Source in the same way through the same place at no
  further charge.  You need not require recipients to copy the
  Corresponding Source along with the object code.  If the place to
  copy the object code is a network server, the Corresponding Source
  may be on a different server (operated by you or a third party)
  that supports equivalent copying facilities, provided you maintain
  clear directions next to the object code saying where to find the
  Corresponding Source.  Regardless of what server hosts the
  Corresponding Source, you remain obligated to ensure that it is
  available for as long as needed to satisfy these requirements.

  \item Convey the object code using peer-to-peer transmission, provided
  you inform other peers where the object code and Corresponding
  Source of the work are being offered to the general public at no
  charge under subsection 6d.
  \end{enumerate}

A separable portion of the object code, whose source code is excluded
from the Corresponding Source as a System Library, need not be
included in conveying the object code work.

A ``User Product'' is either (1) a ``consumer product'', which means any
tangible personal property which is normally used for personal, family,
or household purposes, or (2) anything designed or sold for incorporation
into a dwelling.  In determining whether a product is a consumer product,
doubtful cases shall be resolved in favor of coverage.  For a particular
product received by a particular user, ``normally used'' refers to a
typical or common use of that class of product, regardless of the status
of the particular user or of the way in which the particular user
actually uses, or expects or is expected to use, the product.  A product
is a consumer product regardless of whether the product has substantial
commercial, industrial or non-consumer uses, unless such uses represent
the only significant mode of use of the product.

``Installation Information'' for a User Product means any methods,
procedures, authorization keys, or other information required to install
and execute modified versions of a covered work in that User Product from
a modified version of its Corresponding Source.  The information must
suffice to ensure that the continued functioning of the modified object
code is in no case prevented or interfered with solely because
modification has been made.

If you convey an object code work under this section in, or with, or
specifically for use in, a User Product, and the conveying occurs as
part of a transaction in which the right of possession and use of the
User Product is transferred to the recipient in perpetuity or for a
fixed term (regardless of how the transaction is characterized), the
Corresponding Source conveyed under this section must be accompanied
by the Installation Information.  But this requirement does not apply
if neither you nor any third party retains the ability to install
modified object code on the User Product (for example, the work has
been installed in ROM).

The requirement to provide Installation Information does not include a
requirement to continue to provide support service, warranty, or updates
for a work that has been modified or installed by the recipient, or for
the User Product in which it has been modified or installed.  Access to a
network may be denied when the modification itself materially and
adversely affects the operation of the network or violates the rules and
protocols for communication across the network.

Corresponding Source conveyed, and Installation Information provided,
in accord with this section must be in a format that is publicly
documented (and with an implementation available to the public in
source code form), and must require no special password or key for
unpacking, reading or copying.

\item Additional Terms.

``Additional permissions'' are terms that supplement the terms of this
License by making exceptions from one or more of its conditions.
Additional permissions that are applicable to the entire Program shall
be treated as though they were included in this License, to the extent
that they are valid under applicable law.  If additional permissions
apply only to part of the Program, that part may be used separately
under those permissions, but the entire Program remains governed by
this License without regard to the additional permissions.

When you convey a copy of a covered work, you may at your option
remove any additional permissions from that copy, or from any part of
it.  (Additional permissions may be written to require their own
removal in certain cases when you modify the work.)  You may place
additional permissions on material, added by you to a covered work,
for which you have or can give appropriate copyright permission.

Notwithstanding any other provision of this License, for material you
add to a covered work, you may (if authorized by the copyright holders of
that material) supplement the terms of this License with terms:
  \begin{enumerate}
  \item Disclaiming warranty or limiting liability differently from the
  terms of sections 15 and 16 of this License; or

  \item Requiring preservation of specified reasonable legal notices or
  author attributions in that material or in the Appropriate Legal
  Notices displayed by works containing it; or

  \item Prohibiting misrepresentation of the origin of that material, or
  requiring that modified versions of such material be marked in
  reasonable ways as different from the original version; or

  \item Limiting the use for publicity purposes of names of licensors or
  authors of the material; or

  \item Declining to grant rights under trademark law for use of some
  trade names, trademarks, or service marks; or

  \item Requiring indemnification of licensors and authors of that
  material by anyone who conveys the material (or modified versions of
  it) with contractual assumptions of liability to the recipient, for
  any liability that these contractual assumptions directly impose on
  those licensors and authors.
  \end{enumerate}

All other non-permissive additional terms are considered ``further
restrictions'' within the meaning of section 10.  If the Program as you
received it, or any part of it, contains a notice stating that it is
governed by this License along with a term that is a further
restriction, you may remove that term.  If a license document contains
a further restriction but permits relicensing or conveying under this
License, you may add to a covered work material governed by the terms
of that license document, provided that the further restriction does
not survive such relicensing or conveying.

If you add terms to a covered work in accord with this section, you
must place, in the relevant source files, a statement of the
additional terms that apply to those files, or a notice indicating
where to find the applicable terms.

Additional terms, permissive or non-permissive, may be stated in the
form of a separately written license, or stated as exceptions;
the above requirements apply either way.

\item Termination.

You may not propagate or modify a covered work except as expressly
provided under this License.  Any attempt otherwise to propagate or
modify it is void, and will automatically terminate your rights under
this License (including any patent licenses granted under the third
paragraph of section 11).

However, if you cease all violation of this License, then your
license from a particular copyright holder is reinstated (a)
provisionally, unless and until the copyright holder explicitly and
finally terminates your license, and (b) permanently, if the copyright
holder fails to notify you of the violation by some reasonable means
prior to 60 days after the cessation.

Moreover, your license from a particular copyright holder is
reinstated permanently if the copyright holder notifies you of the
violation by some reasonable means, this is the first time you have
received notice of violation of this License (for any work) from that
copyright holder, and you cure the violation prior to 30 days after
your receipt of the notice.

Termination of your rights under this section does not terminate the
licenses of parties who have received copies or rights from you under
this License.  If your rights have been terminated and not permanently
reinstated, you do not qualify to receive new licenses for the same
material under section 10.

\item Acceptance Not Required for Having Copies.

You are not required to accept this License in order to receive or
run a copy of the Program.  Ancillary propagation of a covered work
occurring solely as a consequence of using peer-to-peer transmission
to receive a copy likewise does not require acceptance.  However,
nothing other than this License grants you permission to propagate or
modify any covered work.  These actions infringe copyright if you do
not accept this License.  Therefore, by modifying or propagating a
covered work, you indicate your acceptance of this License to do so.

\item Automatic Licensing of Downstream Recipients.

Each time you convey a covered work, the recipient automatically
receives a license from the original licensors, to run, modify and
propagate that work, subject to this License.  You are not responsible
for enforcing compliance by third parties with this License.

An ``entity transaction'' is a transaction transferring control of an
organization, or substantially all assets of one, or subdividing an
organization, or merging organizations.  If propagation of a covered
work results from an entity transaction, each party to that
transaction who receives a copy of the work also receives whatever
licenses to the work the party's predecessor in interest had or could
give under the previous paragraph, plus a right to possession of the
Corresponding Source of the work from the predecessor in interest, if
the predecessor has it or can get it with reasonable efforts.

You may not impose any further restrictions on the exercise of the
rights granted or affirmed under this License.  For example, you may
not impose a license fee, royalty, or other charge for exercise of
rights granted under this License, and you may not initiate litigation
(including a cross-claim or counterclaim in a lawsuit) alleging that
any patent claim is infringed by making, using, selling, offering for
sale, or importing the Program or any portion of it.

\item Patents.

A ``contributor'' is a copyright holder who authorizes use under this
License of the Program or a work on which the Program is based.  The
work thus licensed is called the contributor's ``contributor version''.

A contributor's ``essential patent claims'' are all patent claims
owned or controlled by the contributor, whether already acquired or
hereafter acquired, that would be infringed by some manner, permitted
by this License, of making, using, or selling its contributor version,
but do not include claims that would be infringed only as a
consequence of further modification of the contributor version.  For
purposes of this definition, ``control'' includes the right to grant
patent sublicenses in a manner consistent with the requirements of
this License.

Each contributor grants you a non-exclusive, worldwide, royalty-free
patent license under the contributor's essential patent claims, to
make, use, sell, offer for sale, import and otherwise run, modify and
propagate the contents of its contributor version.

In the following three paragraphs, a ``patent license'' is any express
agreement or commitment, however denominated, not to enforce a patent
(such as an express permission to practice a patent or covenant not to
sue for patent infringement).  To ``grant'' such a patent license to a
party means to make such an agreement or commitment not to enforce a
patent against the party.

If you convey a covered work, knowingly relying on a patent license,
and the Corresponding Source of the work is not available for anyone
to copy, free of charge and under the terms of this License, through a
publicly available network server or other readily accessible means,
then you must either (1) cause the Corresponding Source to be so
available, or (2) arrange to deprive yourself of the benefit of the
patent license for this particular work, or (3) arrange, in a manner
consistent with the requirements of this License, to extend the patent
license to downstream recipients.  ``Knowingly relying'' means you have
actual knowledge that, but for the patent license, your conveying the
covered work in a country, or your recipient's use of the covered work
in a country, would infringe one or more identifiable patents in that
country that you have reason to believe are valid.

If, pursuant to or in connection with a single transaction or
arrangement, you convey, or propagate by procuring conveyance of, a
covered work, and grant a patent license to some of the parties
receiving the covered work authorizing them to use, propagate, modify
or convey a specific copy of the covered work, then the patent license
you grant is automatically extended to all recipients of the covered
work and works based on it.

A patent license is ``discriminatory'' if it does not include within
the scope of its coverage, prohibits the exercise of, or is
conditioned on the non-exercise of one or more of the rights that are
specifically granted under this License.  You may not convey a covered
work if you are a party to an arrangement with a third party that is
in the business of distributing software, under which you make payment
to the third party based on the extent of your activity of conveying
the work, and under which the third party grants, to any of the
parties who would receive the covered work from you, a discriminatory
patent license (a) in connection with copies of the covered work
conveyed by you (or copies made from those copies), or (b) primarily
for and in connection with specific products or compilations that
contain the covered work, unless you entered into that arrangement,
or that patent license was granted, prior to 28 March 2007.

Nothing in this License shall be construed as excluding or limiting
any implied license or other defenses to infringement that may
otherwise be available to you under applicable patent law.

\item No Surrender of Others' Freedom.

If conditions are imposed on you (whether by court order, agreement or
otherwise) that contradict the conditions of this License, they do not
excuse you from the conditions of this License.  If you cannot convey a
covered work so as to satisfy simultaneously your obligations under this
License and any other pertinent obligations, then as a consequence you may
not convey it at all.  For example, if you agree to terms that obligate you
to collect a royalty for further conveying from those to whom you convey
the Program, the only way you could satisfy both those terms and this
License would be to refrain entirely from conveying the Program.

\item Use with the GNU Affero General Public License.

Notwithstanding any other provision of this License, you have
permission to link or combine any covered work with a work licensed
under version 3 of the GNU Affero General Public License into a single
combined work, and to convey the resulting work.  The terms of this
License will continue to apply to the part which is the covered work,
but the special requirements of the GNU Affero General Public License,
section 13, concerning interaction through a network will apply to the
combination as such.

\item Revised Versions of this License.

The Free Software Foundation may publish revised and/or new versions of
the GNU General Public License from time to time.  Such new versions will
be similar in spirit to the present version, but may differ in detail to
address new problems or concerns.

Each version is given a distinguishing version number.  If the
Program specifies that a certain numbered version of the GNU General
Public License ``or any later version'' applies to it, you have the
option of following the terms and conditions either of that numbered
version or of any later version published by the Free Software
Foundation.  If the Program does not specify a version number of the
GNU General Public License, you may choose any version ever published
by the Free Software Foundation.

If the Program specifies that a proxy can decide which future
versions of the GNU General Public License can be used, that proxy's
public statement of acceptance of a version permanently authorizes you
to choose that version for the Program.

Later license versions may give you additional or different
permissions.  However, no additional obligations are imposed on any
author or copyright holder as a result of your choosing to follow a
later version.

\item Disclaimer of Warranty.

\begin{sloppypar}
 THERE IS NO WARRANTY FOR THE PROGRAM, TO THE EXTENT PERMITTED BY
 APPLICABLE LAW.  EXCEPT WHEN OTHERWISE STATED IN WRITING THE
 COPYRIGHT HOLDERS AND/OR OTHER PARTIES PROVIDE THE PROGRAM ``AS IS''
 WITHOUT WARRANTY OF ANY KIND, EITHER EXPRESSED OR IMPLIED,
 INCLUDING, BUT NOT LIMITED TO, THE IMPLIED WARRANTIES OF
 MERCHANTABILITY AND FITNESS FOR A PARTICULAR PURPOSE.  THE ENTIRE
 RISK AS TO THE QUALITY AND PERFORMANCE OF THE PROGRAM IS WITH YOU.
 SHOULD THE PROGRAM PROVE DEFECTIVE, YOU ASSUME THE COST OF ALL
 NECESSARY SERVICING, REPAIR OR CORRECTION.
\end{sloppypar}

\item Limitation of Liability.

 IN NO EVENT UNLESS REQUIRED BY APPLICABLE LAW OR AGREED TO IN
 WRITING WILL ANY COPYRIGHT HOLDER, OR ANY OTHER PARTY WHO MODIFIES
 AND/OR CONVEYS THE PROGRAM AS PERMITTED ABOVE, BE LIABLE TO YOU FOR
 DAMAGES, INCLUDING ANY GENERAL, SPECIAL, INCIDENTAL OR CONSEQUENTIAL
 DAMAGES ARISING OUT OF THE USE OR INABILITY TO USE THE PROGRAM
 (INCLUDING BUT NOT LIMITED TO LOSS OF DATA OR DATA BEING RENDERED
 INACCURATE OR LOSSES SUSTAINED BY YOU OR THIRD PARTIES OR A FAILURE
 OF THE PROGRAM TO OPERATE WITH ANY OTHER PROGRAMS), EVEN IF SUCH
 HOLDER OR OTHER PARTY HAS BEEN ADVISED OF THE POSSIBILITY OF SUCH
 DAMAGES.

\item Interpretation of Sections 15 and 16.

If the disclaimer of warranty and limitation of liability provided
above cannot be given local legal effect according to their terms,
reviewing courts shall apply local law that most closely approximates
an absolute waiver of all civil liability in connection with the
Program, unless a warranty or assumption of liability accompanies a
copy of the Program in return for a fee.

\begin{center}
{\Large\sc End of Terms and Conditions}

\bigskip
How to Apply These Terms to Your New Programs
\end{center}

If you develop a new program, and you want it to be of the greatest
possible use to the public, the best way to achieve this is to make it
free software which everyone can redistribute and change under these terms.

To do so, attach the following notices to the program.  It is safest
to attach them to the start of each source file to most effectively
state the exclusion of warranty; and each file should have at least
the ``copyright'' line and a pointer to where the full notice is found.

{\footnotesize
\begin{verbatim}
<one line to give the program's name and a brief idea of what it does.>

Copyright (C) <textyear>  <name of author>

This program is free software: you can redistribute it and/or modify
it under the terms of the GNU General Public License as published by
the Free Software Foundation, either version 3 of the License, or
(at your option) any later version.

This program is distributed in the hope that it will be useful,
but WITHOUT ANY WARRANTY; without even the implied warranty of
MERCHANTABILITY or FITNESS FOR A PARTICULAR PURPOSE.  See the
GNU General Public License for more details.

You should have received a copy of the GNU General Public License
along with this program.  If not, see <http://www.gnu.org/licenses/>.
\end{verbatim}
}

Also add information on how to contact you by electronic and paper mail.

If the program does terminal interaction, make it output a short
notice like this when it starts in an interactive mode:

{\footnotesize
\begin{verbatim}
<program>  Copyright (C) <year>  <name of author>

This program comes with ABSOLUTELY NO WARRANTY; for details type `show w'.
This is free software, and you are welcome to redistribute it
under certain conditions; type `show c' for details.
\end{verbatim}
}

The hypothetical commands {\tt show w} and {\tt show c} should show
the appropriate
parts of the General Public License.  Of course, your program's commands
might be different; for a GUI interface, you would use an ``about box''.

You should also get your employer (if you work as a programmer) or
school, if any, to sign a ``copyright disclaimer'' for the program, if
necessary.  For more information on this, and how to apply and follow
the GNU GPL, see \texttt{http://www.gnu.org/licenses/}.

The GNU General Public License does not permit incorporating your
program into proprietary programs.  If your program is a subroutine
library, you may consider it more useful to permit linking proprietary
applications with the library.  If this is what you want to do, use
the GNU Lesser General Public License instead of this License.  But
first, please read \texttt{http://www.gnu.org/philosophy/why-not-lgpl.html}.

\end{enumerate}
\end{fullpage}

\bibliographystyle{amsalpha}
\begin{fullpage}
\bibliography{refman}
\end{fullpage}

\clearpage
\remarks

\end{document}
