\chapter{The Binoth Les Houches Accord Interface}
\label{sec:blha}

\section{Initialisation Phase}
The script \texttt{gosam.py --olp} which comes with \gosamv{} can be used to generate
matrix elements compatible with the specifications of the Binoth Les Houches
Accord~\cite{Binoth:2010xt,Alioli:2013nda}. This script expects at least the name of an order file.
This order file is usually but not necessarily created by a Monte Carlo program. An
example file for the partonic $2\to3$ processes of $pp\rightarrow t\bar{t}+\text{jets}$
is given below:
\lstset{literate={->}{{$\rightarrow$}}1}
\begin{lstlisting}[language=olp,numbers=left]
MatrixElementSquareType CHsummed
IRregularisation        tHV
OperationMode           CouplingsStrippedOff
SubdivideSubprocess     yes
AlphasPower             3
CorrectionType          QCD

# Here comes the list of subprocesses
# specified through PDG codes
# g     g -> t t-bar g
 21    21 -> 6 -6   21
# u u-bar -> t t-bar g
  2    -2 -> 6 -6   21
# u     g -> t t-bar u
  2    21 -> 6 -6    2
\end{lstlisting}
The line numbers are not part of the file.
The arrow `$\rightarrow$' is generated by the two characters `\verb|->|'.
The following options are part of the Standard and accepted by
\gosamv{}:
\lstset{language=olp}
\begin{description}
\item[\texttt{MatrixElementSquareType}] accepts the values
	\lstinline!Hsummed!, \lstinline!Csummed!,
	\lstinline!Haveraged!, \lstinline!Caveraged!,
	\lstinline!CHsummed!, \lstinline!CHaveraged!.

	The value \lstinline!NOTsummed! is not supported. 
	Sensible combinations are also allowed, as in
\begin{lstlisting}[language=olp]
MatrixElementSquareType Hsummed Caveraged
\end{lstlisting}

        In \gosamv{} this statement is optional.
	Any quantity which is not explicitely averaged is assumed to be
	summed
\item[\texttt{CorrectionType}] accepts the values
        \lstinline!QCD!, \lstinline!QED! and \lstinline!EW!, whereas
	\gosamv{} does not distinguish between the latter two (this behaviour
	might change in the future when appropriate model files are available).

	This statement is mandatory and must not be omitted.
\item[\texttt{IRregularisation}] accepts the values \lstinline!tHV!
        ('t~Hooft-Veltman scheme) and \lstinline!DRED! (dimensional reduction).
	The value \lstinline!CDR! (conventional dimensional regularisation)
	is not supported and therefore rejected.

%	\attention The support for \lstinline!DRED! is limited.
%        This featureis still considered experimental. 
%       It should only be used with great care.

	This statement is mandatory and must not be omitted.
\item[\texttt{MassiveParticleScheme}] accepts the value \lstinline!OnShell!
	only.
	At the moment this option has no effect on the generation of the matrix
	element. This statement is optional; if it appears in the order file
	a warning is issued, reminding the user that no UV-counterterms for
	massive particles are implemented yet.
\item[\texttt{IRsubtractionMethod}] accepts the value \lstinline!None! only.
	\gosamv{} does not provide any subtracted output.

	This statement is optional.
\item[\texttt{ModelFile}] accepts the name of parameter file in the
	Les Houches Accord format. The script reads the parameter file
	setting all masses to zero which are not specified explicitly
	to be non-zero.

	This statement is mandatory.

	It is recommended to use absolute paths here as the file will later
	be read in the function \texttt{OLP\_Start} in the matrix element
	code, which might be located elsewhere.
\item[\texttt{OperationMode}] accepts the value
	\lstinline!CouplingsStrippedOff! only.

	This statement is optional. If it is given, the coupling constants
	are stripped off from the amplitude.
\item[\texttt{SubDivideSubProcess}] accepts logical values
	(\lstinline!yes! or \lstinline!no!).

	If the value is \lstinline!yes! a separate channel for each
	helicity is assigned. Otherwise there will be one channel per
	subprocess.

	This statement is optional. Its default value is \lstinline!no!.
\item[\texttt{AlphasPower}] the power of $\alpha_s$ of the Born cross-section.
	At least one of the options \lstinline!AlphaPower! and
	\lstinline!AlphasPower! has to be specified.
\item[\texttt{AlphaPower}] the power of $\alpha$ of the Born cross-section.
	At least one of the options \lstinline!AlphaPower! and
	\lstinline!AlphasPower! has to be specified.
\end{description}

The options which have been proposed for electro-weak corrections
are currently not supported.

\subsection{Command Line Arguments of \texttt{gosam.py --olp}}
The syntax for the invocation of \texttt{gosam.py} is as follows:
\begin{example}
\$ gosam.py --olp \{$\langle$\textit{option}$\rangle$\}\\
    \contl $\langle$\textit{order file}$\rangle$
    \{$\langle$\textit{order file}$\rangle$\} \\
    \contl \{$\langle$\textit{key}$\rangle$=$\langle$\textit{value}$\rangle$\}
\end{example}
The allowed options are given below. The list of
$\langle$\textit{key}$\rangle$\texttt{=}$\langle$\textit{value}$\rangle$-pairs
supplements the options given in the configuration files.
\begin{description}
\item[\texttt{-h}, \texttt{--help}]
      Prints a help screen with all available command line options and exits.
\item[\texttt{-d}, \texttt{--debug}] 
      With this options the script will print lots of extra information to
      the screen, which is usually not useful for non-experts.
\item[\texttt{-v}, \texttt{--verbose}] 
      The script will print information e.g. about creating directories
      and reading files.
\item[\texttt{-w}, \texttt{--warn}]
      Warnings and errors are printed. This is the default setting.
\item[\texttt{-q}, \texttt{--quiet}]
      Only errors are printed, no warnings are issued.
\item[\texttt{-l}\textit{file}, \texttt{--log-file=}\textit{file}]
      All messages are written to a log file. When one or more log files
      are specified the information is still written to the screen with
      the latest specified level of detail. The following example will
      read the order file \texttt{test.olo}; messages at the debug
      level will be written to \texttt{detailed.log}, warnings and
      errors are written to \texttt{short.log} and only errors are printed
      to the screen.
      \begin{example}
\$ gosam.py --olp -d -ldetailed.log -w \\
	\contl{} -lshort.log -q test.olo
      \end{example}
\item[\texttt{-c}\textit{file}, \texttt{--config=}\textit{file}]
      Overlay default config files by the specified file.
      Usually, the script first searches in the default locations for
      configuration files. Afterwards, all files specified by \texttt{-c}
      options are read in the order in which they are encountered. Values
      which are already set by earlier files will be overwritten.
      See also option `\texttt{-C}'.
\item[\texttt{-C}, \texttt{--no-defaults}]
      The script will not search for configuration
      files (\texttt{.golem} and \texttt{golem.in}) in the standard locations
      (\gosamv{} installation directory, user's home directory and
      current working directory).
\item[\texttt{-f}, \texttt{--force}] Overwrite contract files without asking.
      The default behaviour is that contract files are not overwritten. If
      a contract file already exists the program gives an error message.
\item[\texttt{-e}, \texttt{--use-single-quotes}]
      Activates syntax extensions that allow the use of single quotes in
      order and contract files (See Section~\ref{sec:olp:extensions}).
\item[\texttt{-E}, \texttt{--use-double-quotes}]
      Activates syntax extensions that allow the use of double quotes in
      order and contract files (See Section~\ref{sec:olp:extensions}).
\item[\texttt{-b}, \texttt{--use-backslash}]
      Activates syntax extensions that allow the use of backslash escape
      sequences in
      order and contract files (See Section~\ref{sec:olp:extensions}).
\item[\texttt{-i}, \texttt{--ignore-case}] 
      Activates syntax estensions which make the parsing of order
      and contract files case-insensitive
      (See Section~\ref{sec:olp:extensions}).
\item[\texttt{-x}, \texttt{--ignore-unknown}]
      Unknown statements or values in order and contract files will be ignored.
      The default behaviour is that unknown statements and/or values will lead
      to an error message.
\item[\texttt{-o}\textit{file}, \texttt{--output-file=}\textit{file}]
      Specifies the name of the contract file(s). The following set of
      wildcard sequences can be used to derive the name of the contract file
      from the name of the order file. A value of
      `\texttt{-}' writes to the standard output.
      \begin{description}
      \item[\texttt{\%f}] The full file name (e.g. `\texttt{dir/process.olo}')
      \item[\texttt{\%F}] The file name without any leading path
         (`\texttt{process.olo}')
      \item[\texttt{\%p}] Path name only (`\texttt{dir/}')
      \item[\texttt{\%s}] The stem of the file name (`\texttt{process}')
      \item[\texttt{\%e}] The extension of the file name (`\texttt{.olo}')
      \end{description}
      By default this option is set to `\texttt{\%p\%s.olc}'.
\item[\texttt{-D}\textit{dir}, \texttt{--destination=}\textit{dir}]
      Chooses the output directory, to which each process is written.
      The same wildcards as above can be used. By default, all output is
      written to the current working directory. It is therefore not recommended
      to set this option using wildcards when more than one order file is
      specified.
\item[\texttt{-t}\textit{path}, \texttt{--templates=}\textit{path}]
      Sets an alternative templates directory or template
      XML-file.
\item[\texttt{-z}, \texttt{--scratch}]
      Overwrites all process files, including those which otherwise
      would be preserved (\texttt{Makefile.conf}, \texttt{config.f90} etc).
\end{description}

\subsection{\gosam{} Extensions to the Original Standard}
\label{sec:olp:extensions}
Modern file systems allow for path names which cannot be expressed in
the original formulation of the Les Houches accord. Therefore \gosamv{}
implements syntax extensions for order and contract files for including
special characters in statements, especially in file names (as in
\lstinline!ModelFile!).

\begin{description}
\item[double qoutes] This syntax extension proposes that inside a pair
   of double quotes (ASCII character~\#34) special characters lose their
   special meaning. The backslash acts as escape character, with the following
   set of escape sequences being allowed:
   \begin{itemize}
   \item[\texttt{\textbackslash t}] expands to a horizontal tabulator
        character (ASCII character~\#9),
   \item[\texttt{\textbackslash n}] expands to a new line character
   	(ASCII character~\#10),
   \item[\texttt{\textbackslash f}] expands to a form feed character
   	(ASCII character~\#12),
   \item[\texttt{\textbackslash r}] expands to a carriage return character
   	(ASCII character~\#13),
   \item[\texttt{\textbackslash x}\textit{hh}], where \textit{hh} are two
        hexadecimal digits expands to the character of which the
        ASCII code is the hexadecimal number represented by the
	digits~\textit{hh}.
   \item any other character following a backslash expands to itself,
        in particular \texttt{\textbackslash"}
	and~\texttt{\textbackslash\textbackslash}.
   \end{itemize}
\item[single quotes] This syntax extension proposes that inside a pair
   of single quotes (ASCII character~\#39) all characters lose their
   special meaning. There is no escape character. A literal single quote
   is generated by a sequence of two single quotes (Pascal like).
\item[backslash escapes] This syntax extension proposes that any character
   following a backslash loses its special meaning.
\end{description}

Different extensions might prove useful on different operating systems.
On a Windows system, the file name \verb!F:\gosam Files\mssm.slha! can
only be expressed with the proposed syntax extensions and would have the
following three equivalent representations:
\begin{itemize}
\item \verb!F:\\gosam\ Files\\mssm.slha!
\item \verb!'F:\gosam Files\mssm.slha'!
\item \verb!"F:\\gosam Files\\mssm.slha"!
\end{itemize}

The three extensions can be switched on by the command line options
of \texttt{gosam.py --olp},
`\texttt{-E}', `\texttt{-e}' and `\texttt{-b}' respectively.

\subsection{Advanced Usage}
The core functionality of the script \texttt{gosam.py --olp} is implemented
by the function \texttt{golem.util.olp.process\_order\_file}, which has the
the following signature:
\begin{example}
process\_order\_file(order\_file\_name, out\_file, out\_dir,\\
   \ \ conf,
   templates=None,
   ignore\_case=False,\\
   \ \ ignore\_unknown=False,
   single\_quotes=False,\\
   \ \ double\_quotes=False,
   backslash\_escape=False)
\end{example}
\begin{description}
\item[\texttt{order\_file\_name}] (character string) name of the order
   file.
\item[\texttt{out\_file}] (file object, open for writing)
   contract file.
\item[\texttt{out\_dir}] (character string) name of an existing directory
   to which all matrix-element files will be written.
\item[\texttt{conf}] (\texttt{golem.util.config.Properties})
   configuration shared by all subprocesses.
\item[\texttt{templates}] (character string) template directory or
   name of an XML-file.
\item[\dots] all other arguments activate the corresponding
   syntax extensions.
\end{description}
The return value is zero in case of a success and one if an error occurred.

A list of options read from default config files can be obtained by
the function
\texttt{golem.util.main\_misc.find\_config\_files()}. The following
example suggests the usage of the interface from a \python{}-based
Monte Carlo program
\begin{lstlisting}[language=python]
import os
import golem

# Monte Carlo program prepars the process
# and writes order file proc.olo ...
# (not shown in example)

conf = golem.util.main_misc.find_config_files()
f = open("proc.olc", 'w')
os.mkdir("proc/")

# Add own options
conf[golem.properties.model] = \
   "FeynRules, \${HOME}/models/mssm_ufo"
conf[golem.properties.fc_bin] = "gfortran"
err_flag = golem.util.olp.process_order_file(\
   "proc.olo", f, "proc/", conf)

if err_flag > 0:
	print "Problems generating OLP"
	print "Please consult the file proc.olc"
\end{lstlisting}


\section{Runtime Phase}

After the script \texttt{gosam.py --olp} or any equivalent program has been
run successfully, the files in the newly created process directories are
compiled by invoking \texttt{make} in the respective top-level directory.
This generates the object file \texttt{olp\_module.o} which contains all
API functions. The library for a given process can be linked using the
script \texttt{config.sh} in the same directory. The make-file of a client
program would typically contain code similar to the following:
\begin{lstlisting}[language=bash]
PROCESS_PATH=path/to/your/process-files
LDFLAGS+=$(shell sh $(PROCESS_PATH)/config.sh -libs)
\end{lstlisting}

\attention The module \texttt{olp\_module.f90} uses \fortranMMIII{}
extensions (\texttt{ISO\_C\_BINDING}) for establishing a well defined
interface for the linker. Older \fortranXC{} compilers might therefore
not be able to compile this module. Please refer to the compiler documentation
for details.

\subsection{API Functions}
\lstset{language=C}
The file \texttt{olp.h} contains the following prototypes.
\begin{lstlisting}[language=C]
void OLP_Start(char*,int*);
void OLP_EvalSubProcess(int,double*,
   double,double*,double*);
void OLP_Finalize();
void OLP_Option(char*,int*);
\end{lstlisting}
The first two functions are defined exactly as proposed
in~\cite{Binoth:2010xt}. The other two functions extend the original
standard. It should, however, be noted that the generated matrix element code
can be run without any calls to either \lstinline!OLP_Finalize!
or \lstinline!OLP_Option!.

\subsubsection*{\texttt{OLP\_Start}}
\begin{lstlisting}[language=C]
void OLP_Start(char* contract_file, int* success);
\end{lstlisting}
This function should be called before the first evaluation of the matrix
element. It ensures that all global variables in the matrix element code
are initialized properly. The argument \lstinline!contract_file!
should receive the (full) name of the contract file which was generated
together with the matrix element. The integer \lstinline!success! is
initialized by \lstinline!OLP_Start! to either the value one, indicating
success, or zero, indicating that an error occurred during initialization.

Matrix elements generated with \gosamv{} will try to read the SLHA model
file specified by the option \lstset{language=olp}\lstinline!ModelFile!
in the contract file. It is not required that the contract file used in
the runtime phase points to the same model file as used during the
initialisation phase. However, values which were set to zero during
initialisation will remain zero during the runtime phase.

\subsubsection*{\texttt{OLP\_EvalSubProcess}}
\lstset{language=C}
\begin{lstlisting}[language=C]
void OLP_EvalSubProcess(int label, double* momenta,
   double scale, double* parameter, double* amp);
\end{lstlisting}
This function retrieves the values for a channel of the OLP for a given
phase space point. A channel might be a subprocess or a
gauge invariant partial amplitude, depending on the settings in the
contract file. The channel is labeled by the argument \lstinline!label!.
The second argument is a one-dimensional array holding the $5\times N$
components of the momenta for an $N$-particle process. They are in the order
\begin{displaymath}
(E^{(1)}, p_x^{(1)}, p_y^{(1)}, p_z^{(1)}, m^{(1)},
 E^{(2)}, p_x^{(2)}, p_y^{(2)}, p_z^{(2)}, m^{(2)},
 \ldots, m^{(N)})
\end{displaymath}
The third argument is the renormalization scale (not its square).
A list of scale dependent parameters is passed in the fourth argument.
Its first entry is expected to be $\alpha_s(\mu)$. Any further entries
are user-defined; the user is expected to adapt the subroutine
\texttt{init\_event\_parameters} in \texttt{olp\_module.f90}
if he wishes to make use of any additional parameters.

The last argument is an array of length four. Its entries are, in this order,
\begin{enumerate}
\item the coefficient of the $1/\varepsilon^2$ pole in the Laurent series
   of the interference term between virtual and Born amplitude,
\item the coefficient of the $1/\varepsilon$ pole in the Laurent series
   of the interference term between virtual and Born amplitude,
\item the $\mathcal{O}(1)$ term in the Laurent series
   of the interference term between virtual and Born amplitude,
\item the square of the Born amplitude.
\end{enumerate}

Matrix elements generated by \gosamv{} use the convention that in case
of an error during the evaluation of the matrix element, the fourth
entry is set to $(-1)$. It is therefore recommended that client programs
check for the positiveness of the Born matrix element.

\subsubsection{\texttt{OLP\_Finalize}}
\begin{lstlisting}[language=C]
void OLP_Finalize();
\end{lstlisting}

This function should be called after the last evaluation of the matrix
element. It allows the OLP to close any open file handles, to release
allocated memory and to exit gracefully.
Although on most modern operating systems this is done automatically,
it is good practice and therefore recommended to always call this function
before exiting the program.

\subsubsection{\texttt{OLP\_Option}}
\begin{lstlisting}[language=C]
void OLP_Option(char* assignment, int* success);
\end{lstlisting}

This function can be used to update internal parameters of the OLP which
are not part of the standard. The first argument is a character string
containing a textual representation of the requested assignment. The
second argument will be set by the function according to the success
of the request.

Matrix elements generated with \gosamv{} accept any string which would
also be valid as a (non-comment) line in a parameter
file (see~\texttt{model.f90}). Typical calls would be
\begin{lstlisting}[language=C]
OLP_Option("samurai_test=3", &flag);
/* The previous call requires 
 * reinitialization of the OLP */
OLP_Start(contract_file, &flag);
OLP_Option("Nf=5", &flag);
/* Setting the Higgs mass: */
OLP_Option("mH=124.5", &flag);
\end{lstlisting}
