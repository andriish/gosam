\documentclass[11pt,a4paper]{refrep}
\usepackage{xspace}
\usepackage[final]{remark}
\usepackage{listings}
\usepackage{amsmath,amssymb}
\usepackage{syntax}
\usepackage[pdftex]{hyperref}

\input lstform
\input lstolp
\newcommand{\golemversion}{{1{.}0}}
\newcommand{\golem}{{\tt GoSam}\xspace}
%\newcommand{\golemv}[1][\golemversion]{{{\tt Golem}~\golemversion}\xspace}
\newcommand{\golemv}[1][\golemversion]{{\tt GoSam}\xspace}
\newcommand{\golemVC}{{\tt golem95}\xspace}
\newcommand{\packagename}{{gosam-\golemversion}}
\newcommand{\hepforge}{{\sc HepForge}\xspace}

\newcommand{\qgraf}{{\tt QGraf}\xspace}
\newcommand{\form}{{\tt Form}\xspace}
\newcommand{\python}{{\tt Python}\xspace}
\newcommand{\fortranXC}{{\tt Fortran\,95}\xspace}
\newcommand{\fortranMMIII}{{\tt Fortran\,2003}\xspace}
\newcommand{\pjfry}{{\tt PJFry}\xspace}
\newcommand{\haggies}{{\tt haggies}\xspace}
\newcommand{\samurai}{{\sc Samurai}\xspace}

\newcommand{\contl}{{\ensuremath{\hookrightarrow}}}
\newcommand{\fmslash}[1]{{#1}\!\!\!/}
\newcommand{\pslash}[1][{}]{\ensuremath{\fmslash{p}_{#1}}}
\newcommand{\kslash}[1][{}]{\ensuremath{\fmslash{k}_{#1}}}

\title{\golemv Developer's Manual}
\author{Thomas Reiter}
\date{Version \today}

\begin{document}
\hypersetup{%
	pdfborder={0 0 0},%
	pdftitle={GoSam \golemversion{} Developer's Manual},%
	pdfauthor={Thomas Reiter},%
	pdfsubject={High Energy Physics/Higher Order Corrections},%
	pdfkeywords={NLO, automatization},%
	pdfdisplaydoctitle
}

\begin{fullpage}
\maketitle
\tableofcontents
\end{fullpage}

\chapter{Introduction}
\section{Overview}
The \golem{} source code can be divided in three main parts:
\begin{itemize}
\item A \python{} program controlling the first step of the code generation,
\item \form{} code, most of which is part of the templates
\item \fortranXC{} code, which is completely generated by templates.
\end{itemize}
Moreover there are few percent of the code not covered by the above list:
\begin{itemize}
\item The file \texttt{process.hh} is produced directly by the \python{}
   code, there is no template for it.
\item \haggies{} is written in Java. It is only included as a JAR file
   in the \texttt{haggies} subdirectory. The source code is not part of
   \golem{}.
\item The hand-written model files in the directory \texttt{models}
   contain \qgraf{} model files.
\item The subdirectory \texttt{olp/templates} contains files written
   in \texttt{C}, \texttt{Lex} and \texttt{Yacc}.
\item The \form{} files for the Lorentz representations (propagators and
   wave functions) are written in the Literate Programming system
   \texttt{nuweb}.
\end{itemize}

Most of the time a developer will only need to modify the files in the
template directory. Therefore this part of the code is discussed first.

\chapter{The Template System}
\section{Introduction}
\golem{} introduces a template processor (sophisticated preprocessor)
to create another level of abstraction between the \python{} code and
the generated code. The advantages of such a system are:
\begin{itemize}
\item There is no need to know \python{} in order to apply simple
   modifications and extensions to the program.
\item The template language is side effect free reducing the risk
   of hidden bugs.
\item The templates allow one to think in terms of the target language
   and reduce the need of escaping syntactical entities.
\end{itemize}
Unfortunately, there is no light without shadows:
\begin{itemize}
\item The developer needs to learn another language, i.e. the
   template markup language.
\item The simplicity of the markup language sometimes requires
   to express things in a complicated way.
\end{itemize}

The template processor is written in \python{} and is controlled
by a XML file, by default this is the file \texttt{templates/template.xml}
in the \golem{} installation directory.
\attention The variable \texttt{templates} in the process input file
can either be set to a XML file or to a directory which contains
a file called \marginlabel{\texttt{template.xml}}\texttt{template.xml}.
If the variable is set, that
file is used instead of \golem{}'s \texttt{templates/template.xml}.
This feature could be used to provide template directories for specific
purposes.

\section{Template Markup Syntax}
\subsection{Simple Tags}
Template files are a mixture of text which is copied to the output file in
verbatim and tags enclosed in \lit{[\%} and \lit{\%]}. In the simplest
case, a tag contains only a variable name, e.g.
\begin{lstlisting}
   real(ki), dimension([%num_legs%]) :: vecs
\end{lstlisting}
In the output, the sequence \lit{[\%num\_legs\%]} is replaced by the
actual number of external particles in the process.

\appendix
\chapter{Available Template Classes}
\section{Predefined Symbols}
All symbols which are defined in the list of properties
in \texttt{golem.properties} are read from the process card and
made available in any template. Furthermore, the following symbols
are set up while the input file is processed.
\begin{description}
\item[\texttt{topolopy.keep.tree}] comma seperated list of diagram labels
(tree diagrams) which passed the filters.
\item[\texttt{topolopy.keep.virt}] comma seperated list of diagram labels
(loop diagrams) which passed the filters.
\item[\texttt{topolopy.count.tree}] number of elements in
\texttt{topolopy.keep.tree}, i.e. the number of tree level diagrams in this
process.
\item[\texttt{topolopy.count.virt}] number of elements in
\texttt{topolopy.keep.tree}, i.e. the number of loop diagrams in this
process.
\item[\texttt{generate\_lo\_diagrams}] whether or not to generate tree
diagrams (depends on the value of \texttt{order}).
\item[\texttt{generate\_nlo\_virt}] whether or not to generate loop
diagrams (depends on the value of \texttt{order}).
\item[\texttt{generate\_uv\_counterterms}] reserved for future use,
currently always set to \texttt{false}.
\end{description}

\section{The Base Class}
\section{\texttt{Kinematics}}
\section{\texttt{Integrals}}
\section{\texttt{Model}}
\section{\texttt{Multi}}
\section{\texttt{LightCone}}
\section{\texttt{Verbatim}}
\section{\texttt{OLP}}
\end{document}
