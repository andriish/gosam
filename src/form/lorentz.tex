\newcommand{\NWtarget}[2]{#2}
\newcommand{\NWlink}[2]{#2}
\newcommand{\NWtxtMacroDefBy}{Fragment defined by}
\newcommand{\NWtxtMacroRefIn}{Fragment referenced in}
\newcommand{\NWtxtMacroNoRef}{Fragment never referenced}
\newcommand{\NWtxtDefBy}{Defined by}
\newcommand{\NWtxtRefIn}{Referenced in}
\newcommand{\NWtxtNoRef}{Not referenced}
\newcommand{\NWtxtFileDefBy}{File defined by}
\newcommand{\NWtxtIdentsUsed}{Uses:}
\newcommand{\NWtxtIdentsNotUsed}{Never used}
\newcommand{\NWtxtIdentsDefed}{Defines:}
\newcommand{\NWsep}{${\diamond}$}
\newcommand{\NWnotglobal}{(not defined globally)}
\newcommand{\NWuseHyperlinks}{}
\documentclass[a4paper,12pt]{amsart}

\usepackage[pdftex]{hyperref}
\usepackage{dsfont}
\usepackage{units}
\usepackage[draft]{remark}

\DeclareMathOperator{\Trace}{tr}
\newcommand{\fmslash}[1]{\ensuremath{/\!\!\!{#1}}}
\newcommand{\pslash}[1][{}]{\fmslash{p}_{#1}}
\newcommand{\qslash}[1][{}]{\fmslash{q}_{#1}}
\newcommand{\kslash}[1][{}]{\fmslash{k}_{#1}}
\newcommand{\tr}[2][{}]{\Trace^{#1}\!\left\{{#2}\right\}}
\newcommand{\form}{{\texttt{Form}}}
\newcommand{\golem}{{\texttt{GOLEM}}}
\newcommand{\bra}[1]{\langle #1 \vert}
\newcommand{\brb}[1]{[ #1 \vert}
\newcommand{\kea}[1]{\vert #1 \rangle}
\newcommand{\keb}[1]{\vert #1 ]}
\newcommand{\Spaa}[1]{\langle #1 \rangle}
\newcommand{\Spab}[1]{\langle #1]}
\newcommand{\Spba}[1]{[ #1 \rangle}
\newcommand{\Spbb}[1]{[ #1 ]}

\renewcommand{\NWtarget}[2]{\hypertarget{#1}{#2}}
\renewcommand{\NWlink}[2]{\hyperlink{#1}{#2}}
\renewcommand{\NWtxtMacroDefBy}{Macro defined by}
\renewcommand{\NWtxtMacroRefIn}{Macro referenced in}
\renewcommand{\NWtxtMacroNoRef}{Macro never referenced}
\renewcommand{\NWtxtDefBy}{Defined by}
\renewcommand{\NWtxtRefIn}{Referenced in}
\renewcommand{\NWtxtNoRef}{Not referenced}
\renewcommand{\NWtxtFileDefBy}{File defined by}
\renewcommand{\NWsep}{${\diamond}$}

\newcommand{\doc}[1]{\paragraph{\hspace{-3em}\sc #1}}

\date{Last updated: \today}
\author{The GoSam Collaboration\\Thomas Reiter}
\title{Representations for Particles up to Spin~$3/2$}
\email{\href{mailto:thomasr@nikhef.nl}{\nolinkurl{thomasr@nikhef.nl}}}
\urladdr{\url{http://www.nikhef.nl/~thomasr}}

\makeatletter
\hypersetup{
	pdfauthor={Thomas Reiter / The GoSam Collaboration},
	pdftitle={\@title},
	pdfsubject={High Energy Physics, Computer Algebra},
	pdfkeywords={Representation, Lorentz Group},
	colorlinks=true,
	pdfborder={0 0 0}
}
\makeatother

\begin{document}
\begin{abstract}
This document describes the implementation of wavefunctions and
propagators in~\golem.
\end{abstract}
\maketitle
\tableofcontents
\section{Global Structure}
%---#[ Global Structure :
The replacements for the wave-functions go into the file \texttt{legs.hh},
propagators are found in \texttt{propagators.hh}. The diagram generator is
expected to yield the following functions:
\begin{description}
\item[\texttt{inplorentz($2s$, $i$, $k$, $m$)}] for each initial state
particle of spin $s$, momentum $k$ and mass $m$. The index $i$ is a
Lorentz index in the corresponding representation which connects the
wave function to the rest of the diagram. In cases where particle and
antiparticle are distinct, the parameter $2s$ is signed ($-2s$ for the
antiparticle).
\item[\texttt{outlorentz($2s$, $i$, $k$, $m$)}] as above,
but for each final state particle.
\item[\texttt{proplorentz($2s$, $k$, $m$, $\Gamma$, $A$, $i_1$, $i_2$)}]
denotes the
Lorentz part of a propagator for a particle of spin $s$, momentum $k$ and
mass $m$. The indices $i_1$ and $i_2$ are corresponding Lorentz indices.
The decay width of the particle is $\Gamma$. The parameter $A$ is a flag
that indicates special properties of a field and is non-zero if the
propagator needs special treatment.
\item[\texttt{inpcolor($n$, $i$)}] for each initial state particle.
Associates the colour index $i$ with the initial state particle number $n$.
This function is not treated in this file.
\item[\texttt{outcolor($n$, $i$)}] for each final state particle.
Associates the colour index $i$ with the finaltial state particle number $n$.
This function is not treated in this file.
\item[\texttt{propcolor($r$, $i_1$, $i_2$)}] denotes the colour part of
a propagator, where $r$ is a representation label; $r$ is
either 1 (trivial rep.), -3 or 3 (fundamental rep.) or 8 (adjoint rep.).
The indices $i_1$ and $i_2$ are the colour indices of that propagator.
\item[\texttt{inp($f$, $k$, [$h$], [$k^\flat$], [$q$])}]
carries the helicity information
$h$ of an initial state particle of the field $f$ and momentum $k$.
For massive gauge
bosons, the parameters $k^\flat$ and $q$ are the two momenta of the
light-cone splitting. For massless gauge the parameter $k^\flat$ is
ommitted.
The parameters $h$, $k^\flat$ and $q$ are not generated
by the diagram generator but added at an earlier point
in the \form\ program.
\item[\texttt{out($f$, $k$, [$h$], [$k^\flat$], [$q$])}]
same as \texttt{inp} but for final state particles.
\end{description}

On the output side we use the symbols introduced by the \texttt{spinney}
library plus the scalar propagator
\begin{equation}
\mathtt{inv}(k, m) = \frac{1}{k^2-m^2+i0^+}\quad\text{and}\quad
\mathtt{inv}(k, m,\Gamma) = \frac{1}{k^2-m^2-im\Gamma+i0^+}
\end{equation}
\begin{flushleft} \small
\begin{minipage}{\linewidth}\label{scrap1}\raggedright\small
\NWtarget{nuweb2a}{} $\langle\,${\itshape common header}\nobreak\ {\footnotesize {2a}}$\,\rangle\equiv$
\vspace{-1ex}
\begin{list}{}{} \item
\mbox{}\verb@* vim: ts=3:sw=3@\\
\mbox{}\verb@* This file is generated from lorentz.nw.@\\
\mbox{}\verb@* Do not edit this file directly.@\\
\mbox{}\verb@* Instead change src/form/lorentz.nw and run make.@\\
\mbox{}\verb@@{\NWsep}
\end{list}
\vspace{-1.5ex}
\footnotesize
\begin{list}{}{\setlength{\itemsep}{-\parsep}\setlength{\itemindent}{-\leftmargin}}
\item \NWtxtMacroRefIn\ \NWlink{nuweb2b}{2b}\NWlink{nuweb3}{, 3}.

\item{}
\end{list}
\end{minipage}\vspace{4ex}
\end{flushleft}
\begin{flushleft} \small\label{scrap2}\raggedright\small
\NWtarget{nuweb2b}{} \verb@"legs.hh"@\nobreak\ {\footnotesize {2b}}$\equiv$
\vspace{-1ex}
\begin{list}{}{} \item
\mbox{}\verb@@\hbox{$\langle\,${\itshape common header}\nobreak\ {\footnotesize \NWlink{nuweb2a}{2a}}$\,\rangle$}\verb@@\\
\mbox{}\verb@*---#[ Scalars :@\\
\mbox{}\verb@@\hbox{$\langle\,${\itshape scalar wave-functions}\nobreak\ {\footnotesize \NWlink{nuweb4a}{4a}}$\,\rangle$}\verb@@\\
\mbox{}\verb@*---#] Scalars :@\\
\mbox{}\verb@*---#[ Spinors :@\\
\mbox{}\verb@*---#[   Massless Spinors :@\\
\mbox{}\verb@@\hbox{$\langle\,${\itshape wave-functions for massless spinors}\nobreak\ {\footnotesize \NWlink{nuweb5c}{5c}}$\,\rangle$}\verb@@\\
\mbox{}\verb@*---#]   Massless Spinors :@\\
\mbox{}\verb@*---#[   Massive Spinors :@\\
\mbox{}\verb@@\hbox{$\langle\,${\itshape wave-functions for massive spinors}\nobreak\ {\footnotesize \NWlink{nuweb7a}{7a}}$\,\rangle$}\verb@@\\
\mbox{}\verb@*---#]   Massive Spinors :@\\
\mbox{}\verb@*---#] Spinors :@\\
\mbox{}\verb@*---#[ Polarisation Vectors for Gauge Bosons :@\\
\mbox{}\verb@*---#[    Massless Gauge Bosons :@\\
\mbox{}\verb@@\hbox{$\langle\,${\itshape gauge boson wave-functions, light-like}\nobreak\ {\footnotesize \NWlink{nuweb9b}{9b}}$\,\rangle$}\verb@@\\
\mbox{}\verb@*---#]    Massless Gauge Bosons :@\\
\mbox{}\verb@*---#[    Massive Gauge Bosons :@\\
\mbox{}\verb@@\hbox{$\langle\,${\itshape gauge boson wave-functions, massive}\nobreak\ {\footnotesize \NWlink{nuweb11d}{11d}}$\,\rangle$}\verb@@\\
\mbox{}\verb@*---#]    Massive Gauge Bosons :@\\
\mbox{}\verb@*---#] Polarisation Vectors for Gauge Bosons :@\\
\mbox{}\verb@*---#[ wave functions for Vector-Spinors :@\\
\mbox{}\verb@@\hbox{\sffamily\bfseries Repeat}\verb@;@\\
\mbox{}\verb@   @\hbox{$\langle\,${\itshape vector-spinor wave functions}\nobreak\ {\footnotesize \NWlink{nuweb13}{13}}$\,\rangle$}\verb@@\\
\mbox{}\verb@@\hbox{\sffamily\bfseries EndRepeat}\verb@;@\\
\mbox{}\verb@*---#] wave functions for Vector-Spinors :@\\
\mbox{}\verb@*---#[ wave functions for gravitons :@\\
\mbox{}\verb@@\hbox{\sffamily\bfseries Repeat}\verb@;@\\
\mbox{}\verb@   @\hbox{$\langle\,${\itshape graviton wave functions}\nobreak\ {\footnotesize \NWlink{nuweb19b}{19b}}$\,\rangle$}\verb@@\\
\mbox{}\verb@@\hbox{\sffamily\bfseries EndRepeat}\verb@;@\\
\mbox{}\verb@*---#] wave functions for gravitons :@\\
\mbox{}\verb@@{\NWsep}
\end{list}
\vspace{-1.5ex}
\footnotesize
\begin{list}{}{\setlength{\itemsep}{-\parsep}\setlength{\itemindent}{-\leftmargin}}

\item{}
\end{list}
\vspace{4ex}
\end{flushleft}
\begin{flushleft} \small\label{scrap3}\raggedright\small
\NWtarget{nuweb3}{} \verb@"propagators.hh"@\nobreak\ {\footnotesize {3}}$\equiv$
\vspace{-1ex}
\begin{list}{}{} \item
\mbox{}\verb@@\hbox{$\langle\,${\itshape common header}\nobreak\ {\footnotesize \NWlink{nuweb2a}{2a}}$\,\rangle$}\verb@@\\
\mbox{}\verb@@\hbox{$\langle\,${\itshape colour part of the propagators}\nobreak\ {\footnotesize \NWlink{nuweb21}{21}}$\,\rangle$}\verb@@\\
\mbox{}\verb@*---#[ Scalar Bosons :@\\
\mbox{}\verb@@\hbox{$\langle\,${\itshape scalar propagator}\nobreak\ {\footnotesize \NWlink{nuweb4b}{4b}}$\,\rangle$}\verb@@\\
\mbox{}\verb@*---#] Scalar Bosons :@\\
\mbox{}\verb@*---#[ Fermions :@\\
\mbox{}\verb@@\hbox{$\langle\,${\itshape fermion propagator}\nobreak\ {\footnotesize \NWlink{nuweb5a}{5a}}$\,\rangle$}\verb@@\\
\mbox{}\verb@@\hbox{$\langle\,${\itshape handed fermion propagator}\nobreak\ {\footnotesize \NWlink{nuweb5b}{5b}}$\,\rangle$}\verb@@\\
\mbox{}\verb@*---#] Fermions :@\\
\mbox{}\verb@*---#[ Gauge Bosons :@\\
\mbox{}\verb@@\hbox{$\langle\,${\itshape gauge boson propagator}\nobreak\ {\footnotesize \NWlink{nuweb8a}{8a}}$\,\rangle$}\verb@@\\
\mbox{}\verb@*---#] Gauge Bosons :@\\
\mbox{}\verb@*---#[ Vector-Spinor propagator :@\\
\mbox{}\verb@@\hbox{$\langle\,${\itshape vector-spinor propagators}\nobreak\ {\footnotesize \NWlink{nuweb12}{12}}$\,\rangle$}\verb@@\\
\mbox{}\verb@*---#] Vector-Spinor propagator :@\\
\mbox{}\verb@*---#[ Tensor Bosons :@\\
\mbox{}\verb@@\hbox{\sffamily\bfseries Repeat}\verb@;@\\
\mbox{}\verb@   @\hbox{$\langle\,${\itshape tensor ghost propagator}\nobreak\ {\footnotesize \NWlink{nuweb18a}{18a}}$\,\rangle$}\verb@@\\
\mbox{}\verb@   @\hbox{$\langle\,${\itshape graviton propagator}\nobreak\ {\footnotesize \NWlink{nuweb18b}{18b}}$\,\rangle$}\verb@@\\
\mbox{}\verb@   @\hbox{$\langle\,${\itshape graviton effective propagator}\nobreak\ {\footnotesize \NWlink{nuweb19a}{19a}}$\,\rangle$}\verb@@\\
\mbox{}\verb@@\hbox{\sffamily\bfseries EndRepeat}\verb@;@\\
\mbox{}\verb@*---#] Tensor Bosons :@\\
\mbox{}\verb@@{\NWsep}
\end{list}
\vspace{-1.5ex}
\footnotesize
\begin{list}{}{\setlength{\itemsep}{-\parsep}\setlength{\itemindent}{-\leftmargin}}

\item{}
\end{list}
\vspace{4ex}
\end{flushleft}
For the Feynman rules we stick to the conventions of~\cite{1}.
%---#] Global Structure :
\section{Spin-0 Particles}
%---#[ Spin-0 Particles :
The wave function of a spin-0 particle is represented by a pure number.
\begin{flushleft} \small
\begin{minipage}{\linewidth}\label{scrap4}\raggedright\small
\NWtarget{nuweb4a}{} $\langle\,${\itshape scalar wave-functions}\nobreak\ {\footnotesize {4a}}$\,\rangle\equiv$
\vspace{-1ex}
\begin{list}{}{} \item
\mbox{}\verb@@\hbox{\sffamily\bfseries Id}\verb@ inplorentz(0, iv?, k1?, m?) = 1;@\\
\mbox{}\verb@@\hbox{\sffamily\bfseries Id}\verb@ outlorentz(0, iv?, k1?, m?) = 1;@{\NWsep}
\end{list}
\vspace{-1.5ex}
\footnotesize
\begin{list}{}{\setlength{\itemsep}{-\parsep}\setlength{\itemindent}{-\leftmargin}}
\item \NWtxtMacroRefIn\ \NWlink{nuweb2b}{2b}.

\item{}
\end{list}
\end{minipage}\vspace{4ex}
\end{flushleft}
Its propagator is just the plain propagator
\begin{equation}
\frac{i}{k^2-m^2-im\Gamma+i0^+}\text{.}
\end{equation}
\begin{flushleft} \small
\begin{minipage}{\linewidth}\label{scrap5}\raggedright\small
\NWtarget{nuweb4b}{} $\langle\,${\itshape scalar propagator}\nobreak\ {\footnotesize {4b}}$\,\rangle\equiv$
\vspace{-1ex}
\begin{list}{}{} \item
\mbox{}\verb@@\hbox{\sffamily\bfseries Id}\verb@ proplorentz(0, k1?, m?, sDUMMY1?, 0, iv1?, iv2?) =@\\
\mbox{}\verb@   PREFACTOR(@\hbox{\sffamily\bfseries i}\verb@_) * inv(k1, m, sDUMMY1);@\\
\mbox{}\verb@@\hbox{\sffamily\bfseries Id}\verb@ proplorentz(0, 0, m?, sDUMMY1?, 0,iv1?, iv2?) =@\\
\mbox{}\verb@   PREFACTOR(@\hbox{\sffamily\bfseries i}\verb@_) * inv(ZERO, m, sDUMMY1);@{\NWsep}
\end{list}
\vspace{-1.5ex}
\footnotesize
\begin{list}{}{\setlength{\itemsep}{-\parsep}\setlength{\itemindent}{-\leftmargin}}
\item \NWtxtMacroRefIn\ \NWlink{nuweb3}{3}.

\item{}
\end{list}
\end{minipage}\vspace{4ex}
\end{flushleft}
%---#] Spin-0 Particles :

\section{Spin-\texorpdfstring{\nicefrac{1}{2}}{1/2} Particles}
%---#[ Spin-1/2 Particles :
For spinor wave functions we have the following assignment in the
notation of~\cite{1}:
\begin{center}
\begin{tabular}{l|cc}
     &     $l^-$, $q$ & $l^+$, $\bar{q}$\\
\hline
initial & $u_\alpha(k, j_3)$ & $\bar{v}_\alpha(k, j_3)$ \\
final & $\bar{u}_\alpha(k, j_3)$ & $v_\alpha(k, j_3)$
\end{tabular}
\end{center}
Here, $l$ and $q$ stand for leptons and quarks respectively.
The index $\alpha$ denotes a spinor index and $j_3$ is the 3-component
of the spin.
We label the states by $j_3=\pm1$ instead of the physical values
$j_3=\pm1/2$.

The propagator both for the massive and the massless case is
\begin{equation}
\frac{i(\kslash + m)_{\alpha\beta}}{k^2-m^2-im\Gamma+i0^+}
\end{equation}
where the momentum flow is from $\beta$ to $\alpha$.
\begin{flushleft} \small
\begin{minipage}{\linewidth}\label{scrap6}\raggedright\small
\NWtarget{nuweb5a}{} $\langle\,${\itshape fermion propagator}\nobreak\ {\footnotesize {5a}}$\,\rangle\equiv$
\vspace{-1ex}
\begin{list}{}{} \item
\mbox{}\verb@@\hbox{\sffamily\bfseries Id}\verb@ proplorentz(1, k1?, m?, sDUMMY1?, 0, iv1?, iv2?) =@\\
\mbox{}\verb@  PREFACTOR(@\hbox{\sffamily\bfseries i}\verb@_) * (NCContainer(Sm(k1), iv2, iv1)@\\
\mbox{}\verb@   + csqrt(m*(m-@\hbox{\sffamily\bfseries i}\verb@_*sDUMMY1)) * NCContainer(1, iv2, iv1)@\\
\mbox{}\verb@  ) * inv(k1, m, sDUMMY1);@\\
\mbox{}\verb@@\hbox{\sffamily\bfseries Id}\verb@ proplorentz(1, 0, m?, sDUMMY1?, 0, iv1?, iv2?) =@\\
\mbox{}\verb@   + PREFACTOR(@\hbox{\sffamily\bfseries i}\verb@_ * csqrt(m*(m-@\hbox{\sffamily\bfseries i}\verb@_*sDUMMY1))) * NCContainer(1, iv2, iv1) * inv(ZERO, m, sDUMMY1);@{\NWsep}
\end{list}
\vspace{-1.5ex}
\footnotesize
\begin{list}{}{\setlength{\itemsep}{-\parsep}\setlength{\itemindent}{-\leftmargin}}
\item \NWtxtMacroRefIn\ \NWlink{nuweb3}{3}.

\item{}
\end{list}
\end{minipage}\vspace{4ex}
\end{flushleft}
For massless fermions,
the auxilliary field can also have values $1$ and $-1$ for
left- and right-handed particles. This follows the CalcHEP convention.
\begin{center}
\begin{tabular}{rlc}
Golem & CalcHEP & Expression\\
\hline
$+1$ & `L' & $\frac{\pslash\Pi_+}{p^2}$ \\
$-1$ & `R' & $\frac{\pslash\Pi_-}{p^2}$
\end{tabular}
\end{center}

\begin{flushleft} \small
\begin{minipage}{\linewidth}\label{scrap7}\raggedright\small
\NWtarget{nuweb5b}{} $\langle\,${\itshape handed fermion propagator}\nobreak\ {\footnotesize {5b}}$\,\rangle\equiv$
\vspace{-1ex}
\begin{list}{}{} \item
\mbox{}\verb@@\hbox{\sffamily\bfseries Id}\verb@ proplorentz(1, k1?, 0, 0, 1, iv1?, iv2?) =@\\
\mbox{}\verb@  PREFACTOR(@\hbox{\sffamily\bfseries i}\verb@_) * NCContainer(Sm(k1)*ProjPlus, iv2, iv1) * inv(k1, 0);@\\
\mbox{}\verb@@\hbox{\sffamily\bfseries Id}\verb@ proplorentz(1, k1?, 0, 0, -1, iv1?, iv2?) =@\\
\mbox{}\verb@  PREFACTOR(@\hbox{\sffamily\bfseries i}\verb@_) * NCContainer(Sm(k1)*ProjMinus, iv2, iv1) * inv(k1, 0);@{\NWsep}
\end{list}
\vspace{-1.5ex}
\footnotesize
\begin{list}{}{\setlength{\itemsep}{-\parsep}\setlength{\itemindent}{-\leftmargin}}
\item \NWtxtMacroRefIn\ \NWlink{nuweb3}{3}.

\item{}
\end{list}
\end{minipage}\vspace{4ex}
\end{flushleft}
\subsection{Massless Case}
%---#[   Massless Spin-1/2 Particles :
For massless spinors we translate the spin states directly into
helicity eigenstates as follows\footnote{Please, refer to the
\texttt{spinney} documentation for notational conventions of
bra- and ket-spinors.}:
\begin{subequations}
\begin{align}
u_\alpha(k, +1) &= \kea{k} & \bar{u}_\alpha(k, +1) &= \brb{k}
\label{eq:m0spinors:1}\\
u_\alpha(k, -1) &= \keb{k} & \bar{u}_\alpha(k, -1) &= \bra{k}
\label{eq:m0spinors:2}\\
v_\alpha(k, +1) &= \keb{k} & \bar{v}_\alpha(k, +1) &= \bra{k}
\label{eq:m0spinors:3}\\
v_\alpha(k, -1) &= \kea{k} & \bar{v}_\alpha(k, -1) &= \brb{k}
\label{eq:m0spinors:4}
\end{align}
\end{subequations}
\begin{flushleft} \small
\begin{minipage}{\linewidth}\label{scrap8}\raggedright\small
\NWtarget{nuweb5c}{} $\langle\,${\itshape wave-functions for massless spinors}\nobreak\ {\footnotesize {5c}}$\,\rangle\equiv$
\vspace{-1ex}
\begin{list}{}{} \item
\mbox{}\verb@@\hbox{$\langle\,${\itshape implementation of Equation~\eqref{eq:m0spinors:1}}\nobreak\ {\footnotesize \NWlink{nuweb6a}{6a}}$\,\rangle$}\verb@@\\
\mbox{}\verb@@\hbox{$\langle\,${\itshape implementation of Equation~\eqref{eq:m0spinors:2}}\nobreak\ {\footnotesize \NWlink{nuweb6b}{6b}}$\,\rangle$}\verb@@\\
\mbox{}\verb@@\hbox{$\langle\,${\itshape implementation of Equation~\eqref{eq:m0spinors:3}}\nobreak\ {\footnotesize \NWlink{nuweb6c}{6c}}$\,\rangle$}\verb@@\\
\mbox{}\verb@@\hbox{$\langle\,${\itshape implementation of Equation~\eqref{eq:m0spinors:4}}\nobreak\ {\footnotesize \NWlink{nuweb6d}{6d}}$\,\rangle$}\verb@@{\NWsep}
\end{list}
\vspace{-1.5ex}
\footnotesize
\begin{list}{}{\setlength{\itemsep}{-\parsep}\setlength{\itemindent}{-\leftmargin}}
\item \NWtxtMacroRefIn\ \NWlink{nuweb2b}{2b}.

\item{}
\end{list}
\end{minipage}\vspace{4ex}
\end{flushleft}
\begin{flushleft} \small
\begin{minipage}{\linewidth}\label{scrap9}\raggedright\small
\NWtarget{nuweb6a}{} $\langle\,${\itshape implementation of Equation~\eqref{eq:m0spinors:1}}\nobreak\ {\footnotesize {6a}}$\,\rangle\equiv$
\vspace{-1ex}
\begin{list}{}{} \item
\mbox{}\verb@@\hbox{\sffamily\bfseries Id}\verb@ inplorentz( 1, iv?, k1?, 0) *@\\
\mbox{}\verb@      inp(field1?, k1?,  1) =@\\
\mbox{}\verb@   NCContainer(USpa(k1), iv);@\\
\mbox{}\verb@@\hbox{\sffamily\bfseries Id}\verb@ outlorentz( 1, iv?, k1?, 0) *@\\
\mbox{}\verb@      out(field1?, k1?,  1) =@\\
\mbox{}\verb@   NCContainer(UbarSpb(k1), iv);@{\NWsep}
\end{list}
\vspace{-1.5ex}
\footnotesize
\begin{list}{}{\setlength{\itemsep}{-\parsep}\setlength{\itemindent}{-\leftmargin}}
\item \NWtxtMacroRefIn\ \NWlink{nuweb5c}{5c}.

\item{}
\end{list}
\end{minipage}\vspace{4ex}
\end{flushleft}
\begin{flushleft} \small
\begin{minipage}{\linewidth}\label{scrap10}\raggedright\small
\NWtarget{nuweb6b}{} $\langle\,${\itshape implementation of Equation~\eqref{eq:m0spinors:2}}\nobreak\ {\footnotesize {6b}}$\,\rangle\equiv$
\vspace{-1ex}
\begin{list}{}{} \item
\mbox{}\verb@@\hbox{\sffamily\bfseries Id}\verb@ inplorentz( 1, iv?, k1?, 0) *@\\
\mbox{}\verb@      inp(field1?, k1?, -1) =@\\
\mbox{}\verb@   NCContainer(USpb(k1), iv);@\\
\mbox{}\verb@@\hbox{\sffamily\bfseries Id}\verb@ outlorentz( 1, iv?, k1?, 0) *@\\
\mbox{}\verb@      out(field1?, k1?, -1) =@\\
\mbox{}\verb@   NCContainer(UbarSpa(k1), iv);@{\NWsep}
\end{list}
\vspace{-1.5ex}
\footnotesize
\begin{list}{}{\setlength{\itemsep}{-\parsep}\setlength{\itemindent}{-\leftmargin}}
\item \NWtxtMacroRefIn\ \NWlink{nuweb5c}{5c}.

\item{}
\end{list}
\end{minipage}\vspace{4ex}
\end{flushleft}
\begin{flushleft} \small
\begin{minipage}{\linewidth}\label{scrap11}\raggedright\small
\NWtarget{nuweb6c}{} $\langle\,${\itshape implementation of Equation~\eqref{eq:m0spinors:3}}\nobreak\ {\footnotesize {6c}}$\,\rangle\equiv$
\vspace{-1ex}
\begin{list}{}{} \item
\mbox{}\verb@@\hbox{\sffamily\bfseries Id}\verb@ outlorentz(-1, iv?, k1?, 0) *@\\
\mbox{}\verb@      out(field1?, k1?,  1) =@\\
\mbox{}\verb@   NCContainer(USpb(k1), iv);@\\
\mbox{}\verb@@\hbox{\sffamily\bfseries Id}\verb@ inplorentz(-1, iv?, k1?, 0) *@\\
\mbox{}\verb@      inp(field1?, k1?,  1) =@\\
\mbox{}\verb@   NCContainer(UbarSpa(k1), iv);@{\NWsep}
\end{list}
\vspace{-1.5ex}
\footnotesize
\begin{list}{}{\setlength{\itemsep}{-\parsep}\setlength{\itemindent}{-\leftmargin}}
\item \NWtxtMacroRefIn\ \NWlink{nuweb5c}{5c}.

\item{}
\end{list}
\end{minipage}\vspace{4ex}
\end{flushleft}
\begin{flushleft} \small
\begin{minipage}{\linewidth}\label{scrap12}\raggedright\small
\NWtarget{nuweb6d}{} $\langle\,${\itshape implementation of Equation~\eqref{eq:m0spinors:4}}\nobreak\ {\footnotesize {6d}}$\,\rangle\equiv$
\vspace{-1ex}
\begin{list}{}{} \item
\mbox{}\verb@@\hbox{\sffamily\bfseries Id}\verb@ outlorentz(-1, iv?, k1?, 0) *@\\
\mbox{}\verb@      out(field1?, k1?, -1) =@\\
\mbox{}\verb@   NCContainer(USpa(k1), iv);@\\
\mbox{}\verb@@\hbox{\sffamily\bfseries Id}\verb@ inplorentz(-1, iv?, k1?, 0) *@\\
\mbox{}\verb@      inp(field1?, k1?, -1) =@\\
\mbox{}\verb@   NCContainer(UbarSpb(k1), iv);@{\NWsep}
\end{list}
\vspace{-1.5ex}
\footnotesize
\begin{list}{}{\setlength{\itemsep}{-\parsep}\setlength{\itemindent}{-\leftmargin}}
\item \NWtxtMacroRefIn\ \NWlink{nuweb5c}{5c}.

\item{}
\end{list}
\end{minipage}\vspace{4ex}
\end{flushleft}
%---#]   Massless Spin-1/2 Particles :
\subsection{Massive Case}
%---#[   Massive Spin-1/2 Particles :
Massive spinors translate to spinney notation in the following
sense:
\begin{subequations}
\begin{align}
u_\alpha(k, +1) &= \kea{k^+} & \bar{u}_\alpha(k, +1) &= \brb{k^+}
\label{eq:Mspinors:1}\\
u_\alpha(k, -1) &= \keb{k^+} & \bar{u}_\alpha(k, -1) &= \bra{k^+}
\label{eq:Mspinors:2}\\
v_\alpha(k, +1) &= \keb{k^-} & \bar{v}_\alpha(k, +1) &= \bra{k^-}
\label{eq:Mspinors:3}\\
v_\alpha(k, -1) &= \kea{k^-} & \bar{v}_\alpha(k, -1) &= \brb{k^-}
\label{eq:Mspinors:4}
\end{align}
\end{subequations}
\begin{flushleft} \small
\begin{minipage}{\linewidth}\label{scrap13}\raggedright\small
\NWtarget{nuweb7a}{} $\langle\,${\itshape wave-functions for massive spinors}\nobreak\ {\footnotesize {7a}}$\,\rangle\equiv$
\vspace{-1ex}
\begin{list}{}{} \item
\mbox{}\verb@@\hbox{$\langle\,${\itshape implementation of Equation~\eqref{eq:Mspinors:1}}\nobreak\ {\footnotesize \NWlink{nuweb7b}{7b}}$\,\rangle$}\verb@@\\
\mbox{}\verb@@\hbox{$\langle\,${\itshape implementation of Equation~\eqref{eq:Mspinors:2}}\nobreak\ {\footnotesize \NWlink{nuweb7c}{7c}}$\,\rangle$}\verb@@\\
\mbox{}\verb@@\hbox{$\langle\,${\itshape implementation of Equation~\eqref{eq:Mspinors:3}}\nobreak\ {\footnotesize \NWlink{nuweb7d}{7d}}$\,\rangle$}\verb@@\\
\mbox{}\verb@@\hbox{$\langle\,${\itshape implementation of Equation~\eqref{eq:Mspinors:4}}\nobreak\ {\footnotesize \NWlink{nuweb7e}{7e}}$\,\rangle$}\verb@@{\NWsep}
\end{list}
\vspace{-1.5ex}
\footnotesize
\begin{list}{}{\setlength{\itemsep}{-\parsep}\setlength{\itemindent}{-\leftmargin}}
\item \NWtxtMacroRefIn\ \NWlink{nuweb2b}{2b}.

\item{}
\end{list}
\end{minipage}\vspace{4ex}
\end{flushleft}
\begin{flushleft} \small
\begin{minipage}{\linewidth}\label{scrap14}\raggedright\small
\NWtarget{nuweb7b}{} $\langle\,${\itshape implementation of Equation~\eqref{eq:Mspinors:1}}\nobreak\ {\footnotesize {7b}}$\,\rangle\equiv$
\vspace{-1ex}
\begin{list}{}{} \item
\mbox{}\verb@@\hbox{\sffamily\bfseries Id}\verb@ inplorentz( 1, iv?, k1?, m?) *@\\
\mbox{}\verb@      inp(field1?, k1?,  1) =@\\
\mbox{}\verb@   NCContainer(USpa(k1, +1), iv);@\\
\mbox{}\verb@@\hbox{\sffamily\bfseries Id}\verb@ outlorentz( 1, iv?, k1?, m?) *@\\
\mbox{}\verb@      out(field1?, k1?,  1) =@\\
\mbox{}\verb@   NCContainer(UbarSpb(k1, +1), iv);@{\NWsep}
\end{list}
\vspace{-1.5ex}
\footnotesize
\begin{list}{}{\setlength{\itemsep}{-\parsep}\setlength{\itemindent}{-\leftmargin}}
\item \NWtxtMacroRefIn\ \NWlink{nuweb7a}{7a}.

\item{}
\end{list}
\end{minipage}\vspace{4ex}
\end{flushleft}
\begin{flushleft} \small
\begin{minipage}{\linewidth}\label{scrap15}\raggedright\small
\NWtarget{nuweb7c}{} $\langle\,${\itshape implementation of Equation~\eqref{eq:Mspinors:2}}\nobreak\ {\footnotesize {7c}}$\,\rangle\equiv$
\vspace{-1ex}
\begin{list}{}{} \item
\mbox{}\verb@@\hbox{\sffamily\bfseries Id}\verb@ inplorentz( 1, iv?, k1?, m?) *@\\
\mbox{}\verb@      inp(field1?, k1?, -1) =@\\
\mbox{}\verb@   NCContainer(USpb(k1, +1), iv);@\\
\mbox{}\verb@@\hbox{\sffamily\bfseries Id}\verb@ outlorentz( 1, iv?, k1?, m?) *@\\
\mbox{}\verb@      out(field1?, k1?, -1) =@\\
\mbox{}\verb@   NCContainer(UbarSpa(k1, +1), iv);@{\NWsep}
\end{list}
\vspace{-1.5ex}
\footnotesize
\begin{list}{}{\setlength{\itemsep}{-\parsep}\setlength{\itemindent}{-\leftmargin}}
\item \NWtxtMacroRefIn\ \NWlink{nuweb7a}{7a}.

\item{}
\end{list}
\end{minipage}\vspace{4ex}
\end{flushleft}
\begin{flushleft} \small
\begin{minipage}{\linewidth}\label{scrap16}\raggedright\small
\NWtarget{nuweb7d}{} $\langle\,${\itshape implementation of Equation~\eqref{eq:Mspinors:3}}\nobreak\ {\footnotesize {7d}}$\,\rangle\equiv$
\vspace{-1ex}
\begin{list}{}{} \item
\mbox{}\verb@@\hbox{\sffamily\bfseries Id}\verb@ outlorentz(-1, iv?, k1?, m?) *@\\
\mbox{}\verb@      out(field1?, k1?,  1) =@\\
\mbox{}\verb@   NCContainer(USpb(k1, -1), iv);@\\
\mbox{}\verb@@\hbox{\sffamily\bfseries Id}\verb@ inplorentz(-1, iv?, k1?, m?) *@\\
\mbox{}\verb@      inp(field1?, k1?,  1) =@\\
\mbox{}\verb@   NCContainer(UbarSpa(k1, -1), iv);@{\NWsep}
\end{list}
\vspace{-1.5ex}
\footnotesize
\begin{list}{}{\setlength{\itemsep}{-\parsep}\setlength{\itemindent}{-\leftmargin}}
\item \NWtxtMacroRefIn\ \NWlink{nuweb7a}{7a}.

\item{}
\end{list}
\end{minipage}\vspace{4ex}
\end{flushleft}
\begin{flushleft} \small
\begin{minipage}{\linewidth}\label{scrap17}\raggedright\small
\NWtarget{nuweb7e}{} $\langle\,${\itshape implementation of Equation~\eqref{eq:Mspinors:4}}\nobreak\ {\footnotesize {7e}}$\,\rangle\equiv$
\vspace{-1ex}
\begin{list}{}{} \item
\mbox{}\verb@@\hbox{\sffamily\bfseries Id}\verb@ outlorentz(-1, iv?, k1?, m?) *@\\
\mbox{}\verb@      out(field1?, k1?, -1) =@\\
\mbox{}\verb@   NCContainer(USpa(k1, -1), iv);@\\
\mbox{}\verb@@\hbox{\sffamily\bfseries Id}\verb@ inplorentz(-1, iv?, k1?, m?) *@\\
\mbox{}\verb@      inp(field1?, k1?, -1) =@\\
\mbox{}\verb@   NCContainer(UbarSpb(k1, -1), iv);@{\NWsep}
\end{list}
\vspace{-1.5ex}
\footnotesize
\begin{list}{}{\setlength{\itemsep}{-\parsep}\setlength{\itemindent}{-\leftmargin}}
\item \NWtxtMacroRefIn\ \NWlink{nuweb7a}{7a}.

\item{}
\end{list}
\end{minipage}\vspace{4ex}
\end{flushleft}
%---#]   Massive Spin-1/2 Particles :
%---#] Spin-1/2 Particles :
\section{Spin-1 Particles}
%---#[ Spin-1 Particles :
For ingoing gauge bosons we use the polarisation vector
$\varepsilon_\mu(k, j_3)$, and for outgoing particles its
conjugate $\varepsilon_\mu^\ast(k, j_3)$ in accordance with the
notation of~\cite{1}. For internal particles we work in Feynman gauge
and hence get the propagator
\begin{equation}
\frac{-ig^{\mu\nu}}{k^2-m^2-im\Gamma+i0^+}\text{.}
\end{equation}
\begin{flushleft} \small
\begin{minipage}{\linewidth}\label{scrap18}\raggedright\small
\NWtarget{nuweb8a}{} $\langle\,${\itshape gauge boson propagator}\nobreak\ {\footnotesize {8a}}$\,\rangle\equiv$
\vspace{-1ex}
\begin{list}{}{} \item
\mbox{}\verb@@\hbox{\sffamily\bfseries Id}\verb@ proplorentz(2, k1?, m?, sDUMMY1?, 0, iv1?, iv2?) =@\\
\mbox{}\verb@   - PREFACTOR(@\hbox{\sffamily\bfseries i}\verb@_) * d(iv1, iv2) * inv(k1, m, sDUMMY1);@\\
\mbox{}\verb@@\hbox{\sffamily\bfseries Id}\verb@ proplorentz(2, 0, m?, sDUMMY1?, 0, iv1?, iv2?) =@\\
\mbox{}\verb@   - PREFACTOR(@\hbox{\sffamily\bfseries i}\verb@_) * d(iv1, iv2) * inv(ZERO, m, sDUMMY1);@{\NWsep}
\end{list}
\vspace{-1.5ex}
\footnotesize
\begin{list}{}{\setlength{\itemsep}{-\parsep}\setlength{\itemindent}{-\leftmargin}}
\item \NWtxtMacroRefIn\ \NWlink{nuweb3}{3}.

\item{}
\end{list}
\end{minipage}\vspace{4ex}
\end{flushleft}
\subsection{Massless Case}
%---#[   Massless Case :
We represent massless gauge bosons in the way proposed by~\cite{Xu},
\begin{align}
\varepsilon_\mu(k, +1) &= \frac{\Spab{q\vert\gamma_\mu\vert k}}{%
   \sqrt{2}\Spaa{qk}}\\
\varepsilon_\mu(k, -1) &= \frac{\Spba{q\vert\gamma_\mu\vert k}}{%
   \sqrt{2}\Spbb{kq}}
\end{align}
which requires an arbitrary, light-like auxilliary vector $q$. It
follows that in this representation
\begin{equation}\label{eq:polarisation-conjugate}
\left(\varepsilon_\mu(k,\pm1)\right)^\ast = \varepsilon_\mu(k,\mp1)\text{.}
\end{equation}

Below we implement the above expressions with the notation
$k=\mathtt{k1}$, $q=\mathtt{vDUMMY1}$ and $\mu=\mathtt{ivL2}$.
\begin{flushleft} \small
\begin{minipage}{\linewidth}\label{scrap19}\raggedright\small
\NWtarget{nuweb8b}{} $\langle\,${\itshape expression for $\varepsilon(k, +1)$}\nobreak\ {\footnotesize {8b}}$\,\rangle\equiv$
\vspace{-1ex}
\begin{list}{}{} \item
\mbox{}\verb@1/sqrt2 * SpDenominator(Spa2(vDUMMY1, k1)) *@\\
\mbox{}\verb@        UbarSpa(vDUMMY1) * Sm(ivL2) * USpb(k1)@{\NWsep}
\end{list}
\vspace{-1.5ex}
\footnotesize
\begin{list}{}{\setlength{\itemsep}{-\parsep}\setlength{\itemindent}{-\leftmargin}}
\item \NWtxtMacroRefIn\ \NWlink{nuweb9a}{9a}\NWlink{nuweb9b}{b}.

\item{}
\end{list}
\end{minipage}\vspace{4ex}
\end{flushleft}
\begin{flushleft} \small
\begin{minipage}{\linewidth}\label{scrap20}\raggedright\small
\NWtarget{nuweb8c}{} $\langle\,${\itshape expression for $\varepsilon(k, -1)$}\nobreak\ {\footnotesize {8c}}$\,\rangle\equiv$
\vspace{-1ex}
\begin{list}{}{} \item
\mbox{}\verb@1/sqrt2 * SpDenominator(Spb2(k1, vDUMMY1)) *@\\
\mbox{}\verb@        UbarSpb(vDUMMY1) * Sm(ivL2) * USpa(k1)@{\NWsep}
\end{list}
\vspace{-1.5ex}
\footnotesize
\begin{list}{}{\setlength{\itemsep}{-\parsep}\setlength{\itemindent}{-\leftmargin}}
\item \NWtxtMacroRefIn\ \NWlink{nuweb8d}{8d}\NWlink{nuweb9b}{, 9b}.

\item{}
\end{list}
\end{minipage}\vspace{4ex}
\end{flushleft}
Using Equation~\eqref{eq:polarisation-conjugate} we can also
define macros for the conjugate vectors
\begin{flushleft} \small
\begin{minipage}{\linewidth}\label{scrap21}\raggedright\small
\NWtarget{nuweb8d}{} $\langle\,${\itshape expression for $\varepsilon^\ast(k, +1)$}\nobreak\ {\footnotesize {8d}}$\,\rangle\equiv$
\vspace{-1ex}
\begin{list}{}{} \item
\mbox{}\verb@@\hbox{$\langle\,${\itshape expression for $\varepsilon(k, -1)$}\nobreak\ {\footnotesize \NWlink{nuweb8c}{8c}}$\,\rangle$}\verb@@{\NWsep}
\end{list}
\vspace{-1.5ex}
\footnotesize
\begin{list}{}{\setlength{\itemsep}{-\parsep}\setlength{\itemindent}{-\leftmargin}}
\item \NWtxtMacroRefIn\ \NWlink{nuweb9b}{9b}.

\item{}
\end{list}
\end{minipage}\vspace{4ex}
\end{flushleft}
\begin{flushleft} \small
\begin{minipage}{\linewidth}\label{scrap22}\raggedright\small
\NWtarget{nuweb9a}{} $\langle\,${\itshape expression for $\varepsilon^\ast(k, -1)$}\nobreak\ {\footnotesize {9a}}$\,\rangle\equiv$
\vspace{-1ex}
\begin{list}{}{} \item
\mbox{}\verb@@\hbox{$\langle\,${\itshape expression for $\varepsilon(k, +1)$}\nobreak\ {\footnotesize \NWlink{nuweb8b}{8b}}$\,\rangle$}\verb@@{\NWsep}
\end{list}
\vspace{-1.5ex}
\footnotesize
\begin{list}{}{\setlength{\itemsep}{-\parsep}\setlength{\itemindent}{-\leftmargin}}
\item \NWtxtMacroRefIn\ \NWlink{nuweb9b}{9b}.

\item{}
\end{list}
\end{minipage}\vspace{4ex}
\end{flushleft}
The usual properties for polarisation vectors are easy to prove. The
polarisation vector is transverse both to $k$ and $q$:
\begin{align}
\varepsilon(k, j_3)\cdot k &= 0\text{,}\\
\varepsilon(k, j_3)\cdot q &= 0\text{.}
\end{align}

The polarisation vectors fulfill the completeness relation of an
axial gauge,
\begin{equation}
\sum_{j_3=\pm1}\varepsilon^\mu(k, j_3)%
\left(\varepsilon^\nu(k, j_3)\right)^\ast=
-g^{\mu\nu}+\frac{k^\mu q^\nu+k^\nu q^\mu}{k\cdot q}\text{.}
\end{equation}

By making use of the Schouten identity one can show that a change
of the auxilliary vector $q\rightarrow p$ amounts to a term
proportional to $k_\mu$,
\begin{subequations}
\begin{align}
\frac{\Spab{q\vert\gamma_\mu\vert k}}{%
   \sqrt{2}\Spaa{qk}}
&=\frac{\Spab{p\vert\gamma_\mu\vert k}}{%
   \sqrt{2}\Spaa{pk}}+
\frac{\sqrt{2}\Spaa{pq}}{\Spaa{pk}\Spaa{qk}}k_\mu\\
\frac{\Spba{q\vert\gamma_\mu\vert k}}{%
   \sqrt{2}\Spbb{kq}}
&=\frac{\Spba{p\vert\gamma_\mu\vert k}}{%
   \sqrt{2}\Spbb{kp}}+
\frac{\sqrt{2}\Spbb{qp}}{\Spbb{kp}\Spbb{kq}}k_\mu
\end{align}
\end{subequations}

\begin{flushleft} \small
\begin{minipage}{\linewidth}\label{scrap23}\raggedright\small
\NWtarget{nuweb9b}{} $\langle\,${\itshape gauge boson wave-functions, light-like}\nobreak\ {\footnotesize {9b}}$\,\rangle\equiv$
\vspace{-1ex}
\begin{list}{}{} \item
\mbox{}\verb@@\hbox{\sffamily\bfseries Id}\verb@ outlorentz(2, ivL2?, k1?, 0) *@\\
\mbox{}\verb@      out(field1?, k1?,  1, vDUMMY1?) =@\\
\mbox{}\verb@   @\hbox{$\langle\,${\itshape expression for $\varepsilon^\ast(k, +1)$}\nobreak\ {\footnotesize \NWlink{nuweb8d}{8d}}$\,\rangle$}\verb@;@\\
\mbox{}\verb@@\hbox{\sffamily\bfseries Id}\verb@ outlorentz(2, ivL2?, k1?, 0) *@\\
\mbox{}\verb@      out(field1?, k1?, -1, vDUMMY1?) =@\\
\mbox{}\verb@   @\hbox{$\langle\,${\itshape expression for $\varepsilon^\ast(k, -1)$}\nobreak\ {\footnotesize \NWlink{nuweb9a}{9a}}$\,\rangle$}\verb@;@\\
\mbox{}\verb@@\hbox{\sffamily\bfseries Id}\verb@ inplorentz(2, ivL2?, k1?, 0) *@\\
\mbox{}\verb@      inp(field1?, k1?,  1, vDUMMY1?) =@\\
\mbox{}\verb@   @\hbox{$\langle\,${\itshape expression for $\varepsilon(k, +1)$}\nobreak\ {\footnotesize \NWlink{nuweb8b}{8b}}$\,\rangle$}\verb@;@\\
\mbox{}\verb@@\hbox{\sffamily\bfseries Id}\verb@ inplorentz(2, ivL2?, k1?, 0) *@\\
\mbox{}\verb@      inp(field1?, k1?, -1, vDUMMY1?) =@\\
\mbox{}\verb@   @\hbox{$\langle\,${\itshape expression for $\varepsilon(k, -1)$}\nobreak\ {\footnotesize \NWlink{nuweb8c}{8c}}$\,\rangle$}\verb@;@{\NWsep}
\end{list}
\vspace{-1.5ex}
\footnotesize
\begin{list}{}{\setlength{\itemsep}{-\parsep}\setlength{\itemindent}{-\leftmargin}}
\item \NWtxtMacroRefIn\ \NWlink{nuweb2b}{2b}.

\item{}
\end{list}
\end{minipage}\vspace{4ex}
\end{flushleft}
%---#]   Massless Case :
\subsection{Massive Case}
%---#[   Massive Case :
For the polarisation vectors of massive gauge bosons, where $k^2=m^2$,
we require\footnote{See for example Appendix A{.}1{.}1{.}6 of \cite{1}.}
\begin{align}
&\varepsilon(k, j_3)\cdot k = 0 & &\text{transverality,}\\
&\varepsilon(k, j_3)\cdot \varepsilon(k, j_3^\prime)=-\delta_{j_3j_3^\prime}&
&\text{orthonormality and}\\
&\sum_{j_3=-1}^1\varepsilon_\mu(k, j_3)\left(\varepsilon_\nu(k, j_3)\right)^\ast
=-g^{\mu\nu}+\frac{k^\mu k^\nu}{m^2}& &\text{completeness.}
\end{align}

We choose a representation based on the splitting of
the momentum $k$ into a pair of light-like vector $k^\flat$ and $q$,
as it is implemented in the \texttt{spinney} procedure
\texttt{SpLightConeDecomposition}.
\begin{equation}
k=k^\flat+\frac{m^2}{2k^\flat\cdot q}q
\end{equation}
Similarly to the massless case, two of the polarisations can be chosen
as
\begin{align}
\varepsilon_\mu(k, +1) &= \frac{\Spab{q\vert\gamma_\mu\vert k^\flat}}{%
   \sqrt{2}\Spaa{qk^\flat}}\quad\text{and}\\
\varepsilon_\mu(k, -1) &= \frac{\Spba{q\vert\gamma_\mu\vert k^\flat}}{%
   \sqrt{2}\Spbb{k^\flat q}}\text{.}
\end{align}
As before, these vectors have the property that they are complex
conjugate to each other.
The third polarisation vector is
\begin{equation}
\varepsilon_\mu(k, 0) = \frac{1}{m}\left(k^\flat_\mu
-\frac{m^2}{2k^\flat\cdot q}q_\mu\right)=
\frac{1}{m}\left(2k^\flat_\nu-k_\nu\right)\text{.}
\end{equation}

In the implementation we have $\mu=\mathtt{ivL2}$, $k=\mathtt{k1}$,
$k^\flat=\mathtt{k2}$ and $q=\mathtt{k3}$.
\begin{flushleft} \small
\begin{minipage}{\linewidth}\label{scrap24}\raggedright\small
\NWtarget{nuweb10a}{} $\langle\,${\itshape expression for massive $\varepsilon(k, +1)$}\nobreak\ {\footnotesize {10a}}$\,\rangle\equiv$
\vspace{-1ex}
\begin{list}{}{} \item
\mbox{}\verb@(1/sqrt2 * SpDenominator(Spa2(k3, k2))) *@\\
\mbox{}\verb@   UbarSpa(k3) * Sm(ivL2) * USpb(k2)@{\NWsep}
\end{list}
\vspace{-1.5ex}
\footnotesize
\begin{list}{}{\setlength{\itemsep}{-\parsep}\setlength{\itemindent}{-\leftmargin}}
\item \NWtxtMacroRefIn\ \NWlink{nuweb11b}{11b}\NWlink{nuweb11d}{d}\NWlink{nuweb14}{, 14}\NWlink{nuweb15}{, 15}\NWlink{nuweb20a}{, 20a}.

\item{}
\end{list}
\end{minipage}\vspace{4ex}
\end{flushleft}
\begin{flushleft} \small
\begin{minipage}{\linewidth}\label{scrap25}\raggedright\small
\NWtarget{nuweb10b}{} $\langle\,${\itshape expression for massive $\varepsilon(k, -1)$}\nobreak\ {\footnotesize {10b}}$\,\rangle\equiv$
\vspace{-1ex}
\begin{list}{}{} \item
\mbox{}\verb@(1/sqrt2 * SpDenominator(Spb2(k2, k3))) *@\\
\mbox{}\verb@   UbarSpb(k3) * Sm(ivL2) * USpa(k2)@{\NWsep}
\end{list}
\vspace{-1.5ex}
\footnotesize
\begin{list}{}{\setlength{\itemsep}{-\parsep}\setlength{\itemindent}{-\leftmargin}}
\item \NWtxtMacroRefIn\ \NWlink{nuweb11a}{11a}\NWlink{nuweb11d}{d}\NWlink{nuweb14}{, 14}\NWlink{nuweb15}{, 15}\NWlink{nuweb20a}{, 20a}.

\item{}
\end{list}
\end{minipage}\vspace{4ex}
\end{flushleft}
\begin{flushleft} \small
\begin{minipage}{\linewidth}\label{scrap26}\raggedright\small
\NWtarget{nuweb10c}{} $\langle\,${\itshape expression for massive $\varepsilon(k, 0)$}\nobreak\ {\footnotesize {10c}}$\,\rangle\equiv$
\vspace{-1ex}
\begin{list}{}{} \item
\mbox{}\verb@(1/m) * (k2(ivL2) - m * SpDenominator(Spa2(k2,k3)) *@\\
\mbox{}\verb@   m * SpDenominator(Spb2(k3,k2)) * k3(ivL2))@{\NWsep}
\end{list}
\vspace{-1.5ex}
\footnotesize
\begin{list}{}{\setlength{\itemsep}{-\parsep}\setlength{\itemindent}{-\leftmargin}}
\item \NWtxtMacroRefIn\ \NWlink{nuweb11c}{11c}\NWlink{nuweb11d}{d}\NWlink{nuweb14}{, 14}\NWlink{nuweb15}{, 15}\NWlink{nuweb20a}{, 20a}.

\item{}
\end{list}
\end{minipage}\vspace{4ex}
\end{flushleft}
The conjugate polarisation vectors are as follows.
\begin{flushleft} \small
\begin{minipage}{\linewidth}\label{scrap27}\raggedright\small
\NWtarget{nuweb11a}{} $\langle\,${\itshape expression for massive $\varepsilon^\ast(k, +1)$}\nobreak\ {\footnotesize {11a}}$\,\rangle\equiv$
\vspace{-1ex}
\begin{list}{}{} \item
\mbox{}\verb@@\hbox{$\langle\,${\itshape expression for massive $\varepsilon(k, -1)$}\nobreak\ {\footnotesize \NWlink{nuweb10b}{10b}}$\,\rangle$}\verb@@{\NWsep}
\end{list}
\vspace{-1.5ex}
\footnotesize
\begin{list}{}{\setlength{\itemsep}{-\parsep}\setlength{\itemindent}{-\leftmargin}}
\item \NWtxtMacroRefIn\ \NWlink{nuweb11d}{11d}\NWlink{nuweb16}{, 16}\NWlink{nuweb17}{, 17}\NWlink{nuweb20c}{, 20c}.

\item{}
\end{list}
\end{minipage}\vspace{4ex}
\end{flushleft}
\begin{flushleft} \small
\begin{minipage}{\linewidth}\label{scrap28}\raggedright\small
\NWtarget{nuweb11b}{} $\langle\,${\itshape expression for massive $\varepsilon^\ast(k, -1)$}\nobreak\ {\footnotesize {11b}}$\,\rangle\equiv$
\vspace{-1ex}
\begin{list}{}{} \item
\mbox{}\verb@@\hbox{$\langle\,${\itshape expression for massive $\varepsilon(k, +1)$}\nobreak\ {\footnotesize \NWlink{nuweb10a}{10a}}$\,\rangle$}\verb@@{\NWsep}
\end{list}
\vspace{-1.5ex}
\footnotesize
\begin{list}{}{\setlength{\itemsep}{-\parsep}\setlength{\itemindent}{-\leftmargin}}
\item \NWtxtMacroRefIn\ \NWlink{nuweb11d}{11d}\NWlink{nuweb16}{, 16}\NWlink{nuweb17}{, 17}\NWlink{nuweb20c}{, 20c}.

\item{}
\end{list}
\end{minipage}\vspace{4ex}
\end{flushleft}
\begin{flushleft} \small
\begin{minipage}{\linewidth}\label{scrap29}\raggedright\small
\NWtarget{nuweb11c}{} $\langle\,${\itshape expression for massive $\varepsilon^\ast(k, 0)$}\nobreak\ {\footnotesize {11c}}$\,\rangle\equiv$
\vspace{-1ex}
\begin{list}{}{} \item
\mbox{}\verb@@\hbox{$\langle\,${\itshape expression for massive $\varepsilon(k, 0)$}\nobreak\ {\footnotesize \NWlink{nuweb10c}{10c}}$\,\rangle$}\verb@@{\NWsep}
\end{list}
\vspace{-1.5ex}
\footnotesize
\begin{list}{}{\setlength{\itemsep}{-\parsep}\setlength{\itemindent}{-\leftmargin}}
\item \NWtxtMacroRefIn\ \NWlink{nuweb11d}{11d}\NWlink{nuweb16}{, 16}\NWlink{nuweb17}{, 17}\NWlink{nuweb20c}{, 20c}.

\item{}
\end{list}
\end{minipage}\vspace{4ex}
\end{flushleft}
Finally, we can express all six possibilities of initial state
and final state polarisation vectors:
\begin{flushleft} \small\label{scrap30}\raggedright\small
\NWtarget{nuweb11d}{} $\langle\,${\itshape gauge boson wave-functions, massive}\nobreak\ {\footnotesize {11d}}$\,\rangle\equiv$
\vspace{-1ex}
\begin{list}{}{} \item
\mbox{}\verb@@\hbox{\sffamily\bfseries Id}\verb@ outlorentz(2, ivL2?, k1?, m?) *@\\
\mbox{}\verb@      out(field1?, k1?,  1, k2?, k3?) =@\\
\mbox{}\verb@   @\hbox{$\langle\,${\itshape expression for massive $\varepsilon^\ast(k, +1)$}\nobreak\ {\footnotesize \NWlink{nuweb11a}{11a}}$\,\rangle$}\verb@;@\\
\mbox{}\verb@@\hbox{\sffamily\bfseries Id}\verb@ outlorentz(2, ivL2?, k1?, m?) *@\\
\mbox{}\verb@      out(field1?, k1?, -1, k2?, k3?) =@\\
\mbox{}\verb@   @\hbox{$\langle\,${\itshape expression for massive $\varepsilon^\ast(k, -1)$}\nobreak\ {\footnotesize \NWlink{nuweb11b}{11b}}$\,\rangle$}\verb@;@\\
\mbox{}\verb@@\hbox{\sffamily\bfseries Id}\verb@ outlorentz(2, ivL2?, k1?, m?) *@\\
\mbox{}\verb@      out(field1?, k1?,  0, k2?, k3?) =@\\
\mbox{}\verb@   @\hbox{$\langle\,${\itshape expression for massive $\varepsilon^\ast(k, 0)$}\nobreak\ {\footnotesize \NWlink{nuweb11c}{11c}}$\,\rangle$}\verb@;@\\
\mbox{}\verb@@\hbox{\sffamily\bfseries Id}\verb@ inplorentz(2, ivL2?, k1?, m?) *@\\
\mbox{}\verb@      inp(field1?, k1?,  1, k2?, k3?) =@\\
\mbox{}\verb@   @\hbox{$\langle\,${\itshape expression for massive $\varepsilon(k, +1)$}\nobreak\ {\footnotesize \NWlink{nuweb10a}{10a}}$\,\rangle$}\verb@;@\\
\mbox{}\verb@@\hbox{\sffamily\bfseries Id}\verb@ inplorentz(2, ivL2?, k1?, m?) *@\\
\mbox{}\verb@      inp(field1?, k1?, -1, k2?, k3?) =@\\
\mbox{}\verb@   @\hbox{$\langle\,${\itshape expression for massive $\varepsilon(k, -1)$}\nobreak\ {\footnotesize \NWlink{nuweb10b}{10b}}$\,\rangle$}\verb@;@\\
\mbox{}\verb@@\hbox{\sffamily\bfseries Id}\verb@ inplorentz(2, ivL2?, k1?, m?) *@\\
\mbox{}\verb@      inp(field1?, k1?,  0, k2?, k3?) =@\\
\mbox{}\verb@   @\hbox{$\langle\,${\itshape expression for massive $\varepsilon(k, 0)$}\nobreak\ {\footnotesize \NWlink{nuweb10c}{10c}}$\,\rangle$}\verb@;@{\NWsep}
\end{list}
\vspace{-1.5ex}
\footnotesize
\begin{list}{}{\setlength{\itemsep}{-\parsep}\setlength{\itemindent}{-\leftmargin}}
\item \NWtxtMacroRefIn\ \NWlink{nuweb2b}{2b}.

\item{}
\end{list}
\vspace{4ex}
\end{flushleft}
\TODO{}\textit{In cases where no massless vectors are in the process
\golem\ chooses the procedure \texttt{SpLightConeSplitting} where a pair
of massive vectors $P$, $Q$ is split into a pair of light-like vectors
$p$, $q$. The corresponding formul\ae\ for polarisation vectors have to
be worked out. Since this case is for very specific processes only we
leave this for the future.}

%---#]   Massive Case :
%---#] Spin-1 Particles :

\section{Spin-\texorpdfstring{$\frac32$}{3/2} Particles}
%---#[ Spin-3/2 Particles:
For the implementation of massive Spin-$\frac32$ fields we
follow~\cite{Kilian:2007gr}.

The projector is
\begin{equation}
\Pi^{\mu\nu}=\left(\fmslash{p}+m\right)
\left(\frac{p^\mu p^\nu}{m^2}-g^{\mu\nu}\right)-\frac13
\left(\gamma^\mu+\frac{k^\mu}{m}\right)
\left(\fmslash{k}-m\right)
\left(\gamma^\nu+\frac{k^\nu}{m}\right)\text{.}
\end{equation}
\begin{flushleft} \small
\begin{minipage}{\linewidth}\label{scrap31}\raggedright\small
\NWtarget{nuweb12}{} $\langle\,${\itshape vector-spinor propagators}\nobreak\ {\footnotesize {12}}$\,\rangle\equiv$
\vspace{-1ex}
\begin{list}{}{} \item
\mbox{}\verb@@\hbox{\sffamily\bfseries Repeat}\verb@;@\\
\mbox{}\verb@   @\hbox{\sffamily\bfseries Id}\verb@ @\hbox{\sffamily\bfseries once}\verb@ proplorentz(3, k1?, m?, sDUMMY1?, 0, iv1?, iv2?) =@\\
\mbox{}\verb@      PREFACTOR(@\hbox{\sffamily\bfseries i}\verb@_) *@\\
\mbox{}\verb@      SplitLorentzIndex(iv1, iv1L2, iv1L1) *@\\
\mbox{}\verb@      SplitLorentzIndex(iv2, iv2L2, iv2L1) * 1/3 * (@\\
\mbox{}\verb@         + 4*k1(iv1L2)*k1(iv2L2)/csqrt(m*(m-i_*sDUMMY1))@\\
\mbox{}\verb@         - 3*d(iv1L2,iv2L2)*csqrt(m*(m-i_*sDUMMY1))@\\
\mbox{}\verb@         + 2*NCContainer(Sm(k1),iv1L1,iv2L1)*k1(iv1L2)*k1(iv2L2)/csqrt(m*(m-i_*sDUMMY1))^2@\\
\mbox{}\verb@         - 3*NCContainer(Sm(k1),iv1L1,iv2L1)*d(iv1L2,iv2L2)@\\
\mbox{}\verb@         - NCContainer(Sm(k1)*Sm(iv2L2),iv1L1,iv2L1)*k1(iv1L2)/csqrt(m*(m-i_*sDUMMY1))@\\
\mbox{}\verb@         + NCContainer(Sm(iv1L2),iv1L1,iv2L1)*k1(iv2L2)@\\
\mbox{}\verb@         - NCContainer(Sm(iv1L2)*Sm(k1),iv1L1,iv2L1)*k1(iv2L2)/csqrt(m*(m-i_*sDUMMY1))@\\
\mbox{}\verb@         - NCContainer(Sm(iv1L2)*Sm(k1)*Sm(iv2L2),iv1L1,iv2L1)@\\
\mbox{}\verb@         + NCContainer(Sm(iv1L2)*Sm(iv2L2),iv1L1,iv2L1)*csqrt(m*(m-i_*sDUMMY1))@\\
\mbox{}\verb@         + NCContainer(Sm(iv2L2),iv1L1,iv2L1)*k1(iv1L2)@\\
\mbox{}\verb@      ) * inv(k1, m, sDUMMY1);@\\
\mbox{}\verb@   Sum iv1L2, iv1L1, iv2L2, iv2L1;@\\
\mbox{}\verb@@\hbox{\sffamily\bfseries EndRepeat}\verb@;@\\
\mbox{}\verb@@\hbox{\sffamily\bfseries Repeat}\verb@;@\\
\mbox{}\verb@   @\hbox{\sffamily\bfseries Id}\verb@ @\hbox{\sffamily\bfseries once}\verb@ proplorentz(3, k1?, m?, 0, 0, iv1?, iv2?) =@\\
\mbox{}\verb@      PREFACTOR(@\hbox{\sffamily\bfseries i}\verb@_) *@\\
\mbox{}\verb@      SplitLorentzIndex(iv1, iv1L2, iv1L1) *@\\
\mbox{}\verb@      SplitLorentzIndex(iv2, iv2L2, iv2L1) * 1/3 * (@\\
\mbox{}\verb@         + 4*k1(iv1L2)*k1(iv2L2)/m@\\
\mbox{}\verb@         - 3*d(iv1L2,iv2L2)*m@\\
\mbox{}\verb@         + 2*NCContainer(Sm(k1),iv1L1,iv2L1)*k1(iv1L2)*k1(iv2L2)/m^2@\\
\mbox{}\verb@         - 3*NCContainer(Sm(k1),iv1L1,iv2L1)*d(iv1L2,iv2L2)@\\
\mbox{}\verb@         - NCContainer(Sm(k1)*Sm(iv2L2),iv1L1,iv2L1)*k1(iv1L2)/m@\\
\mbox{}\verb@         + NCContainer(Sm(iv1L2),iv1L1,iv2L1)*k1(iv2L2)@\\
\mbox{}\verb@         - NCContainer(Sm(iv1L2)*Sm(k1),iv1L1,iv2L1)*k1(iv2L2)/m@\\
\mbox{}\verb@         - NCContainer(Sm(iv1L2)*Sm(k1)*Sm(iv2L2),iv1L1,iv2L1)@\\
\mbox{}\verb@         + NCContainer(Sm(iv1L2)*Sm(iv2L2),iv1L1,iv2L1)*m@\\
\mbox{}\verb@         + NCContainer(Sm(iv2L2),iv1L1,iv2L1)*k1(iv1L2)@\\
\mbox{}\verb@      ) * inv(k1, m, 0);@\\
\mbox{}\verb@   @\hbox{\sffamily\bfseries Sum}\verb@ iv1L2, iv1L1, iv2L2, iv2L1;@\\
\mbox{}\verb@@\hbox{\sffamily\bfseries EndRepeat}\verb@;@{\NWsep}
\end{list}
\vspace{-1.5ex}
\footnotesize
\begin{list}{}{\setlength{\itemsep}{-\parsep}\setlength{\itemindent}{-\leftmargin}}
\item \NWtxtMacroRefIn\ \NWlink{nuweb3}{3}.

\item{}
\end{list}
\end{minipage}\vspace{4ex}
\end{flushleft}
A set of eigenvectors is provided by the following five states:
\begin{eqnarray}
\epsilon^\mu_{+3/2}(p) &=& \epsilon^\mu_+(p)\epsilon_+(p)\\
\epsilon^\mu_{+1/2}(p) &=&
   \frac1{\sqrt{3}}\epsilon^\mu_+(p)\epsilon_-(p)
  +\sqrt{\frac23}\epsilon^\mu_0(p)\epsilon_+(p) \\
\epsilon^\mu_{-1/2}(p) &=&
   \frac1{\sqrt{3}}\epsilon^\mu_-(p)\epsilon_+(p)
  +\sqrt{\frac23}\epsilon^\mu_0(p)\epsilon_-(p) \\
\epsilon^\mu_{-3/2}(p) &=& \epsilon^\mu_-(p)\epsilon_-(p)
\end{eqnarray}

There are sixteen different cases
\begin{itemize}
\item in-/outgoing
\item particle/anti-particle
\item polarisation $-\frac32, -\frac12, \frac12, \frac32$
\end{itemize}

\begin{flushleft} \small
\begin{minipage}{\linewidth}\label{scrap32}\raggedright\small
\NWtarget{nuweb13}{} $\langle\,${\itshape vector-spinor wave functions}\nobreak\ {\footnotesize {13}}$\,\rangle\equiv$
\vspace{-1ex}
\begin{list}{}{} \item
\mbox{}\verb@@\hbox{$\langle\,${\itshape ingoing vector-spinor particle}\nobreak\ {\footnotesize \NWlink{nuweb14}{14}}$\,\rangle$}\verb@@\\
\mbox{}\verb@@\hbox{$\langle\,${\itshape ingoing vector-spinor anti-particle}\nobreak\ {\footnotesize \NWlink{nuweb15}{15}}$\,\rangle$}\verb@@\\
\mbox{}\verb@@\hbox{$\langle\,${\itshape outgoing vector-spinor particle}\nobreak\ {\footnotesize \NWlink{nuweb16}{16}}$\,\rangle$}\verb@@\\
\mbox{}\verb@@\hbox{$\langle\,${\itshape outgoing vector-spinor anti-particle}\nobreak\ {\footnotesize \NWlink{nuweb17}{17}}$\,\rangle$}\verb@@{\NWsep}
\end{list}
\vspace{-1.5ex}
\footnotesize
\begin{list}{}{\setlength{\itemsep}{-\parsep}\setlength{\itemindent}{-\leftmargin}}
\item \NWtxtMacroRefIn\ \NWlink{nuweb2b}{2b}.

\item{}
\end{list}
\end{minipage}\vspace{4ex}
\end{flushleft}
\begin{flushleft} \small
\begin{minipage}{\linewidth}\label{scrap33}\raggedright\small
\NWtarget{nuweb14}{} $\langle\,${\itshape ingoing vector-spinor particle}\nobreak\ {\footnotesize {14}}$\,\rangle\equiv$
\vspace{-1ex}
\begin{list}{}{} \item
\mbox{}\verb@@\hbox{\sffamily\bfseries Id}\verb@ @\hbox{\sffamily\bfseries once}\verb@ inplorentz(3, ivL?, k1?, m?!{0,}) *@\\
\mbox{}\verb@      inp(field1?, k1?, -2, k2?, k3?) =@\\
\mbox{}\verb@   SplitLorentzIndex(ivL, ivL2, ivL1) *@\\
\mbox{}\verb@   @\hbox{$\langle\,${\itshape expression for massive $\varepsilon(k, -1)$}\nobreak\ {\footnotesize \NWlink{nuweb10b}{10b}}$\,\rangle$}\verb@ *@\\
\mbox{}\verb@   NCContainer(USpb(k1,+1), ivL1);@\\
\mbox{}\verb@   @\hbox{\sffamily\bfseries Sum}\verb@ ivL2, ivL1;@\\
\mbox{}\verb@@\hbox{\sffamily\bfseries Id}\verb@ @\hbox{\sffamily\bfseries once}\verb@ inplorentz(3, ivL?, k1?, m?!{0,}) *@\\
\mbox{}\verb@      inp(field1?, k1?, -1, k2?, k3?) = 1/sqrt3 *@\\
\mbox{}\verb@   SplitLorentzIndex(ivL, ivL2, ivL1) * (@\\
\mbox{}\verb@   + @\hbox{$\langle\,${\itshape expression for massive $\varepsilon(k, -1)$}\nobreak\ {\footnotesize \NWlink{nuweb10b}{10b}}$\,\rangle$}\verb@ *@\\
\mbox{}\verb@     NCContainer(USpa(k1,+1), ivL1)@\\
\mbox{}\verb@   + sqrt2 * @\hbox{$\langle\,${\itshape expression for massive $\varepsilon(k, 0)$}\nobreak\ {\footnotesize \NWlink{nuweb10c}{10c}}$\,\rangle$}\verb@ *@\\
\mbox{}\verb@     NCContainer(USpb(k1,+1), ivL1));@\\
\mbox{}\verb@   @\hbox{\sffamily\bfseries Sum}\verb@ ivL2, ivL1;@\\
\mbox{}\verb@@\hbox{\sffamily\bfseries Id}\verb@ @\hbox{\sffamily\bfseries once}\verb@ inplorentz(3, ivL?, k1?, m?!{0,}) *@\\
\mbox{}\verb@      inp(field1?, k1?, +1, k2?, k3?) = 1/sqrt3 *@\\
\mbox{}\verb@   SplitLorentzIndex(ivL, ivL2, ivL1) * (@\\
\mbox{}\verb@   + @\hbox{$\langle\,${\itshape expression for massive $\varepsilon(k, +1)$}\nobreak\ {\footnotesize \NWlink{nuweb10a}{10a}}$\,\rangle$}\verb@ *@\\
\mbox{}\verb@     NCContainer(USpb(k1,+1), ivL1)@\\
\mbox{}\verb@   + sqrt2 * @\hbox{$\langle\,${\itshape expression for massive $\varepsilon(k, 0)$}\nobreak\ {\footnotesize \NWlink{nuweb10c}{10c}}$\,\rangle$}\verb@ *@\\
\mbox{}\verb@     NCContainer(USpa(k1,+1), ivL1));@\\
\mbox{}\verb@   @\hbox{\sffamily\bfseries Sum}\verb@ ivL2, ivL1;@\\
\mbox{}\verb@@\hbox{\sffamily\bfseries Id}\verb@ @\hbox{\sffamily\bfseries once}\verb@ inplorentz(3, ivL?, k1?, m?!{0,}) *@\\
\mbox{}\verb@      inp(field1?, k1?, +2, k2?, k3?) =@\\
\mbox{}\verb@   SplitLorentzIndex(ivL, ivL2, ivL1) *@\\
\mbox{}\verb@   @\hbox{$\langle\,${\itshape expression for massive $\varepsilon(k, +1)$}\nobreak\ {\footnotesize \NWlink{nuweb10a}{10a}}$\,\rangle$}\verb@ *@\\
\mbox{}\verb@   NCContainer(USpa(k1,+1), ivL1);@\\
\mbox{}\verb@   @\hbox{\sffamily\bfseries Sum}\verb@ ivL2, ivL1;@\\
\mbox{}\verb@@{\NWsep}
\end{list}
\vspace{-1.5ex}
\footnotesize
\begin{list}{}{\setlength{\itemsep}{-\parsep}\setlength{\itemindent}{-\leftmargin}}
\item \NWtxtMacroRefIn\ \NWlink{nuweb13}{13}.

\item{}
\end{list}
\end{minipage}\vspace{4ex}
\end{flushleft}
\begin{flushleft} \small
\begin{minipage}{\linewidth}\label{scrap34}\raggedright\small
\NWtarget{nuweb15}{} $\langle\,${\itshape ingoing vector-spinor anti-particle}\nobreak\ {\footnotesize {15}}$\,\rangle\equiv$
\vspace{-1ex}
\begin{list}{}{} \item
\mbox{}\verb@@\hbox{\sffamily\bfseries Id}\verb@ @\hbox{\sffamily\bfseries once}\verb@ inplorentz(-3, ivL?, k1?, m?!{0,}) *@\\
\mbox{}\verb@      inp(field1?, k1?, -2, k2?, k3?) =@\\
\mbox{}\verb@   SplitLorentzIndex(ivL, ivL2, ivL1) *@\\
\mbox{}\verb@   @\hbox{$\langle\,${\itshape expression for massive $\varepsilon(k, -1)$}\nobreak\ {\footnotesize \NWlink{nuweb10b}{10b}}$\,\rangle$}\verb@ *@\\
\mbox{}\verb@   NCContainer(UbarSpb(k1,-1), ivL1);@\\
\mbox{}\verb@   @\hbox{\sffamily\bfseries Sum}\verb@ ivL2, ivL1;@\\
\mbox{}\verb@@\hbox{\sffamily\bfseries Id}\verb@ @\hbox{\sffamily\bfseries once}\verb@ inplorentz(-3, ivL?, k1?, m?!{0,}) *@\\
\mbox{}\verb@      inp(field1?, k1?, -1, k2?, k3?) = 1/sqrt3 *@\\
\mbox{}\verb@   SplitLorentzIndex(ivL, ivL2, ivL1) * (@\\
\mbox{}\verb@   + @\hbox{$\langle\,${\itshape expression for massive $\varepsilon(k, -1)$}\nobreak\ {\footnotesize \NWlink{nuweb10b}{10b}}$\,\rangle$}\verb@ *@\\
\mbox{}\verb@     NCContainer(UbarSpa(k1,-1), ivL1)@\\
\mbox{}\verb@   + sqrt2 * @\hbox{$\langle\,${\itshape expression for massive $\varepsilon(k, 0)$}\nobreak\ {\footnotesize \NWlink{nuweb10c}{10c}}$\,\rangle$}\verb@ *@\\
\mbox{}\verb@     NCContainer(UbarSpb(k1,-1), ivL1));@\\
\mbox{}\verb@   @\hbox{\sffamily\bfseries Sum}\verb@ ivL2, ivL1;@\\
\mbox{}\verb@@\hbox{\sffamily\bfseries Id}\verb@ @\hbox{\sffamily\bfseries once}\verb@ inplorentz(-3, ivL?, k1?, m?!{0,}) *@\\
\mbox{}\verb@      inp(field1?, k1?, +1, k2?, k3?) = 1/sqrt3 *@\\
\mbox{}\verb@   SplitLorentzIndex(ivL, ivL2, ivL1) * (@\\
\mbox{}\verb@   + @\hbox{$\langle\,${\itshape expression for massive $\varepsilon(k, +1)$}\nobreak\ {\footnotesize \NWlink{nuweb10a}{10a}}$\,\rangle$}\verb@ *@\\
\mbox{}\verb@     NCContainer(UbarSpb(k1,-1), ivL1)@\\
\mbox{}\verb@   + sqrt2 * @\hbox{$\langle\,${\itshape expression for massive $\varepsilon(k, 0)$}\nobreak\ {\footnotesize \NWlink{nuweb10c}{10c}}$\,\rangle$}\verb@ *@\\
\mbox{}\verb@     NCContainer(UbarSpa(k1,-1), ivL1));@\\
\mbox{}\verb@   @\hbox{\sffamily\bfseries Sum}\verb@ ivL2, ivL1;@\\
\mbox{}\verb@@\hbox{\sffamily\bfseries Id}\verb@ @\hbox{\sffamily\bfseries once}\verb@ inplorentz(-3, ivL?, k1?, m?!{0,}) *@\\
\mbox{}\verb@      inp(field1?, k1?, +2, k2?, k3?) =@\\
\mbox{}\verb@   SplitLorentzIndex(ivL, ivL2, ivL1) *@\\
\mbox{}\verb@   @\hbox{$\langle\,${\itshape expression for massive $\varepsilon(k, +1)$}\nobreak\ {\footnotesize \NWlink{nuweb10a}{10a}}$\,\rangle$}\verb@ *@\\
\mbox{}\verb@   NCContainer(UbarSpa(k1,-1), ivL1);@\\
\mbox{}\verb@   @\hbox{\sffamily\bfseries Sum}\verb@ ivL2, ivL1;@\\
\mbox{}\verb@@{\NWsep}
\end{list}
\vspace{-1.5ex}
\footnotesize
\begin{list}{}{\setlength{\itemsep}{-\parsep}\setlength{\itemindent}{-\leftmargin}}
\item \NWtxtMacroRefIn\ \NWlink{nuweb13}{13}.

\item{}
\end{list}
\end{minipage}\vspace{4ex}
\end{flushleft}
\begin{flushleft} \small
\begin{minipage}{\linewidth}\label{scrap35}\raggedright\small
\NWtarget{nuweb16}{} $\langle\,${\itshape outgoing vector-spinor particle}\nobreak\ {\footnotesize {16}}$\,\rangle\equiv$
\vspace{-1ex}
\begin{list}{}{} \item
\mbox{}\verb@@\hbox{\sffamily\bfseries Id}\verb@ @\hbox{\sffamily\bfseries once}\verb@ inplorentz(3, ivL?, k1?, m?!{0,}) *@\\
\mbox{}\verb@      inp(field1?, k1?, -2, k2?, k3?) =@\\
\mbox{}\verb@   SplitLorentzIndex(ivL, ivL2, ivL1) *@\\
\mbox{}\verb@   @\hbox{$\langle\,${\itshape expression for massive $\varepsilon^\ast(k, -1)$}\nobreak\ {\footnotesize \NWlink{nuweb11b}{11b}}$\,\rangle$}\verb@ *@\\
\mbox{}\verb@   NCContainer(UbarSpa(k1,+1), ivL1);@\\
\mbox{}\verb@   @\hbox{\sffamily\bfseries Sum}\verb@ ivL2, ivL1;@\\
\mbox{}\verb@@\hbox{\sffamily\bfseries Id}\verb@ @\hbox{\sffamily\bfseries once}\verb@ inplorentz(3, ivL?, k1?, m?!{0,}) *@\\
\mbox{}\verb@      inp(field1?, k1?, -1, k2?, k3?) = 1/sqrt3 *@\\
\mbox{}\verb@   SplitLorentzIndex(ivL, ivL2, ivL1) * (@\\
\mbox{}\verb@   + @\hbox{$\langle\,${\itshape expression for massive $\varepsilon^\ast(k, -1)$}\nobreak\ {\footnotesize \NWlink{nuweb11b}{11b}}$\,\rangle$}\verb@ *@\\
\mbox{}\verb@     NCContainer(UbarSpb(k1,+1), ivL1)@\\
\mbox{}\verb@   + sqrt2 * @\hbox{$\langle\,${\itshape expression for massive $\varepsilon^\ast(k, 0)$}\nobreak\ {\footnotesize \NWlink{nuweb11c}{11c}}$\,\rangle$}\verb@ *@\\
\mbox{}\verb@     NCContainer(UbarSpa(k1,+1), ivL1));@\\
\mbox{}\verb@   @\hbox{\sffamily\bfseries Sum}\verb@ ivL2, ivL1;@\\
\mbox{}\verb@@\hbox{\sffamily\bfseries Id}\verb@ @\hbox{\sffamily\bfseries once}\verb@ inplorentz(3, ivL?, k1?, m?!{0,}) *@\\
\mbox{}\verb@      inp(field1?, k1?, +1, k2?, k3?) = 1/sqrt3 *@\\
\mbox{}\verb@   SplitLorentzIndex(ivL, ivL2, ivL1) * (@\\
\mbox{}\verb@   + @\hbox{$\langle\,${\itshape expression for massive $\varepsilon^\ast(k, +1)$}\nobreak\ {\footnotesize \NWlink{nuweb11a}{11a}}$\,\rangle$}\verb@ *@\\
\mbox{}\verb@     NCContainer(UbarSpa(k1,+1), ivL1)@\\
\mbox{}\verb@   + sqrt2 * @\hbox{$\langle\,${\itshape expression for massive $\varepsilon^\ast(k, 0)$}\nobreak\ {\footnotesize \NWlink{nuweb11c}{11c}}$\,\rangle$}\verb@ *@\\
\mbox{}\verb@     NCContainer(UbarSpb(k1,+1), ivL1));@\\
\mbox{}\verb@   @\hbox{\sffamily\bfseries Sum}\verb@ ivL2, ivL1;@\\
\mbox{}\verb@@\hbox{\sffamily\bfseries Id}\verb@ @\hbox{\sffamily\bfseries once}\verb@ inplorentz(3, ivL?, k1?, m?!{0,}) *@\\
\mbox{}\verb@      inp(field1?, k1?, +2, k2?, k3?) =@\\
\mbox{}\verb@   SplitLorentzIndex(ivL, ivL2, ivL1) *@\\
\mbox{}\verb@   @\hbox{$\langle\,${\itshape expression for massive $\varepsilon^\ast(k, +1)$}\nobreak\ {\footnotesize \NWlink{nuweb11a}{11a}}$\,\rangle$}\verb@ *@\\
\mbox{}\verb@   NCContainer(UbarSpb(k1,+1), ivL1);@\\
\mbox{}\verb@   @\hbox{\sffamily\bfseries Sum}\verb@ ivL2, ivL1;@\\
\mbox{}\verb@@{\NWsep}
\end{list}
\vspace{-1.5ex}
\footnotesize
\begin{list}{}{\setlength{\itemsep}{-\parsep}\setlength{\itemindent}{-\leftmargin}}
\item \NWtxtMacroRefIn\ \NWlink{nuweb13}{13}.

\item{}
\end{list}
\end{minipage}\vspace{4ex}
\end{flushleft}
\begin{flushleft} \small
\begin{minipage}{\linewidth}\label{scrap36}\raggedright\small
\NWtarget{nuweb17}{} $\langle\,${\itshape outgoing vector-spinor anti-particle}\nobreak\ {\footnotesize {17}}$\,\rangle\equiv$
\vspace{-1ex}
\begin{list}{}{} \item
\mbox{}\verb@@\hbox{\sffamily\bfseries Id}\verb@ @\hbox{\sffamily\bfseries once}\verb@ inplorentz(3, ivL?, k1?, m?!{0,}) *@\\
\mbox{}\verb@      inp(field1?, k1?, -2, k2?, k3?) =@\\
\mbox{}\verb@   SplitLorentzIndex(ivL, ivL2, ivL1) *@\\
\mbox{}\verb@   @\hbox{$\langle\,${\itshape expression for massive $\varepsilon^\ast(k, -1)$}\nobreak\ {\footnotesize \NWlink{nuweb11b}{11b}}$\,\rangle$}\verb@ *@\\
\mbox{}\verb@   NCContainer(USpa(k1,-1), ivL1);@\\
\mbox{}\verb@   @\hbox{\sffamily\bfseries Sum}\verb@ ivL2, ivL1;@\\
\mbox{}\verb@@\hbox{\sffamily\bfseries Id}\verb@ @\hbox{\sffamily\bfseries once}\verb@ inplorentz(3, ivL?, k1?, m?!{0,}) *@\\
\mbox{}\verb@      inp(field1?, k1?, -1, k2?, k3?) = 1/sqrt3 *@\\
\mbox{}\verb@   SplitLorentzIndex(ivL, ivL2, ivL1) * (@\\
\mbox{}\verb@   + @\hbox{$\langle\,${\itshape expression for massive $\varepsilon^\ast(k, -1)$}\nobreak\ {\footnotesize \NWlink{nuweb11b}{11b}}$\,\rangle$}\verb@ *@\\
\mbox{}\verb@     NCContainer(USpb(k1,-1), ivL1)@\\
\mbox{}\verb@   + sqrt2 * @\hbox{$\langle\,${\itshape expression for massive $\varepsilon^\ast(k, 0)$}\nobreak\ {\footnotesize \NWlink{nuweb11c}{11c}}$\,\rangle$}\verb@ *@\\
\mbox{}\verb@     NCContainer(USpa(k1,-1), ivL1));@\\
\mbox{}\verb@   @\hbox{\sffamily\bfseries Sum}\verb@ ivL2, ivL1;@\\
\mbox{}\verb@@\hbox{\sffamily\bfseries Id}\verb@ @\hbox{\sffamily\bfseries once}\verb@ inplorentz(3, ivL?, k1?, m?!{0,}) *@\\
\mbox{}\verb@      inp(field1?, k1?, +1, k2?, k3?) = 1/sqrt3 *@\\
\mbox{}\verb@   SplitLorentzIndex(ivL, ivL2, ivL1) * (@\\
\mbox{}\verb@   + @\hbox{$\langle\,${\itshape expression for massive $\varepsilon^\ast(k, +1)$}\nobreak\ {\footnotesize \NWlink{nuweb11a}{11a}}$\,\rangle$}\verb@ *@\\
\mbox{}\verb@     NCContainer(USpa(k1,-1), ivL1)@\\
\mbox{}\verb@   + sqrt2 * @\hbox{$\langle\,${\itshape expression for massive $\varepsilon^\ast(k, 0)$}\nobreak\ {\footnotesize \NWlink{nuweb11c}{11c}}$\,\rangle$}\verb@ *@\\
\mbox{}\verb@     NCContainer(USpb(k1,-1), ivL1));@\\
\mbox{}\verb@   @\hbox{\sffamily\bfseries Sum}\verb@ ivL2, ivL1;@\\
\mbox{}\verb@@\hbox{\sffamily\bfseries Id}\verb@ @\hbox{\sffamily\bfseries once}\verb@ inplorentz(3, ivL?, k1?, m?!{0,}) *@\\
\mbox{}\verb@      inp(field1?, k1?, +2, k2?, k3?) =@\\
\mbox{}\verb@   SplitLorentzIndex(ivL, ivL2, ivL1) *@\\
\mbox{}\verb@   @\hbox{$\langle\,${\itshape expression for massive $\varepsilon^\ast(k, +1)$}\nobreak\ {\footnotesize \NWlink{nuweb11a}{11a}}$\,\rangle$}\verb@ *@\\
\mbox{}\verb@   NCContainer(USpb(k1,-1), ivL1);@\\
\mbox{}\verb@   @\hbox{\sffamily\bfseries Sum}\verb@ ivL2, ivL1;@\\
\mbox{}\verb@@{\NWsep}
\end{list}
\vspace{-1.5ex}
\footnotesize
\begin{list}{}{\setlength{\itemsep}{-\parsep}\setlength{\itemindent}{-\leftmargin}}
\item \NWtxtMacroRefIn\ \NWlink{nuweb13}{13}.

\item{}
\end{list}
\end{minipage}\vspace{4ex}
\end{flushleft}
%---#] Spin-3/2 Particles:

\section{Spin-2 Particles}
\subsection{Tensor Structure}
In order to map the pair of Lorentz indices into a single Multi-Index
we use the function \texttt{SplitLorentzIndex}; the first argument denotes
the multi-index, the second and the last argument are the two Lorentz indices.
%---#[ Spin-2 Particles:
\subsection{Tensor Ghost}
The CalcHEP way of treating colour requires the introduction of
the so-called \textit{tensor ghost}. This auxilliary field is introduced
in order to split the four-gluon vertex into a pair of gluon-gluon-ghost
vertices. The propagator therefore is not dynamical and has the form
\begin{equation}
P(T^{\mu_1\nu_1}(p_1),T^{\mu_2\nu_2}(p_2))=-ig^{\mu_1\mu_2}g^{\nu_1\nu_2}
\end{equation}
Tensor ghosts are indicated by having an auxilliary field value of~1.
\begin{flushleft} \small
\begin{minipage}{\linewidth}\label{scrap37}\raggedright\small
\NWtarget{nuweb18a}{} $\langle\,${\itshape tensor ghost propagator}\nobreak\ {\footnotesize {18a}}$\,\rangle\equiv$
\vspace{-1ex}
\begin{list}{}{} \item
\mbox{}\verb@@\hbox{\sffamily\bfseries Id}\verb@ @\hbox{\sffamily\bfseries once}\verb@ proplorentz(4, k1?, m?, sDUMMY1?, 1, iv1?, iv2?) =@\\
\mbox{}\verb@   - PREFACTOR(@\hbox{\sffamily\bfseries i}\verb@_) *@\\
\mbox{}\verb@     SplitLorentzIndex(iv1, iv1a, iv1b) *@\\
\mbox{}\verb@     SplitLorentzIndex(iv2, iv2a, iv2b) *@\\
\mbox{}\verb@     d(iv1a, iv2a) * d(iv1b, iv2b);@\\
\mbox{}\verb@@\hbox{\sffamily\bfseries Sum}\verb@ iv1a, iv1b, iv2a, iv2b;@{\NWsep}
\end{list}
\vspace{-1.5ex}
\footnotesize
\begin{list}{}{\setlength{\itemsep}{-\parsep}\setlength{\itemindent}{-\leftmargin}}
\item \NWtxtMacroRefIn\ \NWlink{nuweb3}{3}.

\item{}
\end{list}
\end{minipage}\vspace{4ex}
\end{flushleft}
\subsection{Gravitons}

\begin{eqnarray}
\Pi^{\mu\nu,\alpha\beta}(p)&=&
\frac12 \left(g^{\mu\alpha}- \frac{p^\mu p^\alpha}{m^2} \right) \left(g^{\nu\beta}- \frac{p^\nu p^\beta}{m^2} \right)   \nonumber  \\
		    &+& \frac12  \left(g^{\mu\beta}- \frac{p^\mu p^\beta}{m^2} \right) \left(g^{\nu\alpha}- \frac{p^\nu p^\alpha}{m^2} \right)
     \nonumber  \\
      &-&\frac{1}{3} \left(g^{\mu\nu}- \frac{p^\mu p^\nu}{m^2} \right) \left(g^{\alpha\beta}- \frac{p^\alpha p^\beta}{m^2} \right)
\end{eqnarray}

\begin{flushleft} \small
\begin{minipage}{\linewidth}\label{scrap38}\raggedright\small
\NWtarget{nuweb18b}{} $\langle\,${\itshape graviton propagator}\nobreak\ {\footnotesize {18b}}$\,\rangle\equiv$
\vspace{-1ex}
\begin{list}{}{} \item
\mbox{}\verb@@\hbox{\sffamily\bfseries Id}\verb@ @\hbox{\sffamily\bfseries once}\verb@ proplorentz(4, k1?, m?, sDUMMY1?, 0, iv1?, iv2?) =@\\
\mbox{}\verb@   SplitLorentzIndex(iv1, iv1a, iv1b) *@\\
\mbox{}\verb@   SplitLorentzIndex(iv2, iv2a, iv2b) *@\\
\mbox{}\verb@   (@\\
\mbox{}\verb@        + 1/2 * (@\\
\mbox{}\verb@                + (d_(iv1a, iv2a) - k1(iv1a)*k1(iv2a)/m^2) *@\\
\mbox{}\verb@                  (d_(iv1b, iv2b) - k1(iv1b)*k1(iv2b)/m^2)@\\
\mbox{}\verb@                + (d_(iv1a, iv2b) - k1(iv1a)*k1(iv2b)/m^2) *@\\
\mbox{}\verb@                  (d_(iv1b, iv2a) - k1(iv1b)*k1(iv2a)/m^2)@\\
\mbox{}\verb@        )@\\
\mbox{}\verb@        - 1/3 * (@\\
\mbox{}\verb@                + (d_(iv1a, iv1b) - k1(iv1a)*k1(iv1b)/m^2) *@\\
\mbox{}\verb@                  (d_(iv2a, iv2b) - k1(iv2a)*k1(iv2b)/m^2)@\\
\mbox{}\verb@        )@\\
\mbox{}\verb@   ) * inv(k1, m, sDUMMY1);@\\
\mbox{}\verb@@{\NWsep}
\end{list}
\vspace{-1.5ex}
\footnotesize
\begin{list}{}{\setlength{\itemsep}{-\parsep}\setlength{\itemindent}{-\leftmargin}}
\item \NWtxtMacroRefIn\ \NWlink{nuweb3}{3}.

\item{}
\end{list}
\end{minipage}\vspace{4ex}
\end{flushleft}
\begin{eqnarray}
\epsilon^{\mu\nu}_{\pm2}(p) &=& \epsilon_\pm^\mu(p)\epsilon_\pm^\nu(p)\\
\epsilon^{\mu\nu}_{\pm1}(p) &=& \frac1{\sqrt2}\left(
	\epsilon_\pm^\mu(p)\epsilon_0^\nu(p)
       +\epsilon_\pm^\nu(p)\epsilon_0^\mu(p)\right)\\
\epsilon^{\mu\nu}_0(p) &=&\frac1{\sqrt6}\left(
        \epsilon_+^\mu(p)\epsilon_-^\nu(p)
       +\epsilon_-^\mu(p)\epsilon_+^\nu(p)
       +2\epsilon_0^\mu(p)\epsilon_0^\nu(p)\right)
\end{eqnarray}


There is the possibility, to have custom propagators for Spin-2 particles.
This is especially needed for the sum over KK modes in the ADD and RS model.
For this the \texttt{customSpin2Prop} Fortran function need to be defined by hand.
The mass terms in the Lorentz structure are neglected.

\begin{flushleft} \small
\begin{minipage}{\linewidth}\label{scrap39}\raggedright\small
\NWtarget{nuweb19a}{} $\langle\,${\itshape graviton effective propagator}\nobreak\ {\footnotesize {19a}}$\,\rangle\equiv$
\vspace{-1ex}
\begin{list}{}{} \item
\mbox{}\verb@@\hbox{\sffamily\bfseries Id}\verb@ @\hbox{\sffamily\bfseries once}\verb@ proplorentz(4, k1?, m?, sDUMMY1?, 2, iv1?, iv2?) =@\\
\mbox{}\verb@   SplitLorentzIndex(iv1, iv1a, iv1b) *@\\
\mbox{}\verb@   SplitLorentzIndex(iv2, iv2a, iv2b) *@\\
\mbox{}\verb@   (@\\
\mbox{}\verb@        + 1/2 * (@\\
\mbox{}\verb@                + (d_(iv1a, iv2a) ) *@\\
\mbox{}\verb@                  (d_(iv1b, iv2b) )@\\
\mbox{}\verb@                + (d_(iv1a, iv2b) ) *@\\
\mbox{}\verb@                  (d_(iv1b, iv2a) )@\\
\mbox{}\verb@        )@\\
\mbox{}\verb@        - 1/3 * (@\\
\mbox{}\verb@                + (d_(iv1a, iv1b) ) *@\\
\mbox{}\verb@                  (d_(iv2a, iv2b) )@\\
\mbox{}\verb@        )@\\
\mbox{}\verb@   ) * customSpin2Prop(k1, m, sDUMMY1);@\\
\mbox{}\verb@@{\NWsep}
\end{list}
\vspace{-1.5ex}
\footnotesize
\begin{list}{}{\setlength{\itemsep}{-\parsep}\setlength{\itemindent}{-\leftmargin}}
\item \NWtxtMacroRefIn\ \NWlink{nuweb3}{3}.

\item{}
\end{list}
\end{minipage}\vspace{4ex}
\end{flushleft}
We have to consider 10 cases:
\begin{itemize}
\item in-/outgoing
\item polarisations $\pm2$, $\pm1$, $0$
\end{itemize}

\begin{flushleft} \small
\begin{minipage}{\linewidth}\label{scrap40}\raggedright\small
\NWtarget{nuweb19b}{} $\langle\,${\itshape graviton wave functions}\nobreak\ {\footnotesize {19b}}$\,\rangle\equiv$
\vspace{-1ex}
\begin{list}{}{} \item
\mbox{}\verb@@\hbox{$\langle\,${\itshape ingoing graviton wave functions}\nobreak\ {\footnotesize \NWlink{nuweb20a}{20a}}$\,\rangle$}\verb@@\\
\mbox{}\verb@@\hbox{$\langle\,${\itshape outgoing graviton wave functions}\nobreak\ {\footnotesize \NWlink{nuweb20c}{20c}}$\,\rangle$}\verb@@{\NWsep}
\end{list}
\vspace{-1.5ex}
\footnotesize
\begin{list}{}{\setlength{\itemsep}{-\parsep}\setlength{\itemindent}{-\leftmargin}}
\item \NWtxtMacroRefIn\ \NWlink{nuweb2b}{2b}.

\item{}
\end{list}
\end{minipage}\vspace{4ex}
\end{flushleft}
Since we have to distinguish two Lorentz indices we use
auxilliary function \texttt{fDUMMY1}
to denote $\epsilon^\mu(k,\lambda)$ and $\epsilon^\ast_\mu(k,\lambda)$
in the ingoing and outgoing case.

\begin{flushleft} \small\label{scrap41}\raggedright\small
\NWtarget{nuweb19c}{} $\langle\,${\itshape generic graviton wave functions}\nobreak\ {\footnotesize {19c}}$\,\rangle\equiv$
\vspace{-1ex}
\begin{list}{}{} \item
\mbox{}\verb@@\hbox{\sffamily\bfseries Id}\verb@ @\hbox{\sffamily\bfseries once}\verb@ @{\tt @}\verb@1lorentz(4, ivL4?, k1?, m?) *@\\
\mbox{}\verb@      @{\tt @}\verb@1(field1?, k1?, +2, k2?, k3?) =@\\
\mbox{}\verb@   SplitLorentzIndex(ivL4, ivL2a, ivL2b) *@\\
\mbox{}\verb@   fDUMMY1(ivL2a, k1, +1, k2, k3, m) * fDUMMY1(ivL2b, k1, +1, k2, k3, m);@\\
\mbox{}\verb@   @\hbox{\sffamily\bfseries Sum}\verb@ ivL2a, ivL2b;@\\
\mbox{}\verb@@\hbox{\sffamily\bfseries Id}\verb@ @\hbox{\sffamily\bfseries once}\verb@ @{\tt @}\verb@1lorentz(4, ivL4?, k1?, m?) *@\\
\mbox{}\verb@      @{\tt @}\verb@1(field1?, k1?, +1, k2?, k3?) =@\\
\mbox{}\verb@   SplitLorentzIndex(ivL4, ivL2a, ivL2b) *@\\
\mbox{}\verb@   1/Sqrt2 * (@\\
\mbox{}\verb@      + fDUMMY1(ivL2a, k1, +1, k2, k3, m) * fDUMMY1(ivL2b, k1,  0, k2, k3, m)@\\
\mbox{}\verb@      + fDUMMY1(ivL2a, k1,  0, k2, k3, m) * fDUMMY1(ivL2b, k1, +1, k2, k3, m)@\\
\mbox{}\verb@   );@\\
\mbox{}\verb@   @\hbox{\sffamily\bfseries Sum}\verb@ ivL2, ivL1;@\\
\mbox{}\verb@@\hbox{\sffamily\bfseries Id}\verb@ @\hbox{\sffamily\bfseries once}\verb@ @{\tt @}\verb@1lorentz(4, ivL4?, k1?, m?) *@\\
\mbox{}\verb@      @{\tt @}\verb@1(field1?, k1?,  0, k2?, k3?) =@\\
\mbox{}\verb@   SplitLorentzIndex(ivL4, ivL2a, ivL2b) *@\\
\mbox{}\verb@   1/Sqrt2/Sqrt3 * (@\\
\mbox{}\verb@      + fDUMMY1(ivL2a, k1, +1, k2, k3, m) * fDUMMY1(ivL2b, k1, -1, k2, k3, m)@\\
\mbox{}\verb@      + fDUMMY1(ivL2a, k1, -1, k2, k3, m) * fDUMMY1(ivL2b, k1, +1, k2, k3, m)@\\
\mbox{}\verb@      + 2 * fDUMMY1(ivL2a, k1, 0, k2, k3, m) * fDUMMY1(ivL2b, k1, 0, k2, k3, m)@\\
\mbox{}\verb@   );@\\
\mbox{}\verb@   @\hbox{\sffamily\bfseries Sum}\verb@ ivL2, ivL1;@\\
\mbox{}\verb@@\hbox{\sffamily\bfseries Id}\verb@ @\hbox{\sffamily\bfseries once}\verb@ @{\tt @}\verb@1lorentz(4, ivL4?, k1?, m?) *@\\
\mbox{}\verb@      @{\tt @}\verb@1(field1?, k1?, -1, k2?, k3?) =@\\
\mbox{}\verb@   SplitLorentzIndex(ivL4, ivL2a, ivL2b) *@\\
\mbox{}\verb@   1/Sqrt2 * (@\\
\mbox{}\verb@      + fDUMMY1(ivL2a, k1, -1, k2, k3, m) * fDUMMY1(ivL2b, k1,  0, k2, k3, m)@\\
\mbox{}\verb@      + fDUMMY1(ivL2a, k1,  0, k2, k3, m) * fDUMMY1(ivL2b, k1, -1, k2, k3, m)@\\
\mbox{}\verb@   );@\\
\mbox{}\verb@   @\hbox{\sffamily\bfseries Sum}\verb@ ivL2, ivL1;@\\
\mbox{}\verb@@\hbox{\sffamily\bfseries Id}\verb@ @\hbox{\sffamily\bfseries once}\verb@ @{\tt @}\verb@1lorentz(4, ivL4?, k1?, m?) *@\\
\mbox{}\verb@      @{\tt @}\verb@1(field1?, k1?, -2, k2?, k3?) =@\\
\mbox{}\verb@   SplitLorentzIndex(ivL4, ivL2a, ivL2b) *@\\
\mbox{}\verb@   fDUMMY1(ivL2a, k1, -1, k2, k3, m) * fDUMMY1(ivL2b, k1, -1, k2, k3, m);@\\
\mbox{}\verb@   @\hbox{\sffamily\bfseries Sum}\verb@ ivL2a, ivL2b;@\\
\mbox{}\verb@@{\NWsep}
\end{list}
\vspace{-1.5ex}
\footnotesize
\begin{list}{}{\setlength{\itemsep}{-\parsep}\setlength{\itemindent}{-\leftmargin}}
\item \NWtxtMacroRefIn\ \NWlink{nuweb20a}{20a}\NWlink{nuweb20c}{c}.

\item{}
\end{list}
\vspace{4ex}
\end{flushleft}
\begin{flushleft} \small
\begin{minipage}{\linewidth}\label{scrap42}\raggedright\small
\NWtarget{nuweb20a}{} $\langle\,${\itshape ingoing graviton wave functions}\nobreak\ {\footnotesize {20a}}$\,\rangle\equiv$
\vspace{-1ex}
\begin{list}{}{} \item
\mbox{}\verb@@\hbox{$\langle\,${\itshape generic graviton wave functions}\nobreak\ ({\footnotesize 20b\label{scrap43}
 }\mbox{}\verb@inp@ ) {\footnotesize \NWlink{nuweb19c}{19c}}$\,\rangle$}\verb@@\\
\mbox{}\verb@@\hbox{\sffamily\bfseries Id}\verb@ fDUMMY1(ivL2?, k1?, +1, k2?, k3?, m?) =@\\
\mbox{}\verb@   @\hbox{$\langle\,${\itshape expression for massive $\varepsilon(k, +1)$}\nobreak\ {\footnotesize \NWlink{nuweb10a}{10a}}$\,\rangle$}\verb@;@\\
\mbox{}\verb@@\hbox{\sffamily\bfseries Id}\verb@ fDUMMY1(ivL2?, k1?, 0, k2?, k3?, m?) =@\\
\mbox{}\verb@   @\hbox{$\langle\,${\itshape expression for massive $\varepsilon(k, 0)$}\nobreak\ {\footnotesize \NWlink{nuweb10c}{10c}}$\,\rangle$}\verb@;@\\
\mbox{}\verb@@\hbox{\sffamily\bfseries Id}\verb@ fDUMMY1(ivL2?, k1?, -1, k2?, k3?, m?) =@\\
\mbox{}\verb@   @\hbox{$\langle\,${\itshape expression for massive $\varepsilon(k, -1)$}\nobreak\ {\footnotesize \NWlink{nuweb10b}{10b}}$\,\rangle$}\verb@;@{\NWsep}
\end{list}
\vspace{-1.5ex}
\footnotesize
\begin{list}{}{\setlength{\itemsep}{-\parsep}\setlength{\itemindent}{-\leftmargin}}
\item \NWtxtMacroRefIn\ \NWlink{nuweb19b}{19b}.

\item{}
\end{list}
\end{minipage}\vspace{4ex}
\end{flushleft}
\begin{flushleft} \small
\begin{minipage}{\linewidth}\label{scrap44}\raggedright\small
\NWtarget{nuweb20c}{} $\langle\,${\itshape outgoing graviton wave functions}\nobreak\ {\footnotesize {20c}}$\,\rangle\equiv$
\vspace{-1ex}
\begin{list}{}{} \item
\mbox{}\verb@@\hbox{$\langle\,${\itshape generic graviton wave functions}\nobreak\ ({\footnotesize 20d\label{scrap45}
 }\mbox{}\verb@out@ ) {\footnotesize \NWlink{nuweb19c}{19c}}$\,\rangle$}\verb@@\\
\mbox{}\verb@@\hbox{\sffamily\bfseries Id}\verb@ fDUMMY1(ivL2?, k1?, +1, k2?, k3?, m?) =@\\
\mbox{}\verb@   @\hbox{$\langle\,${\itshape expression for massive $\varepsilon^\ast(k, +1)$}\nobreak\ {\footnotesize \NWlink{nuweb11a}{11a}}$\,\rangle$}\verb@;@\\
\mbox{}\verb@@\hbox{\sffamily\bfseries Id}\verb@ fDUMMY1(ivL2?, k1?, 0, k2?, k3?, m?) =@\\
\mbox{}\verb@   @\hbox{$\langle\,${\itshape expression for massive $\varepsilon^\ast(k, 0)$}\nobreak\ {\footnotesize \NWlink{nuweb11c}{11c}}$\,\rangle$}\verb@;@\\
\mbox{}\verb@@\hbox{\sffamily\bfseries Id}\verb@ fDUMMY1(ivL2?, k1?, -1, k2?, k3?, m?) =@\\
\mbox{}\verb@   @\hbox{$\langle\,${\itshape expression for massive $\varepsilon^\ast(k, -1)$}\nobreak\ {\footnotesize \NWlink{nuweb11b}{11b}}$\,\rangle$}\verb@;@{\NWsep}
\end{list}
\vspace{-1.5ex}
\footnotesize
\begin{list}{}{\setlength{\itemsep}{-\parsep}\setlength{\itemindent}{-\leftmargin}}
\item \NWtxtMacroRefIn\ \NWlink{nuweb19b}{19b}.

\item{}
\end{list}
\end{minipage}\vspace{4ex}
\end{flushleft}
%---#] Spin-2 Particles:
\section{The Colour Part of the propagators}
%---#[ Colour Propagators :
This section is not at the main theme of this document but for
historical reasons these replacement rules are expected in the
file \texttt{propagators.hh}. The colour part of a propagator
for all non-trivial representations is replaced by a Kronecker-$\delta$.
The trivial representation is just ignored.
\begin{flushleft} \small
\begin{minipage}{\linewidth}\label{scrap46}\raggedright\small
\NWtarget{nuweb21}{} $\langle\,${\itshape colour part of the propagators}\nobreak\ {\footnotesize {21}}$\,\rangle\equiv$
\vspace{-1ex}
\begin{list}{}{} \item
\mbox{}\verb@@\hbox{\sffamily\bfseries Id}\verb@ propcolor( 3, iv1?, iv2?) = @\hbox{\sffamily\bfseries d}\verb@_(iv1, iv2);@\\
\mbox{}\verb@@\hbox{\sffamily\bfseries Id}\verb@ propcolor(-3, iv1?, iv2?) = @\hbox{\sffamily\bfseries d}\verb@_(iv1, iv2);@\\
\mbox{}\verb@@\hbox{\sffamily\bfseries Id}\verb@ propcolor( 8, iv1?, iv2?) = @\hbox{\sffamily\bfseries d}\verb@_(iv1, iv2);@\\
\mbox{}\verb@@\hbox{\sffamily\bfseries Id}\verb@ propcolor( 1, iv1?, iv2?) = 1;@{\NWsep}
\end{list}
\vspace{-1.5ex}
\footnotesize
\begin{list}{}{\setlength{\itemsep}{-\parsep}\setlength{\itemindent}{-\leftmargin}}
\item \NWtxtMacroRefIn\ \NWlink{nuweb3}{3}.

\item{}
\end{list}
\end{minipage}\vspace{4ex}
\end{flushleft}
%---#] Colour Propagators :

\appendix

\section{Index of Symbols}


\section{Index of Macros}

{\small\begin{list}{}{\setlength{\itemsep}{-\parsep}\setlength{\itemindent}{-\leftmargin}}
\item $\langle\,$colour part of the propagators\nobreak\ {\footnotesize \NWlink{nuweb21}{21}}$\,\rangle$ {\footnotesize {\NWtxtRefIn} \NWlink{nuweb3}{3}.}
\item $\langle\,$common header\nobreak\ {\footnotesize \NWlink{nuweb2a}{2a}}$\,\rangle$ {\footnotesize {\NWtxtRefIn} \NWlink{nuweb2b}{2b}\NWlink{nuweb3}{, 3}.
}
\item $\langle\,$expression for $\varepsilon(k, +1)$\nobreak\ {\footnotesize \NWlink{nuweb8b}{8b}}$\,\rangle$ {\footnotesize {\NWtxtRefIn} \NWlink{nuweb9a}{9a}\NWlink{nuweb9b}{b}.
}
\item $\langle\,$expression for $\varepsilon(k, -1)$\nobreak\ {\footnotesize \NWlink{nuweb8c}{8c}}$\,\rangle$ {\footnotesize {\NWtxtRefIn} \NWlink{nuweb8d}{8d}\NWlink{nuweb9b}{, 9b}.
}
\item $\langle\,$expression for $\varepsilon^\ast(k, +1)$\nobreak\ {\footnotesize \NWlink{nuweb8d}{8d}}$\,\rangle$ {\footnotesize {\NWtxtRefIn} \NWlink{nuweb9b}{9b}.}
\item $\langle\,$expression for $\varepsilon^\ast(k, -1)$\nobreak\ {\footnotesize \NWlink{nuweb9a}{9a}}$\,\rangle$ {\footnotesize {\NWtxtRefIn} \NWlink{nuweb9b}{9b}.}
\item $\langle\,$expression for massive $\varepsilon(k, +1)$\nobreak\ {\footnotesize \NWlink{nuweb10a}{10a}}$\,\rangle$ {\footnotesize {\NWtxtRefIn} \NWlink{nuweb11b}{11b}\NWlink{nuweb11d}{d}\NWlink{nuweb14}{, 14}\NWlink{nuweb15}{, 15}\NWlink{nuweb20a}{, 20a}.
}
\item $\langle\,$expression for massive $\varepsilon(k, -1)$\nobreak\ {\footnotesize \NWlink{nuweb10b}{10b}}$\,\rangle$ {\footnotesize {\NWtxtRefIn} \NWlink{nuweb11a}{11a}\NWlink{nuweb11d}{d}\NWlink{nuweb14}{, 14}\NWlink{nuweb15}{, 15}\NWlink{nuweb20a}{, 20a}.
}
\item $\langle\,$expression for massive $\varepsilon(k, 0)$\nobreak\ {\footnotesize \NWlink{nuweb10c}{10c}}$\,\rangle$ {\footnotesize {\NWtxtRefIn} \NWlink{nuweb11c}{11c}\NWlink{nuweb11d}{d}\NWlink{nuweb14}{, 14}\NWlink{nuweb15}{, 15}\NWlink{nuweb20a}{, 20a}.
}
\item $\langle\,$expression for massive $\varepsilon^\ast(k, +1)$\nobreak\ {\footnotesize \NWlink{nuweb11a}{11a}}$\,\rangle$ {\footnotesize {\NWtxtRefIn} \NWlink{nuweb11d}{11d}\NWlink{nuweb16}{, 16}\NWlink{nuweb17}{, 17}\NWlink{nuweb20c}{, 20c}.
}
\item $\langle\,$expression for massive $\varepsilon^\ast(k, -1)$\nobreak\ {\footnotesize \NWlink{nuweb11b}{11b}}$\,\rangle$ {\footnotesize {\NWtxtRefIn} \NWlink{nuweb11d}{11d}\NWlink{nuweb16}{, 16}\NWlink{nuweb17}{, 17}\NWlink{nuweb20c}{, 20c}.
}
\item $\langle\,$expression for massive $\varepsilon^\ast(k, 0)$\nobreak\ {\footnotesize \NWlink{nuweb11c}{11c}}$\,\rangle$ {\footnotesize {\NWtxtRefIn} \NWlink{nuweb11d}{11d}\NWlink{nuweb16}{, 16}\NWlink{nuweb17}{, 17}\NWlink{nuweb20c}{, 20c}.
}
\item $\langle\,$fermion propagator\nobreak\ {\footnotesize \NWlink{nuweb5a}{5a}}$\,\rangle$ {\footnotesize {\NWtxtRefIn} \NWlink{nuweb3}{3}.}
\item $\langle\,$gauge boson propagator\nobreak\ {\footnotesize \NWlink{nuweb8a}{8a}}$\,\rangle$ {\footnotesize {\NWtxtRefIn} \NWlink{nuweb3}{3}.}
\item $\langle\,$gauge boson wave-functions, light-like\nobreak\ {\footnotesize \NWlink{nuweb9b}{9b}}$\,\rangle$ {\footnotesize {\NWtxtRefIn} \NWlink{nuweb2b}{2b}.}
\item $\langle\,$gauge boson wave-functions, massive\nobreak\ {\footnotesize \NWlink{nuweb11d}{11d}}$\,\rangle$ {\footnotesize {\NWtxtRefIn} \NWlink{nuweb2b}{2b}.}
\item $\langle\,$generic graviton wave functions\nobreak\ {\footnotesize \NWlink{nuweb19c}{19c}}$\,\rangle$ {\footnotesize {\NWtxtRefIn} \NWlink{nuweb20a}{20a}\NWlink{nuweb20c}{c}.
}
\item $\langle\,$graviton effective propagator\nobreak\ {\footnotesize \NWlink{nuweb19a}{19a}}$\,\rangle$ {\footnotesize {\NWtxtRefIn} \NWlink{nuweb3}{3}.}
\item $\langle\,$graviton propagator\nobreak\ {\footnotesize \NWlink{nuweb18b}{18b}}$\,\rangle$ {\footnotesize {\NWtxtRefIn} \NWlink{nuweb3}{3}.}
\item $\langle\,$graviton wave functions\nobreak\ {\footnotesize \NWlink{nuweb19b}{19b}}$\,\rangle$ {\footnotesize {\NWtxtRefIn} \NWlink{nuweb2b}{2b}.}
\item $\langle\,$handed fermion propagator\nobreak\ {\footnotesize \NWlink{nuweb5b}{5b}}$\,\rangle$ {\footnotesize {\NWtxtRefIn} \NWlink{nuweb3}{3}.}
\item $\langle\,$implementation of Equation~\eqref{eq:m0spinors:1}\nobreak\ {\footnotesize \NWlink{nuweb6a}{6a}}$\,\rangle$ {\footnotesize {\NWtxtRefIn} \NWlink{nuweb5c}{5c}.}
\item $\langle\,$implementation of Equation~\eqref{eq:m0spinors:2}\nobreak\ {\footnotesize \NWlink{nuweb6b}{6b}}$\,\rangle$ {\footnotesize {\NWtxtRefIn} \NWlink{nuweb5c}{5c}.}
\item $\langle\,$implementation of Equation~\eqref{eq:m0spinors:3}\nobreak\ {\footnotesize \NWlink{nuweb6c}{6c}}$\,\rangle$ {\footnotesize {\NWtxtRefIn} \NWlink{nuweb5c}{5c}.}
\item $\langle\,$implementation of Equation~\eqref{eq:m0spinors:4}\nobreak\ {\footnotesize \NWlink{nuweb6d}{6d}}$\,\rangle$ {\footnotesize {\NWtxtRefIn} \NWlink{nuweb5c}{5c}.}
\item $\langle\,$implementation of Equation~\eqref{eq:Mspinors:1}\nobreak\ {\footnotesize \NWlink{nuweb7b}{7b}}$\,\rangle$ {\footnotesize {\NWtxtRefIn} \NWlink{nuweb7a}{7a}.}
\item $\langle\,$implementation of Equation~\eqref{eq:Mspinors:2}\nobreak\ {\footnotesize \NWlink{nuweb7c}{7c}}$\,\rangle$ {\footnotesize {\NWtxtRefIn} \NWlink{nuweb7a}{7a}.}
\item $\langle\,$implementation of Equation~\eqref{eq:Mspinors:3}\nobreak\ {\footnotesize \NWlink{nuweb7d}{7d}}$\,\rangle$ {\footnotesize {\NWtxtRefIn} \NWlink{nuweb7a}{7a}.}
\item $\langle\,$implementation of Equation~\eqref{eq:Mspinors:4}\nobreak\ {\footnotesize \NWlink{nuweb7e}{7e}}$\,\rangle$ {\footnotesize {\NWtxtRefIn} \NWlink{nuweb7a}{7a}.}
\item $\langle\,$ingoing graviton wave functions\nobreak\ {\footnotesize \NWlink{nuweb20a}{20a}}$\,\rangle$ {\footnotesize {\NWtxtRefIn} \NWlink{nuweb19b}{19b}.}
\item $\langle\,$ingoing vector-spinor anti-particle\nobreak\ {\footnotesize \NWlink{nuweb15}{15}}$\,\rangle$ {\footnotesize {\NWtxtRefIn} \NWlink{nuweb13}{13}.}
\item $\langle\,$ingoing vector-spinor particle\nobreak\ {\footnotesize \NWlink{nuweb14}{14}}$\,\rangle$ {\footnotesize {\NWtxtRefIn} \NWlink{nuweb13}{13}.}
\item $\langle\,$outgoing graviton wave functions\nobreak\ {\footnotesize \NWlink{nuweb20c}{20c}}$\,\rangle$ {\footnotesize {\NWtxtRefIn} \NWlink{nuweb19b}{19b}.}
\item $\langle\,$outgoing vector-spinor anti-particle\nobreak\ {\footnotesize \NWlink{nuweb17}{17}}$\,\rangle$ {\footnotesize {\NWtxtRefIn} \NWlink{nuweb13}{13}.}
\item $\langle\,$outgoing vector-spinor particle\nobreak\ {\footnotesize \NWlink{nuweb16}{16}}$\,\rangle$ {\footnotesize {\NWtxtRefIn} \NWlink{nuweb13}{13}.}
\item $\langle\,$scalar propagator\nobreak\ {\footnotesize \NWlink{nuweb4b}{4b}}$\,\rangle$ {\footnotesize {\NWtxtRefIn} \NWlink{nuweb3}{3}.}
\item $\langle\,$scalar wave-functions\nobreak\ {\footnotesize \NWlink{nuweb4a}{4a}}$\,\rangle$ {\footnotesize {\NWtxtRefIn} \NWlink{nuweb2b}{2b}.}
\item $\langle\,$tensor ghost propagator\nobreak\ {\footnotesize \NWlink{nuweb18a}{18a}}$\,\rangle$ {\footnotesize {\NWtxtRefIn} \NWlink{nuweb3}{3}.}
\item $\langle\,$vector-spinor propagators\nobreak\ {\footnotesize \NWlink{nuweb12}{12}}$\,\rangle$ {\footnotesize {\NWtxtRefIn} \NWlink{nuweb3}{3}.}
\item $\langle\,$vector-spinor wave functions\nobreak\ {\footnotesize \NWlink{nuweb13}{13}}$\,\rangle$ {\footnotesize {\NWtxtRefIn} \NWlink{nuweb2b}{2b}.}
\item $\langle\,$wave-functions for massive spinors\nobreak\ {\footnotesize \NWlink{nuweb7a}{7a}}$\,\rangle$ {\footnotesize {\NWtxtRefIn} \NWlink{nuweb2b}{2b}.}
\item $\langle\,$wave-functions for massless spinors\nobreak\ {\footnotesize \NWlink{nuweb5c}{5c}}$\,\rangle$ {\footnotesize {\NWtxtRefIn} \NWlink{nuweb2b}{2b}.}
\end{list}}

\section{Index of Files}

{\small\begin{list}{}{\setlength{\itemsep}{-\parsep}\setlength{\itemindent}{-\leftmargin}}
\item \verb@"legs.hh"@ {\footnotesize {\NWtxtDefBy} \NWlink{nuweb2b}{2b}.}
\item \verb@"propagators.hh"@ {\footnotesize {\NWtxtDefBy} \NWlink{nuweb3}{3}.}
\end{list}}

\begin{thebibliography}{[8]}
\bibitem{1}M.~B\"ohm, A.~Denner, H.~Joos: \textit{GAUGE THEORIES
of the Strong and Electroweak Interaction}, B.~G.~Teubner, Stuttgart,
Leipzig, Wiesbaden, 3rd edition, 2001.
\bibitem{Xu}
  Z.~Xu, D.~H.~Zhang and L.~Chang,
  \textit{Helicity Amplitudes for Multiple Bremsstrahlung
  in Massless Nonabelian Gauge Theories},
  Nucl.\ Phys.\  B {\bf 291} (1987) 392.
  %%CITATION = NUPHA,B291,392;%%
%\cite{Kilian:2007gr}
\bibitem{Kilian:2007gr}
  W.~Kilian, T.~Ohl and J.~Reuter,
  \textit{WHIZARD: Simulating Multi-Particle Processes at LHC and ILC},
  arXiv:0708.4233 [hep-ph].
  %%CITATION = ARXIV:0708.4233;%%
\end{thebibliography}
\end{document}
