[% ' vim: syntax=golem:ts=3:sw=3:expandtab
%]\documentclass[a4paper]{article}
\usepackage{axodraw}
\usepackage{longtable}
\usepackage{listings}
\usepackage{amsmath}
\usepackage{makeidx}
\usepackage{hyperref}
\usepackage{subcaption}
\usepackage{graphicx}
[% @if extension dot2tex %]
\usepackage[x11names, rgb]{xcolor}
\usepackage[utf8]{inputenc}
\usepackage{tikz}
\usetikzlibrary{decorations.pathmorphing,decorations.markings,arrows,shapes}
\usetikzlibrary{arrows}
[% @end @if %]


\newcommand{\bra}[1]{\langle #1 \vert}
\newcommand{\brb}[1]{[ #1 \vert}
\newcommand{\kea}[1]{\vert #1 \rangle}
\newcommand{\keb}[1]{\vert #1 ]}
\newcommand{\Spaa}[1]{\langle #1 \rangle}
\newcommand{\Spab}[1]{\langle #1]}
\newcommand{\Spba}[1]{[ #1 \rangle}
\newcommand{\Spbb}[1]{[ #1 ]}

\allowdisplaybreaks[1]

\title{\texttt{[% golem.full-name %]}: $[%
   @for particles initial %]{[%latex%]}[%
   @end @for %]\rightarrow[%
   @for particles final %]{[%latex%]}[%
   @end @for %]$}
\author{[%user_name%]}
\date{[% time_stamp format=%F^(%H:%M:%S) space=^%]}

\renewcommand{\indexname}{Index of all Loop Diagrams}

\makeindex
\begin{document}
[% @if extension dot2tex %]
% Line decoration for smdiag

% quarks
\tikzstyle{U}=[decoration={markings,mark=at position 0.5 with \arrow{>}},postaction=decorate]
\tikzstyle{Ubar}=[decoration={markings,mark=at position 0.5 with \arrow{<}},postaction=decorate]
\tikzstyle{D}=[decoration={markings,mark=at position 0.5 with \arrow{>}},postaction=decorate]
\tikzstyle{Dbar}=[decoration={markings,mark=at position 0.5 with \arrow{<}},postaction=decorate]
\tikzstyle{C}=[decoration={markings,mark=at position 0.5 with \arrow{>}},postaction=decorate]
\tikzstyle{Cbar}=[decoration={markings,mark=at position 0.5 with \arrow{<}},postaction=decorate]
\tikzstyle{S}=[decoration={markings,mark=at position 0.5 with \arrow{>}},postaction=decorate]
\tikzstyle{Sbar}=[decoration={markings,mark=at position 0.5 with \arrow{<}},postaction=decorate]
\tikzstyle{T}=[decoration={markings,mark=at position 0.5 with \arrow{>}},postaction=decorate]
\tikzstyle{Tbar}=[decoration={markings,mark=at position 0.5 with \arrow{<}},postaction=decorate]
\tikzstyle{B}=[decoration={markings,mark=at position 0.5 with \arrow{>}},postaction=decorate]
\tikzstyle{Bbar}=[decoration={markings,mark=at position 0.5 with \arrow{<}},postaction=decorate]

% gluon and its ghost
\tikzstyle{gh}=[dotted, decoration={markings,mark=at position 0.5 with \arrow{>}},postaction=decorate]
\tikzstyle{ghbar}=[dotted, decoration={markings,mark=at position 0.5 with \arrow{<}},postaction=decorate]
\tikzstyle{g}=[decorate, decoration={coil, amplitude=4pt, segment length=6pt, aspect=0.95}]

% V: EWSM gauge bosons
\tikzstyle{A}=[decorate,decoration={snake}]
\tikzstyle{Wm}=[decorate,decoration={snake}]
\tikzstyle{Wp}=[decorate,decoration={snake}]
\tikzstyle{Z}=[decorate,decoration={snake}]

% leptons
\tikzstyle{em}=[decoration={markings,mark=at position 0.5 with \arrow{>}},postaction=decorate]
\tikzstyle{ep}=[decoration={markings,mark=at position 0.5 with \arrow{<}},postaction=decorate]
\tikzstyle{ne}=[decoration={markings,mark=at position 0.5 with \arrow{>}},postaction=decorate]
\tikzstyle{nebar}=[decoration={markings,mark=at position 0.5 with \arrow{<}},postaction=decorate]
\tikzstyle{mum}=[decoration={markings,mark=at position 0.5 with \arrow{>}},postaction=decorate]
\tikzstyle{mup}=[decoration={markings,mark=at position 0.5 with \arrow{<}},postaction=decorate]
\tikzstyle{nmu}=[decoration={markings,mark=at position 0.5 with \arrow{>}},postaction=decorate]
\tikzstyle{nmubar}=[decoration={markings,mark=at position 0.5 with \arrow{<}},postaction=decorate]
\tikzstyle{taum}=[decoration={markings,mark=at position 0.5 with \arrow{>}},postaction=decorate]
\tikzstyle{taup}=[decoration={markings,mark=at position 0.5 with \arrow{<}},postaction=decorate]
\tikzstyle{ntau}=[decoration={markings,mark=at position 0.5 with \arrow{>}},postaction=decorate]
\tikzstyle{ntaubar}=[decoration={markings,mark=at position 0.5 with \arrow{<}},postaction=decorate]

% G: Faddeev-Popov Ghosts
\tikzstyle{ghA}=[dotted, decoration={markings,mark=at position 0.5 with \arrow{>}},postaction=decorate]
\tikzstyle{ghAbar}=[decoration={markings,mark=at position 0.5 with \arrow{<}},postaction=decorate]
\tikzstyle{ghZ}=[dotted, decoration={markings,mark=at position 0.5 with \arrow{>}},postaction=decorate]
\tikzstyle{ghZbar}=[decoration={markings,mark=at position 0.5 with \arrow{<}},postaction=decorate]
\tikzstyle{ghWm}=[dotted, decoration={markings,mark=at position 0.5 with \arrow{>}},postaction=decorate]
\tikzstyle{ghWmbar}=[decoration={markings,mark=at position 0.5 with \arrow{<}},postaction=decorate]
\tikzstyle{ghWp}=[dotted, decoration={markings,mark=at position 0.5 with \arrow{>}},postaction=decorate]
\tikzstyle{ghWpbar}=[dotted, decoration={markings,mark=at position 0.5 with \arrow{>}},postaction=decorate]

% S: scalars
\tikzstyle{chi}=[dashed]
\tikzstyle{phim}=[dashed]
\tikzstyle{phip}=[dashed]
\tikzstyle{H}=[dashed]
[% @end @if %]


\maketitle
\begin{abstract}
\noindent This process consists of [% @select topolopy.count.tree
 @case 0 %]no LO diagrams[%
 @case 1 %]one LO diagram[%
 @else %][% topolopy.count.tree %] tree-level diagrams[%
 @end @select %][% @for loops_generated %] There are [% @select loop.count.docu
 @case 0 %] no N[%loop%]LO diagrams. [%
 @case 1 %] one N[%loop%]LO diagram. [%
 @else %] [% loop.count.docu %] N[%loop%]LO diagrams. [%
 @end @select %] [% @end @for %] GoSam has identified [% @select count groups
    @case 0%]no different groups[%
    @case 1%]only one group[%
    @else %][% count groups %] groups [%
 @end @select %] of NLO diagrams by analyzing their one-loop integrals.

\end{abstract}
\newpage
\tableofcontents
\newpage

\section{Helicities}

\begin{longtable}[c]{r|[%
   @for particles %]c[%
   @end @for particles %]}
\bf{Index} [%
   @for particles %]&[%index%][%
   @end @for particles %]\\
\hline
\endfirsthead
\bf{Index} [%
   @for particles %]&[%index%][%
   @end @for particles %]\\
\hline
\endhead [%
@for helicities %]
$[%helicity%][%
   @if generated %][%
   @else %]\rightarrow [% map.index %][%
   @end @if %]$[% @for particles %]& $[%hel%]$[%
@end @for particles %]\\[% @end @for helicities %]
\end{longtable}
\section{Wave Functions}
In this section, we use $l_i=k_i$ for massless particles;
in spinors $\kea{i}$ (resp. $\keb{i}$) denote $\kea{l_i}$ (resp. $\keb{l_i}$).[%
@for instructions %][%
   @if is_first %]
For the massive particles we have:
\begin{align}[%
   @else %]\\[%
   @end @if %][%

   @select opcode
   @case 1 %]
l_{[% index1 %]} &= k_{[% index1%]} - \frac{[% mass1 %]^2}{%
      2 k_{[%index1%]}\cdot [%
      @select mass2 
      @case 0%]k_{[%index2%]}[%
      @else %]l_{[%index2%]}[%
      @end @select mass2 %]}[%

      @select mass2 
      @case 0%]k_{[%index2%]}[%
      @else %]l_{[%index2%]}[%
      @end @select mass2 %][%
   @case 2 %]
l_{[% index1 %]/[% index2 %]} &= k_{[% index1%]}
   +\left[-\frac{k_{[%index1%]}\cdot k_{[%index2%]}}{m_{[%index2%]}^2}\pm%
    \sqrt{
\left(\frac{k_{[%index1%]}\cdot k_{[%index2%]}}{m_{[%index2%]}^2}\right)^2
-\frac{m_{[%index1%]}^2}{m_{[%index2%]}^2}}\right]k_{[%index2%]}[%
   @end @select opcode %][%
   @if is_last %]
\end{align}[%
   @end @if %][%
@end @for instructions%]

All helicity amplitudes are defined in terms of the following wave functions:
\begin{itemize}[%

@for particles initial %]
\item $[%latex%](k_{[%index%]})$ [%
   @select 2spin
   @case 0 %]
% scalar incoming particle
\begin{align}
\epsilon(k_{[% index %]}) &= 1
\end{align}[%
   @case 1 %][%
      @if is_massive %]
% massive incoming quark or lepton
\begin{align}
u_+(k_{[%index%]}) &= \kea{[%index%]}+\frac{m_{[%index%]}}{%
        \Spbb{[%index%]\vert [%eval .abs. reference%]}}\keb{[%
           eval .abs. reference%]}\\
u_-(k_{[%index%]}) &= \keb{[%index%]}+\frac{m_{[%index%]}}{%
        \Spaa{[%index%]\vert [%eval .abs. reference%]}}\kea{[%
           eval .abs. reference%]}
\end{align}[%
      @else %]
% massless incoming quark or lepton
\begin{align}
u_+(k_{[%index%]}) &= \kea{[%index%]}\\
u_-(k_{[%index%]}) &= \keb{[%index%]}
\end{align}[%
      @end @if %][%
   @case -1 %][%
      @if is_massive %]
% massive incoming anti-quark or anti-lepton
\begin{align}
\bar{v}_+(k_{[%index%]}) &= \bra{[%index%]}-\frac{m_{[%index%]}}{%
        \Spbb{[%eval .abs. reference%]\vert [%index%]}}\brb{[%
            eval .abs. reference%]}\\
\bar{v}_-(k_{[%index%]}) &= \brb{[%index%]}-\frac{m_{[%index%]}}{%
        \Spaa{[%eval .abs. reference%]\vert [%index%]}}\bra{[%
            eval .abs. reference%]}
\end{align}[%
      @else %]
% massless incoming anti-quark or anti-lepton
\begin{align}
\bar{v}_+(k_{[%index%]}) &= \bra{[%index%]}\\
\bar{v}_-(k_{[%index%]}) &= \brb{[%index%]}
\end{align}[%
      @end @if %][%
   @case -2 2 %]
% incoming vector particle
\begin{align}
\varepsilon^\mu_+(k_{[%index%]}) &=
   \frac{\Spab{[% eval .abs. reference%]\vert\gamma^\mu\vert [%index%]}}{%
   \sqrt{2}\Spaa{[%eval .abs. reference%]\vert [%index%]}}\\
\varepsilon^\mu_-(k_{[%index%]}) &=
   \frac{\Spba{[%eval .abs. reference%]\vert\gamma^\mu\vert [%index%]}}{%
   \sqrt{2}\Spbb{[%index%]\vert [%eval .abs. reference%]}}[%
      @if is_massive %]\\
\varepsilon^\mu_0(k_{[%index%]}) &=
   \frac{1}{m_{[%index%]}}\left[l_{[%index%]}^\mu-\frac{m_{[%index%]}^2}{%
          2l_{[%index%]}\cdot l_{[%eval .abs. reference%]}}l_{[%
              eval .abs. reference%]}^\mu\right][%
      @end @if %]
\end{align}[%
   @case 3 %]
For the definition of vector-spinor states, please refer to the document
\texttt{lorentz.pdf}.[%
   @case -3 %]
For the definition of vector-spinor states, please refer to the document
\texttt{lorentz.pdf}.[%
   @case 4 %]
For the definition of tensor states, please refer to the document
\texttt{lorentz.pdf}.[%
   @end @select %][%
@end @for particles %][%


@for particles final %]
\item $[%latex%](k_[%index%])$ [%
   @select 2spin
   @case 0 %]
\begin{align}
% outgoing scalar particle
\epsilon(k_{[% index %]}) &= 1
\end{align}[%
   @case 1 %][%
      @if is_massive %]
% massive outgoin quark or lepton
\begin{align}
\bar{u}_+(k_{[%index%]}) &= \brb{[%index%]}+\frac{m_{[%index%]}}{%
        \Spaa{[%eval .abs. reference%]\vert [%index%]}}\bra{[%
             eval .abs. reference%]}\\
\bar{u}_-(k_{[%index%]}) &= \bra{[%index%]}+\frac{m_{[%index%]}}{%
        \Spbb{[%eval .abs. reference%]\vert [%index%]}}\brb{[%
             eval .abs. reference%]}
\end{align}[%
      @else %]
% massless outgoing quark or lepton
\begin{align}
\bar{u}_+(k_{[%index%]}) &= \brb{[%index%]}\\
\bar{u}_-(k_{[%index%]}) &= \bra{[%index%]}
\end{align}[%
      @end @if %][%
   @case -1 %][%
      @if is_massive %]
% massive outgoing anti-quark or anti-lepton
\begin{align}
v_+(k_{[%index%]}) &= \keb{[%index%]}-\frac{m_{[%index%]}}{%
        \Spaa{[%index%]\vert [%eval .abs. reference%]}}\kea{[%
            eval .abs. reference%]}\\
v_-(k_{[%index%]}) &= \kea{[%index%]}-\frac{m_{[%index%]}}{%
        \Spbb{[%index%]\vert [%eval .abs. reference%]}}\keb{[%
               eval .abs. reference%]}
\end{align}[%
      @else %]
% massless outgoing anti-quark or anti-lepton
\begin{align}
v_+(k_{[%index%]}) &= \keb{[%index%]}\\
v_-(k_{[%index%]}) &= \kea{[%index%]}
\end{align}[%
      @end @if %][%
   @case -2 2 %]
% incoming vector particle
\begin{align}
\varepsilon^\mu_+(k_{[%index%]})^\ast &=
   \frac{\Spba{[%eval .abs. reference%]\vert\gamma^\mu\vert [%index%]}}{%
   \sqrt{2}\Spbb{[%eval .abs. reference%]\vert [%index%]}}\\
\varepsilon^\mu_-(k_{[%index%]})^\ast &=
   \frac{\Spab{[%eval .abs. reference%]\vert\gamma^\mu\vert [%index%]}}{%
   \sqrt{2}\Spaa{[%index%]\vert [%eval .abs. reference%]}}[%
      @if is_massive %]\\
\varepsilon^\mu_0(k_{[%index%]})^\ast &=
   \frac{1}{m_{[%index%]}}\left[l_{[%index%]}^\mu-\frac{m_{[%index%]}^2}{%
          2l_{[%index%]}\cdot l_{[%eval .abs. reference%]}}l_{[%
            eval .abs. reference%]}^\mu\right][%
      @end @if %]
\end{align}[%
   @case 3 %]
For the definition of vector-spinor states, please refer to the document
\texttt{lorentz.pdf}.[%
   @case -3 %]
For the definition of vector-spinor states, please refer to the document
\texttt{lorentz.pdf}.[%
   @case 4 %]
For the definition of tensor states, please refer to the document
\texttt{lorentz.pdf}.[%
   @end @select %][%
@end @for particles %]
\end{itemize}

%------------------------------------------------------------------------
\section{Colour Basis}
\begin{align}[%
@for color_basis index_shift=1 shift=1%]
\vert c_{[% index %]}\rangle &=[%
   @select count color_lines @case 0 %][%
      @select count color_traces @case 0 %]1[%
      @end @select %][%
   @end @select %][%
   @for color_wf shift=1 %][%
      @select color
      @case 3 %]q_{i_{[%gindex%]}}^{([%gindex%])}[%
      @case -3 %]\bar{q}_{i_{[%gindex%]}}^{([%gindex%])}[%
      @case -8 8 %]g^{A_{[%gindex%]}}_{([%gindex%])}[%
      @end @select %][%
   @end @for %][%
   @for color_lines %]\left([%
      @for color_line_elements %]T^{A_{[%gindex%]}}[%
      @end @for %]\right)_{[%
      @select first_rep
      @case -3 3 %]i_{[%first_gidx%]}[%
      @end @select %][%
      @select last_rep
      @case -3 3 %]i_{[%last_gidx%]}[%
      @end @select %]}[%
   @end @for %][%
   @for color_traces %]\textrm{tr}\left\{[%
      @for color_trace_elements %]T^{A_{[%gindex%]}}[%
      @end @for %]\right\}[%
   @end @for %][%
   @if is_last %][% @else %]\\[% @end @if %][%
@end @for %]
\end{align}

[% @if generate_lo_diagrams %]
\section{Tree Diagrams}
\lstinputlisting[title={QGraf Setup},frame=tlrb]{../diagrams-0.log}

[% 
@if pyxodraw %][%
   @with modules pyxotree.tex path=process %][% 
      @for elements topolopy.keep.tree var=DIAG index=DIAGINDEX %][%
         @if is_first %]\begin{longtable}{cc}
\endfirsthead
\endhead[%
         @end @if %]
%---#[ tree diagram[% DIAG %]:
\hbox{
\begin{minipage}{0.45\textwidth}
\begin{center}
[%diagram suffix=DIAG%]\\
{\sl [%
         @select tree_sign DIAG
         @case -1 %]-[%
         @end @select %]Diagram~[%DIAG%]}
\end{center}
\end{minipage}}
%---#] tree diagram[% DIAG %]:[%
         @if is_last %]
\end{longtable}
[%       @else %]
[%       @if eval DIAGINDEX .mod. 2 .eq. 1 %]\\[% @else %]&[%@end @if%][%
         @end @if %][% 
      @end @for %][%
   @end @with%][% 
@end @if pyxodraw %][%
@end @if generate_lo_diagrams %]

[% @if generate_nlo_virt %]
\section{One-Loop Diagrams}
\subsection*{General Information}
\lstinputlisting[title={QGraf Setup},frame=tlrb]{../diagrams-1.log}

Loop diagrams are grouped into sets of diagrams which share
loop-propagators. A loop integral can be written as
\begin{equation}
\int\!\frac{\mathrm{d}^nk}{i\pi^{\frac{n}{2}}}%
\frac{\mathcal{N}(q)}{\prod_{j=1}{N}\left[(k+r_j)^2-(m_j^2
   -i m_j\Gamma_j) + i\delta\right]}.
\end{equation}
For each group we list $r_j$, $m_j$ and $\Gamma_j$.
For $m_j$ and $\Gamma_j$ only non-vanishing symbols are listed.
Furthermore, we give the matrix $S$ which is defined as
\begin{equation}
S_{\alpha\beta} = (r_\alpha-r_\beta)^2
-(m_\alpha^2-im_\alpha\Gamma_\alpha)
-(m_\beta^2-im_\beta\Gamma_\beta).
\end{equation}
For each diagram we denote how the matrix $S^\prime$ for the specific diagram
is obtained from the original~$S$. The notation
\begin{equation}
S^\prime=S_{Q\to q^\prime}^{\{l_1,l_2,\ldots\}}
\end{equation}
means, that the rows and columns labeled by $l_1,l_2,\ldots$ should be
removed from $S$ (likewise $r_{l_1}, r_{l_2}, \ldots$ are removed from the
list of propagators) and $\mathcal{N}(q)$ has to be replaced by
$\mathcal{N}(q^\prime)$.
The maximum effective rank of a group is the rank that has to be passed
to \textsc{Samurai} if the whole group is reduced at once; this number
is calculated as
\begin{equation}
\max_{\text{diagrams}}\left\{(\text{rank of diagram})+
(\text{number of pinches})\right\}.
\end{equation}
Diagrams with massless closed quark lines are multiplied by a factor
$\mathtt{Nfrat}=\mathtt{Nf}/\mathtt{Nfgen}$. This multiplication is indicated
by the symbol $N_f$ following the rank. By default $\mathtt{Nfrat}$ evaluates
to one but can be changed by modifying $\mathtt{Nf}$ or $\mathtt{Nfgen}$ in the
model~file.

[%
@for groups var=grp %]
\subsection{Group~[% grp %] ([% loopsize group=grp %]-Point)}
\subsubsection*{General Information}
The maximum effective rank in this group is~[% rank %].

\begin{equation}
\begin{split}[%
@for propagators group=grp prefix=k_{ suffix=} %]
r_{[% $_ %]} &= [% momentum%][%
   @if eval mass .ne. 0 %],\quad m_{[% $_ %]} = [% latex_parameter mass %][%
      @if eval width .ne. 0 %],\quad \Gamma_{[% $_ %]} = [%
              latex_parameter width %][%
      @end @if %][%
   @end @if %][%
   @if is_last %][%
   @else %]\\[%
   @end @if %][%
@end @for %]
\end{split}
\end{equation}

\begin{equation}
S=[%
@for smat group=grp %][%
   @if is_first %]\left(\begin{array}{[%
      @with eval loopsize group=grp result=ls %][% 
         @for repeat ls %]c[% @end @for %]}[%
      @end @with %][%
   @end @if %][%
   @if is_zero %]
   0[%
   @else %]
   S_{[%rowindex%],[%colindex%]}[%
   @end @if %][%
   @if eol %][%
      @if is_last %]\end{array}\right)[%@else%]\\[%@end @if%][%
   @else %]&[% @end @if %][%
@end @for %]
\end{equation}[%
@for smat group=grp nonzero upper diagonal 
	powfmt=%s^%d
	prodfmt=%s*%s
	prefix=s_{
	suffix=}
	%][%
   @if is_first %]
\begin{equation}
\begin{split}[%
   @end @if %]
   S_{[%rowindex%],[%colindex%]}&=[%
      @for elements re delimiter=; var=term first=first_term %][%
         @for elements term delimiter=: %][%
            @if is_first %][%
               @if eval $_ .ge. 0 %][%
                  @if first_term %][%
                  @else %]+[%
                  @end @if %][%
               @else %]-[%
               @end @if %][%

               @select $_
               @case 2 -2 %][%
               @case 4 -4%]2[%
               @else %]\frac{[% eval .abs. $_ %]}{2}[%
               @end @select %][%
            @else %][% latex_parameter $_ %][%
            @end @if %][%
         @end @for %][%
      @end @for %][%

      @for elements im delimiter=; var=term first=first_term %][%
         @for elements term delimiter=: %][%
            @if is_first %][%
               @if eval $_ .ge. 0 %]+[%
               @else %]-[%
               @end @if %][%

               @select $_
               @case 2 -2 %]i\cdot{}[%
               @case 4 -4%]2i\cdot{}[%
               @else %]\frac{[% eval .abs. $_ %]i}{2}[%
               @end @select %][%
            @else %][% latex_parameter $_ %][%
            @end @if %][%
         @end @for %][%
      @end @for %][%
   @if is_last %]
\end{split}
\end{equation}[%
   @else%]\\[%
   @end @if%][%
@end @for %]

\subsubsection{Diagrams ([% count diagrams group=grp %])}[%
@if pyxodraw %][%
@with modules pyxovirt.tex path=process %][% 
   @for diagrams group=grp var=DIAG index=DIAGINDEX %][%
      @if is_first %]\begin{longtable}{cc}
\endfirsthead
\endhead[% 
      @end @if %]
%---#[ loop diagram[% DIAG %]:
\index{Diagram[% DIAG convert=number format=%010d %]=Diagram [% DIAG %] (Group [% grp %])}
\hbox{
\begin{minipage}{0.45\textwidth}
\begin{center}
[% diagram suffix=DIAG %]\\
{\sl [%
@select diagram_sign @case - -1 %]-[%
@end @select %]Diagram~[%DIAG%]}\\
$S^\prime=S[%
@for elements pinches %][%
   @if is_first %]^{\{[%
   @else %],[%
   @end @if %][% eval $_ + 1%][%
   @if is_last %]\}}[%
   @end @if %][%
@end @for %][%
   @select sign
   @case + %][%
      @select shift
      @case 0 %][%
      @else %]_{Q\to q-([%shift%])}[%
      @end @select %][%
   @case - %]_{Q\to -q[%
      @select shift
      @case 0 %][%
      @else %]-([%shift%])[%
      @end @select %]}[%
   @end @select %]$, $\mathrm{rk}=[%rank%][%
   @if is_nf %], N_f[% 
   @end @if %]$
\end{center}
\end{minipage}}
[%   @if is_last %]
\end{longtable}
[%   @else %][%
        @if eval DIAGINDEX .mod. 2 .eq. 1 %]\\[% @else %]&[%@end @if%][%
     @end @if %]
%---#] loop diagram[% DIAG %]:
[% @end @for %][%
@end @with%][%
@end @if pyxodraw %][%
@end @for groups %][%
 @end @if generate_nlo_virt %]



[% @if generate_nnlo_virt %]
\section{Two-Loop Diagrams}
\subsection*{General Information}
\lstinputlisting[title={QGraf Setup},frame=tlrb]{../diagrams-2.log}
\clearpage[%
@for elements topolopy.keep.n2lo_virt var=DIAG index=DIAGINDEX %]
[% @if extension dot2tex %]
\resizebox{0.4\textwidth}{!}{
\input{dot2virt.[% DIAG %].xdot.tikz}}[%
@else %][% 
@if is_first%]
\begin{figure}
\begin{center}[%
@end @if %]
\begin{subfigure}{0.25\textwidth}
\includegraphics[height=1\textwidth,width=1\textwidth,keepaspectratio]{dot2virt.[% DIAG %].eps}
\end{subfigure} \quad[%
@if eval ( DIAGINDEX + 1 ) .mod. 3 .eq. 0 %]
\end{center}
\end{figure}[%
@end @if %][%
@if eval ( DIAGINDEX + 1 ) .mod. ( 9 + 3 ) .eq. 0 %]
\clearpage [%
@end @if %][%
@if eval ( DIAGINDEX + 1 ) .mod. 3 .eq. 0 %]
\begin{figure}
\begin{center}[% 
@end @if %][%
@if is_last%]
\end{center}
\end{figure}[%
@end @if %][%
@end @if %][%
@end @for %][% 
@end @if generate_nnlo_virt %]


\printindex

\section{Related Work}
If you publish results obtained by using this matrix element code
please cite the appropriate papers in the bibliography of this document.

Scientific publications prepared using the present version of
\textsc{GoSam} or any modified version of it or any code linking to
\textsc{GoSam} or parts of it should make a clear
reference to the publications~\cite{Cullen:2014yla,Cullen:2011ac}.

For graph generation we use QGraf~\cite{Nogueira:1991ex}.
The Feynman diagrams are further processed with the symbolic manipulation
program FORM~\cite{Kuipers:2012rf,Vermaseren:2000nd} using the FORM library
SPINNEY~\cite{Cullen:2010jv}.
The Fortran~90 code is generated using 
FORM~\cite{Kuipers:2012rf,Vermaseren:2000nd}[% @if internal HAGGIES %]
and HAGGIES~\cite{Reiter:2009ts}[% @end @if %].
[% @if extension ninja %]For the reduction of the tensor integrals
the code uses an implementation of the Laurent series expansion 
method~\cite{Mastrolia:2012bu}
from the library Ninja~\cite{Peraro:2014cba}.[% @end @if %]
[% @if extension samurai %]For the reduction of the tensor integrals
the code uses the implementation of the OPP
method~\cite{Ossola:2006us,Ossola:2007bb}
and extensions thereof from the package
SAMURAI~\cite{Mastrolia:2010nb}.[%
@end @if %][% @if extension golem95 %]For
the reduction of the tensor integrals, the code uses the 
package GOLEM95~\cite{Guillet:2013msa,Binoth:2008uq,Cullen:2011kv}.
The tensor coefficients are
obtained using tensorial reconstruction at the integrand
level~\cite{Heinrich:2010ax}.[%
@end @if %]

Please, make sure, you also give credit to the authors of the scalar
loop libraries, if you configured the amplitude code such that it calls
other libraries than the ones mentioned so far. Depending on your
configuration you might use one or more of the following programs for
the evaluation of the scalar integrals:
\begin{itemize}
\item OneLOop~\cite{vanHameren:2010cp},
\item QCDLoop~\cite{Ellis:2007qk}, which uses FF~\cite{vanOldenborgh:1990yc},
\item LoopTools~\cite{Hahn:1998yk}, which uses FF~\cite{vanOldenborgh:1990yc}.
\item GOLEM95~\cite{Binoth:2008uq,Guillet:2013msa} which uses OneLOop~\cite{vanHameren:2010cp}
   and may be configured such that it uses
   LoopTools~\cite{Hahn:1998yk,vanOldenborgh:1990yc}.
\end{itemize}

\begin{thebibliography}{ABC}
%\cite{Cullen:2014yla}
\bibitem{Cullen:2014yla}
  G.~Cullen, H.~van Deurzen, N.~Greiner, G.~Heinrich, G.~Luisoni, P.~Mastrolia, E.~Mirabella and G.~Ossola {\it et al.},
  ``GoSam-2.0: a tool for automated one-loop calculations within the Standard Model and beyond,''
  Eur.\ Phys.\ J.\ C {\bf 74} (2014) 8,  3001
  [\href{http://arxiv.org/abs/1404.7096}{arXiv:1404.7096 [hep-ph]}].
  %%CITATION = ARXIV:1404.7096;%%
%\cite{Cullen:2011ac}
\bibitem{Cullen:2011ac}
  G.~Cullen, N.~Greiner, G.~Heinrich, G.~Luisoni, P.~Mastrolia, G.~Ossola, T.~Reiter and F.~Tramontano,
  ``Automated One-Loop Calculations with GoSam,''
  Eur.\ Phys.\ J.\ C {\bf 72} (2012) 1889
  [\href{http://arxiv.org/abs/1111.2034}{arXiv:1111.2034 [hep-ph]}].
  %%CITATION = ARXIV:1111.2034;%%
%\cite{Nogueira:1991ex}
\bibitem{Nogueira:1991ex}
  P.~Nogueira,
  ``Automatic Feynman graph generation,''
  J.\ Comput.\ Phys.\  {\bf 105} (1993) 279.
  %%CITATION = JCTPA,105,279;%%
%\cite{Kuipers:2012rf}
\bibitem{Kuipers:2012rf}
  J.~Kuipers, T.~Ueda, J.~A.~M.~Vermaseren and J.~Vollinga,
  ``FORM version 4.0,''
  Comput.\ Phys.\ Commun.\  {\bf 184} (2013) 1453
  [\href{http://arxiv.org/abs/1203.6543}{arXiv:1203.6543 [cs.SC]}].
  %%CITATION = ARXIV:1203.6543;%%
%\cite{Vermaseren:2000nd}
\bibitem{Vermaseren:2000nd}
  J.~A.~M.~Vermaseren,
  ``New features of FORM,''
  arXiv:math-ph/0010025.
  %%CITATION = MATH-PH/0010025;%% 
%\cite{Cullen:2010jv}
\bibitem{Cullen:2010jv}
  G.~Cullen, M.~Koch-Janusz and T.~Reiter,
  ``spinney: A Form Library for Helicity Spinors,''
  \href{http://arxiv.org/abs/1008.0803}{arXiv:1008.0803 [hep-ph]}.
  %%CITATION = ARXIV:1008.0803;%%
%\cite{Reiter:2009ts}[%
@if internal HAGGIES %]
\bibitem{Reiter:2009ts}
  T.~Reiter,
  ``Optimising Code Generation with haggies,''
  Comput.\ Phys.\ Commun.\  {\bf 181} (2010) 1301
  [\href{http://arxiv.org/abs/0907.3714}{arXiv:0907.3714 [hep-ph]}].
  %%CITATION = CPHCB,181,1301;%%[%
@end @if %][%
@if extension ninja %]
%\cite{Peraro:2014cba}
\bibitem{Peraro:2014cba}
  T.~Peraro,
  ``Ninja: Automated Integrand Reduction via Laurent Expansion for One-Loop Amplitudes,''
  Comput.\ Phys.\ Commun.\  {\bf 185} (2014) 2771
  [\href{http://arxiv.org/abs/1403.1229}{arXiv:1403.1229 [hep-ph]}].
  %%CITATION = ARXIV:1403.1229;%%
%\cite{Mastrolia:2012bu}
\bibitem{Mastrolia:2012bu}
  P.~Mastrolia, E.~Mirabella and T.~Peraro,
  ``Integrand reduction of one-loop scattering amplitudes through Laurent series expansion,''
  JHEP {\bf 1206} (2012) 095
   [Erratum-ibid.\  {\bf 1211} (2012) 128]
  [\href{http://arxiv.org/abs/1203.0291}{arXiv:1203.0291 [hep-ph]}].
  %%CITATION = ARXIV:1203.0291;%%[%
@end @if %][%
@if extension samurai %]
%\cite{Ossola:2006us}
\bibitem{Ossola:2006us}
  G.~Ossola, C.~G.~Papadopoulos, R.~Pittau,
  ``Reducing full one-loop amplitudes to scalar integrals at the
    integrand level,''
  Nucl.\ Phys.\  {\bf B763 } (2007)  147-169.
  [hep-ph/0609007].
%\cite{Ossola:2007bb}
\bibitem{Ossola:2007bb}
  G.~Ossola, C.~G.~Papadopoulos, R.~Pittau,
  ``Numerical evaluation of six-photon amplitudes,''
  JHEP {\bf 0707 } (2007)  085.
  [\href{http://arxiv.org/abs/0704.1271}{arXiv:0704.1271 [hep-ph]}].
%\cite{Mastrolia:2010nb}
\bibitem{Mastrolia:2010nb}
  P.~Mastrolia, G.~Ossola, T.~Reiter and F.~Tramontano,
  ``Scattering AMplitudes from Unitarity-based Reduction Algorithm at the Integrand-level,''
  JHEP {\bf 1008} (2010) 080
  [\href{http://arxiv.org/abs/1006.0710}{arXiv:1006.0710 [hep-ph]}].
  %%CITATION = ARXIV:1006.0710;%%[%
@end @if %]
%\cite{Guillet:2013msa}
\bibitem{Guillet:2013msa}
  J.~P.~Guillet, G.~Heinrich and J.~F.~von Soden-Fraunhofen,
  ``Tools for NLO automation: extension of the golem95C integral library,''
  Comput.\ Phys.\ Commun.\  {\bf 185} (2014) 1828
  [\href{http://arxiv.org/abs/1312.3887}{arXiv:1312.3887 [hep-ph]}].
  %%CITATION = ARXIV:1312.3887;%%
%\cite{Binoth:2008uq}
\bibitem{Binoth:2008uq}
  T.~Binoth, J.~P.~Guillet, G.~Heinrich, E.~Pilon and T.~Reiter,
  ``Golem95: a numerical program to calculate one-loop tensor integrals with up
  to six external legs,''
  Comput.\ Phys.\ Commun.\  {\bf 180} (2009) 2317
  [\href{http://arxiv.org/abs/0810.0992}{arXiv:0810.0992 [hep-ph]}].
  %%CITATION = CPHCB,180,2317;%%
%\cite{Cullen:2011kv}
\bibitem{Cullen:2011kv}
  G.~Cullen, J.~P.~.Guillet, G.~Heinrich, T.~Kleinschmidt, E.~Pilon, T.~Reiter, M.~Rodgers,
  ``Golem95C: A library for one-loop integrals with complex masses,''
  Comput.\ Phys.\ Commun.\  {\bf 182 } (2011)  2276-2284.
  [\href{http://arxiv.org/abs/1101.5595}{arXiv:1101.5595 [hep-ph]}].
%\cite{vanHameren:2010cp}
\bibitem{vanHameren:2010cp}
  A.~van Hameren,
  ``OneLOop: For the evaluation of one-loop scalar functions,''
  [\href{http://arxiv.org/abs/1007.4716}{arXiv:1007.4716 [hep-ph]}].
%\cite{Ellis:2007qk}
\bibitem{Ellis:2007qk}
  R.~K.~Ellis, G.~Zanderighi,
  ``Scalar one-loop integrals for QCD,''
  JHEP {\bf 0802 } (2008)  002.
  [\href{http://arxiv.org/abs/0712.1851}{arXiv:0712.1851 [hep-ph]}].
%\cite{vanOldenborgh:1990yc}
\bibitem{vanOldenborgh:1990yc}
  G.~J.~van Oldenborgh,
  ``FF: A Package to evaluate one loop Feynman diagrams,''
  Comput.\ Phys.\ Commun.\  {\bf 66 } (1991)  1-15.
%\cite{Hahn:1998yk}
\bibitem{Hahn:1998yk}
  T.~Hahn, M.~Perez-Victoria,
  ``Automatized one loop calculations in four-dimensions and D-dimensions,''
  Comput.\ Phys.\ Commun.\  {\bf 118 } (1999)  153-165.
  [hep-ph/9807565].
%\cite{Heinrich:2010ax}
\bibitem{Heinrich:2010ax}
  G.~Heinrich, G.~Ossola, T.~Reiter, F.~Tramontano,
  ``Tensorial Reconstruction at the Integrand Level,''
  JHEP {\bf 1010 } (2010)  105.
  [\href{http://arxiv.org/abs/1008.2441}{arXiv:1008.2441 [hep-ph]}].
\end{thebibliography}
\end{document}
